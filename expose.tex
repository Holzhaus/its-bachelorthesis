\documentclass[conference]{template/IEEEtran}
\usepackage{blindtext, graphicx}
\usepackage[utf8]{inputenc}
\usepackage[colorlinks]{hyperref}
\usepackage[url=false,backend=biber]{biblatex}
\addbibresource{expose.bib}
\nocite{*}
\newbibmacro{string+url}[1]{%
  \iffieldundef{url}{%
    #1%
  }{%
    \href{\thefield{url}}{#1}%
  }%
}
\DeclareFieldFormat{title}{\usebibmacro{string+url}{\mkbibemph{#1}}}
\DeclareFieldFormat[article,incollection,inproceedings]{title}%
    {\usebibmacro{string+url}{\mkbibquote{#1}}}

\def\contentsname{Inhalt}
\def\listfigurename{Liste der Abbildungen}
\def\listtablename{Liste der Tabellen}
\def\refname{Referenzen}
\def\indexname{Index}
\def\figurename{Abb.}
\def\tablename{TABELLE}
\def\partname{Teil}
\def\appendixname{Anhang}
\def\abstractname{Kurzbeschreibung}
% IEEE specific names
\def\IEEEkeywordsname{Schlagworte}
\def\IEEEproofname{Beweis}

\begin{document}
\title{Sicherheitsanalyse von XML- und JSON-Konvertern}
\author{\IEEEauthorblockN{Jan Holthuis}
\IEEEauthorblockA{Lehrstuhl für Netz-und Datensicherheit\\
Ruhr-Universität Bochum\\
Matrikelnummer 108\,009\,215\,809\\
\texttt{\small jan.holthuis@ruhr-uni-bochum.de}}
}
\maketitle
\begin{abstract}
\blindtext[1]
\end{abstract}
\begin{IEEEkeywords}
XML, JSON, Konversion, Konvertierung, Sicherheit.
\end{IEEEkeywords}
\IEEEpeerreviewmaketitle
\section{Einleitung}
\blindtext
\subsection{Unterüberschrift hier}
\blindtext
\section{Fazit}
\blindtext
\appendices
\section{Beweis der ersten Zonklar-Gleichung}
\blindtext
\section*{Danksagung}
Der Autor dankt...
\ifCLASSOPTIONcaptionsoff
  \newpage
\fi
\printbibliography{}
\end{document}
