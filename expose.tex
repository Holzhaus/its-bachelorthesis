\documentclass[conference]{template/IEEEtran}
\usepackage{blindtext, graphicx}
\usepackage[utf8]{inputenc}
\usepackage[ngerman]{babel}
\usepackage{inconsolata}
\usepackage[T1]{fontenc}
\usepackage{amsmath}
\usepackage[colorlinks]{hyperref}
\usepackage[url=false,backend=biber]{biblatex}
\addbibresource{expose.bib}
\nocite{*}
\newbibmacro{string+url}[1]{%
  \iffieldundef{url}{%
    #1%
  }{%
    \href{\thefield{url}}{#1}%
  }%
}
\DeclareFieldFormat{title}{\usebibmacro{string+url}{\mkbibemph{#1}}}
\DeclareFieldFormat[article,incollection,inproceedings]{title}%
    {\usebibmacro{string+url}{\mkbibquote{#1}}}
\DefineBibliographyStrings{ngerman}{%
   andothers = {{et\,al\adddot}},%
}

\def\contentsname{Inhalt}
\def\listfigurename{Liste der Abbildungen}
\def\listtablename{Liste der Tabellen}
\def\refname{Referenzen}
\def\indexname{Index}
\def\figurename{Abb.}
\def\tablename{TABELLE}
\def\partname{Teil}
\def\appendixname{Anhang}
\def\abstractname{Kurzbeschreibung}
% IEEE specific names
\def\IEEEkeywordsname{Schlagworte}
\def\IEEEproofname{Beweis}

\begin{document}
\title{Sicherheitsanalyse von XML- und JSON-Konvertern}
\author{\IEEEauthorblockN{Jan Holthuis}
\IEEEauthorblockA{Lehrstuhl für Netz-und Datensicherheit\\
Ruhr-Universität Bochum\\
Matrikelnummer 108\,009\,215\,809\\
\texttt{\small jan.holthuis@ruhr-uni-bochum.de}}
}
\maketitle
\begin{abstract}
Bei XML und JSON handelt es sich um -- insbesondere im Mobile- und Web-Bereich
konkurrierende -- menschenlesbare Formate für die Speicherung und den
Austausch hierarchisch strukturierter Daten. Je nach eingesetzter API,
Programmiersprache oder Programmbibliothek kann es dabei sinnvoll sein, die
Daten zunächst in das jeweils andere Format zu überführen. In der geplanten
Bachelorarbeit soll sich mit der verlustfreien Translation von XML-Daten in
JSON-Strukuren befasst werden. Dazu sollen bestehende Verfahren auf Genauigkeit
und Sicherheit hin untersucht werden. Schlussendlich soll in der Arbeit ein
verlustfreies Verfahren zur Übersetzung von XML-Daten in JSON sowie in
Rückrichtung entwickelt und vorgestellt werden.
\end{abstract}
\begin{IEEEkeywords}
XML, JSON, Konversion, Konvertierung, Abbildung, Datenstrukturen, Sicherheit.
\end{IEEEkeywords}
\IEEEpeerreviewmaketitle{}
\section{Einleitung}
\label{sec:intro}
Für den implementationsunabhängigen Austausch von zugleich menschen- als auch
maschinenlesbaren Daten hat sich die \emph{Extensible Markup Language (XML)}
bewährt. In bestimmten Bereichen wie Web-APIs hat die \emph{JavaScript Object
Notation (JSON)} das bewährte XML-Format inzwischen jedoch überflügelt.

Dabei kann neben der spezifischen Situation auch die eingesetzte
Programmiersprache, die Unterstützung durch das zugrundeliegende Framework
oder die persönliche Präferenz des Entwicklers den Ausschlag geben, welches
Format ein Webservice oder eine Programmbibliothek unterstützt.

Zwecks Interoperabilität zwischen verschiedenen Teilen einer Anwendung kann es
daher notwendig werden, Daten von einem Format temporär in das jeweils andere
zu überführen und später wieder in das ursprüngliche Format zu bringen.

Dabei sollen Daten, die von der Anwendungslogik nicht verändert wurden, auch
nach der Rückübersetzung ins Ursprungsformat unverändert bleiben. Damit dies
gewährleistet ist, darf das zugrunde liegende Konversionverfahren keine
Informationen bei der Umwandlung verwerfen -- es muss also verlustlos % chktex 8
arbeiten.

Das konkrete Konversionsverfahren ist dabei abhängig vom jeweiligen
Ausgangsformat, d.~h.\ die Umwandlungsverfahren für die beiden Richtungen
\begin{enumerate}
    \item $\text{JSON} \rightarrow \text{XML} \rightarrow \text{JSON}$ und
    \item $\text{XML} \rightarrow \text{JSON} \rightarrow \text{XML}$
\end{enumerate}
haben jeweils eigene Anforderungen und sind getrennt voneinander zu betrachten.

Während Abbildung von beliebigen JSON-Datenstrukturen in XML zumindest bei
oberflächlicher Betrachtung trivial erscheint, ist dies beim verlustlosen
Transfer von XML-Daten ins JSON-Format keineswegs der Fall.

Daher soll in der geplanten Bachelorarbeit verschiedene Verfahren analysiert
werden, die beliebige XML-Daten in JSON abbilden und aus der resultieren
JSON-Datenstruktur wieder XML-Dokument erstellen können. Sollte keines
der analysierten Verfahren den vorher aufgestellten Kriterien für eine
zuverlässige und sichere Umwandlung genügen, wird ein eigener
Abbildungsalgorithmus entwickelt, bei dem dies der Fall ist.

\section{Motivation}
\label{sec:motivation}
Das Aufkommen des sogenannten \emph{Web 2.0} und die zunehmenden Vernetzung
durch das \emph{Internet of Things (IoT)} ging mit einer erhöhten
Verfügbarkeit von öffentlichen Web-APIs einher. Als Datenformat wird dabei
häufig JSON oder XML verwendet.

Neben XML-basierten Webservices, die beispielsweise das SAML-Framework, SOAP
oder XML-RPC verwenden, wird XML auch in einer Vielzahl weiterer
Einsatzbereiche eingesetzt. So dient XML den Dateiformaten RSS/ASF, MathML,
Scalable Vector Graphics (SVG) oder XHTML als Basis. Auch die gängigen
Office-Dateiformate -- das \emph{Open Document Format}, Microsofts % chktex 8
\emph{Office Open XML} und Apples \emph{iWorks} -- bauen auf XML auf. % chktex 8

Inzwischen gewinnt jedoch \emph{JSON} vor allem im Mobile-
und Web-Bereich immer mehr an Bedeutung. Laut dem API-Verzeichnis
\emph{ProgrammableWeb} unterstützten im Jahr 2013 ca. 60\% aller neu
hinzugefügten APIs das JSON-Format, während XML im selben Zeitraum lediglich
von 37\% der neuen APIs unterstützt wurde.\cite{duvander2013convergence}

JSON ist bei bestimmten Aufgaben in puncto Geschwindigkeit und
Ressourcenauslastung deutlich effizienter\cite{nurseitov2009comparison} als
XML\@. Inzwischen setzen auch einige populäre NoSQL-Datenbanken wie
\emph{CouchDB} oder \emph{MongoDB} auf \emph{JSON} zur Speicherung. Auch
MySQL verfügt seit Version 5.7.8 über einen nativen JSON-Datentyp.

Die Umwandlung zwischen den beiden Formaten kann aus vielen Gründen
notwendig werden. Soll beispielsweise ein SOAP-Webservice als moderne
JSON-REST-Ressource angeboten werden, muss zwischen XML und JSON konvertiert
werden. Auch bei der Speicherung von XML-Daten in den o.g. NoSQL-Datenbanken
ist dies der Fall. Zudem ist die Unterstützung der Formate durch
Programmiersprachen, Frameworks und Applikationen nicht immer gleich gut.

\section{Vorgehensweise}
\label{sec:workingmethod}
Im Verlauf der Arbeit sollen verschiedene generische
XML-zu-JSON-Konversionsverfahren miteinander verglichen werden. Anbieten würden
sich dazu evtl.

\begin{itemize}
    \item die \emph{BadgerFish}-Konvention\footnote{\url{http://www.sklar.com/badgerfish/}},
    \item die \emph{Parker}-Konvention\footnote{\url{https://github.com/open311/xml-tools/tree/master/Parker/xml2json-xslt}},
    \item die \emph{Google Data (GData)}-Konvention\footnote{\url{https://developers.google.com/gdata/docs/json}},
    \item die \emph{JSON Markup Language (JsonML)}\footnote{\url{http://www.jsonml.org/}},
    \item die \emph{x2js}-Bibliothek\footnote{\url{https://github.com/abdmob/x2js}} (JavaScript, Apache 2.0-Lizenz),
    \item die \emph{Apache camel-xmljson}-Bibliothek\footnote{\url{http://camel.apache.org/xmljson.html}} (Java, Apache 2.0-Lizenz),
    \item die \emph{jxon}-Bibliothek\footnote{\url{https://github.com/tyrasd/jxon}} (JavaScript, GNU Public License 3.0) und
    \item die \emph{pesterfish}-Bibliothek\footnote{\url{https://bitbucket.org/smulloni/pesterfish/overview}} (Python, MIT-Lizenz).
\end{itemize}

Auch der \emph{IBM DataPower}-Gateway unterstützt XML/JSON-Umwandlungen, jedoch scheint
nur die Richtung
\[
\text{JSON} \rightarrow
\text{JSONx (XML)\footnote{\url{https://www.ibm.com/support/knowledgecenter/SS9H2Y_7.5.0/com.ibm.dp.doc/web20-json.html}}}
\rightarrow \text{JSON}\]
möglich zu sein. Falls die Umwandlung auch mit dem Ausgangsformat XML
möglich ist, wäre das proprietäre Übersetzungsverfahren des \emph{IBM
DataPower}-Gateways ebenfalls Gegenstand der Untersuchung.

\subsection{Durchführung}
\label{subsec:execution}
Um eine faire und objektive Bewertung aller Konversionsverfahren zu ermöglichen,
werden vorab bestimmte Kriterien aufgestellt, anhand derer die Verfahren
bewertet werden sollen. Geprüft werden könnte beispielsweise, ob
\begin{itemize}
    \item alle getesteten Dokumente umgewandelt werden, ohne das der Konverter
          abstürzt,
    \item die aus der Konversion resultierenden Daten wohlgeformtes oder
          valides JSON bzw. XML sind,
    \item die Umwandlung verlustfrei abläuft,
    \item die Konversion ohne zusätzliche Angabe von Metadaten über das reine
          XML- bzw. JSON-Dokument hinaus funktioniert und
    \item die Konversion nicht anfällig für die bekannten Angriffe auf
          XML-Parser ist.
\end{itemize}

Im Anschluss an das Aufstellen der Kriterien werden XML-Testdokumente erstellt,
anhand derer deren Konversionsqualität bzw. Verlustlosigkeit überprüft werden
sollen. Jedes XML-Dokument soll dabei die Abbildung bestimmter XML-Features in
JSON testen (Attribute, Text-Node-Reihenfolge, etc.).

Indem die Test-XML-Dokumente zunächst vor und nach einem XML-Round-Trip
(Umwandlung von XML in JSON, danach Rückumwandlung des resultierenden JSON in
XML) verglichen werden, lässt sich feststellen, ob das jeweilige XML-Merkmal
fehlerfrei abgebildet wird. Um variierende Darstellungen logisch äquivalenter
XML-Dokumente Rechnung zu tragen, werden die zu vergleichenden XML-Dokumente
vor dem Vergleich mittels \emph{XML Canonicalization
(C14N)}\cite{boyer01canonicalization} in eine einheitliche Form gebracht.

Auch die Sicherheit der Konversionsverfahren soll getestet werden. Dazu werden
Testdokumente erstellt, die die Verwundbarkeit gegenüber gängigen
XML-Schwachstellen\cite{morgan2014xml,spaeth2016sok} wie
\begin{itemize}
    \item \emph{Exponential Entity Expansion} (\enquote{Billion Laughs}),
    \item \emph{Quadratic Blowup Entity Expansion},
    \item \emph{External Entity Expansion} (\enquote{XXE}, sowohl für lokale Pfade als auch
    URLs)\cite{steuck2002xxe} und
\item \emph{DTD Retrieval}
\end{itemize}
überprüfen.

Zuletzt soll mithilfe eines umfangreichen und komplexen XML-Dokuments
(ggf.~auch mehrerer Test-Dokumente) die Performance der Umwandlung bestimmt
werden. Dazu wird das selbe Test-Dokument mehrmals
umgewandelt und im Anschluss das arithmetische Mittel der Zeitdauer für die
Konvertierung von XML zu JSON bzw. JSON zu XML ermittelt.

Indem die Umwandlungsergebnisse mit einem Prüfprogramm
(einem sog.\ \enquote{Linter}) analysiert werden, wird sichergestellt, das es
sich dabei um wohlgeformte oder valide XML- bzw. JSON-Dokumente handelt. Als
Programme kämen dabei z.~B.\ Tools wie
\emph{demjson/jsonlint}\footnote{\url{http://deron.meranda.us/python/demjson/}},
\emph{xmllint}\footnote{\url{http://xmlsoft.org/xmllint.html}} oder
\emph{XMLStarlet}\footnote{\url{http://xmlstar.sourceforge.net/}} in Betracht.

\subsection{Erarbeitung eines Umwandlungsalgorithmus}
\label{subsec:development}

Sollte sich bei der Durchführung der Tests herausstellen, das keines der
getesteten XML-zu-JSON-Konversionsverfahren den zuvor aufgestellten Kriterien
genügt, soll ein eigener Abbildungsalgorithmus für
XML-Datenstrukturen im JSON-Format erarbeitet werden

Dabei scheint es sinnvoll, das vielversprechendste der zurvor geprüften
Verfahren als Basis zu nutzen und die fehlenden Features und Modifikationen
zu ergänzen.

Falls es im gesetzten zeitlichen Rahmen einer Bachelorarbeit möglich ist, eine
Referenzimplementierung des Algorithmus zu erstellen, soll diese wie alle
anderen Verfahren evaluiert werden.

\subsection{Vorläufige Gliederung}
\label{subsec:structure}

Nach der Einleitung wird zunächst eine Einführung in die Struktur der
Formate XML und JSON gegeben (wobei die Formate allerdings selbstverständlich
nicht erschöpfend dargestellt werden können, da dies den Rahmen sprengen
würde). Im Zuge dessen werden ebenfalls die wichtigsten Unterschiede zwischen
beiden Formaten aufgezeigt (z.~B.\ JSON-Arrays, XML-Attribute, Duck Typing,
etc.).

Daran anschließend wird zunächst das Ziel konkret definiert, d.~h.\ es werden
überprüfbaren Kriterien aufgestellt, denen ein Konversionsverfahren genügen
muss. Dazu gehört beispielsweise, Verlustlosigkeit genauer zu definieren (vor
allem im Bezug auf Fragen wie \enquote{Sind Kommentare relevant?}). Hier wird
auch ein Abschnitt zu Angriffen (\emph{XXE}, \emph{Billion Laughs}, etc.)
eingefügt, gegen den der jeweilige Konverter gewappnet sein soll.

Dann werden die verschiedenen bestehenden Ansätze zur Umformung aufgezählt
und die zugrunde liegenden Konzepte kurz und grob beschrieben.

Der folgende Abschnitt erklärt den Versuchsaufbau (XML-Testdokumente).
Die verschiedenen Konversionsverfahren werden dann anhand des zuvor
aufgestellten Anforderungskatalogs geprüft.

Anschließend werden die Ergebnisse der Tests vorgestellt und diskutiert.
Dabei wird erörtert, ob und welche Verfahren die zuvor genannten Bedingungen
erfüllen.

Falls keines der Konversionsverfahren den zuvor aufgestellten Kriterien genügt,
folgt der Abschnitt, der einen eigenen Umwandlungsalgorithmus
entwickelt\footnote{Siehe Abschnitt~\ref{subsec:development}.}.

Im Diskussions-Abschnitt werden die aktuellen Möglichkeiten zur Umwandlung
von XML in JSON und wieder zurück abschließend eingeschätzt und bewertet.
Dabei soll auch ein Ausblick auf weitere Verbesserungsmöglichkeiten oder
alternative Wege zum Ziel erörtert werden.

Im letzten Kapitel zu verwandten Arbeiten werden weitere Arbeiten zum Thema
beschrieben und Unterschiede zur eigenen Arbeit herausgearbeitet.

%\appendices{}

\ifCLASSOPTIONcaptionsoff{}
  \newpage
\fi
\printbibliography{}
\end{document}
