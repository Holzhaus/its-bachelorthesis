\documentclass[
    alternativetitlepage=bild,
    cornerlogo=hgi_nds_logo2,
    sectionoverview,
]{rubpresentation}
\usepackage[outline]{contour}
\contourlength{0.08em}

\usepackage{pgfplots}
\usepgfplotslibrary{dateplot}

\usetikzlibrary{shapes.geometric}
\usetikzlibrary{shadows.blur}
\usetikzlibrary{arrows.meta}
\tikzset{
    square/.style={regular polygon,regular polygon sides=4},
    icon/.style={draw=none, shade, shading=axis,left color=black!10!white,right color=black!20!white, shading angle=45,square,rounded corners=0.1cm,blur shadow={shadow blur steps=5, shadow xshift=0.5mm, shadow yshift=-0.5mm},outer sep=2mm, inner sep=0,text width=11mm, align=center},
    scopearrow/.style={thick, arrowhead=6mm},
    invisible/.style={opacity=0},
    visible on/.style={alt=#1{}{invisible}},
    alt/.code args={<#1>#2#3}{%
      \alt<#1>{\pgfkeysalso{#2}}{\pgfkeysalso{#3}} % \pgfkeysalso doesn't change the path
    },
}
\newcommand{\jsonlogo}{\tt \{js\includegraphics[width=0.5em]{images/json-logo}n\}}
\newcommand{\xmllogo}{\tt <xml>}
\def\framecrossout{%
    \begin{tikzpicture}[remember picture,overlay]%
        \node[draw=none,yshift=-1cm] at (current page.center) {\includegraphics[width=6cm]{images/checkmark-no}};%
    \end{tikzpicture}%
}
\def\framecheckmark{%
    \begin{tikzpicture}[remember picture,overlay]%
        \node[draw=none,yshift=-1cm] at (current page.center) {\includegraphics[width=6cm]{images/checkmark-yes}};%
    \end{tikzpicture}%
}
\setbeamercovered{invisible}

\title[XML/JSON conversions]
{Secure XML/JSON conversions}
\subtitle{Bachelor colloquium --- Introduction}

\author[Holthuis]{Jan~Holthuis}

\institute[Ruhr-Uni Bochum]
{%\inst{1}%
Chair for Net- and Data Security\\
Ruhr-University of Bochum
}
\date%[KPT 2004] (optional)
{\today}
\subject{Computer Science}

\titlegraphic{titlepage.png}
\sponsorlogo[height=7.6mm,interpolate=true]{hgi_nds_logo}

\begin{document}

\frame[plain]{\titlepage}

%\begin{frame}{The \texttt{\textbackslash note}-Macro}
%\begin{itemize}[<+->]
%\item normal text for the presentation.
%\note<1-2>[item]{Say something to the audience!}
%\item and text for the presentation.
%\item foo
%\end{itemize}
%\note<2>{Another note for you!}
%\end{frame}

%\note[enumerate]{\item foo \item bar \item baz \item foobar}


\section{Introduction}


\subsection{Motivation}

\begin{frame}[plain]
    \frametitle{Why convert between XML and JSON?}
    \framesubtitle{}
    \begin{itemize}
        \item{} Web APIs are booming since \emph{Web 2.0} and \emph{IoT} hype
            \begin{itemize}
                \item{} most of them use XML, JSON or both as data format
            \end{itemize}
        \item{} Some \enquote{normal} websites are now based on these formats (e.g. \emph{AngularJS})\\
        \item{} Lots of file formats are XML-based (e.g. RSF/ASF, MathML, SVG, XHTML, ODT, OOXML, \ldots{})\\
        \item{} There are even JSON-based databases like \emph{CouchDB} and \emph{MongoDB}\\
        \item{} Support by programming languages, frameworks and libraries is inconsistent\\
    \end{itemize}
    \begin{center}
        \bf $\Rightarrow$ plenty of use cases for converting between XML and JSON!
    \end{center}
\end{frame}

\begin{frame}[plain]
    \frametitle{Why convert between XML and JSON?}
    \framesubtitle{}
    \begin{itemize}
        \item{} Support by programming languages, frameworks and libraries is inconsistent\\
        \item{} Parsing JSON is usually less resource heavy and faster than XML\\
        \item{} XML has more features and is widely support by the industry\\
        \item{} XML has more features and is widely support by the industry\\
    \end{itemize}
    \begin{center}
        \bf $\Rightarrow$ plenty of use cases for converting between XML and JSON!
    \end{center}
\end{frame}

\begin{frame}[plain]
    \frametitle{Support by Web APIs}
    \framesubtitle{From mid 2005 until end of 2013}
    \pgfplotstableread{xml-json-apis-2005-to-2013.csv}{\tabledata}
    \begin{tikzpicture}
            \begin{axis}[
                ymajorgrids,
                grid style={line width=.1pt, draw=black!40},
                minor tick num=5,
                set layers,
                anchor=north east,
                smooth, area style, enlarge x limits=false,
                date coordinates in=x,
                xtick = {
                  {2005-01-01},
                  {2006-01-01},
                  {2006-01-01},
                  {2007-01-01},
                  {2008-01-01},
                  {2009-01-01},
                  {2010-01-01},
                  {2011-01-01},
                  {2012-01-01},
                  {2013-01-01}
                },
                ytick = {0, 10, 20, 30, 40, 50, 60, 70, 80},
                xticklabel=\year,
                yticklabel={\pgfmathprintnumber\tick\%},
                xtick pos=left,
                ytick pos=left,
                height=8cm,
                width=12cm,
            ]
            \addplot[green,fill=green,fill opacity=0.4,thick,on layer=axis foreground] table[x=date,y=xml] {\tabledata} \closedcycle;
            \addplot[blue,fill=blue,fill opacity=0.4,thick,on layer=axis foreground] table[x=date,y=json] {\tabledata} \closedcycle;

            \begin{pgfonlayer}{axis foreground}
                \node[draw=none,font=\Large,on layer=axis foreground] at (axis cs: 2008-01-01,40) {XML};
                \node[draw=none,font=\Large,on layer=axis foreground,text=white] at (axis cs: 2012-01-01,20) {JSON};
            \end{pgfonlayer}
        \end{axis}
        \node[below left, rotate=90, font=\tiny] {\tiny Data Source: ProgrammableWeb, Dec. 26 2013};
    \end{tikzpicture}
\end{frame}

\subsection{Scope}

\begin{frame}[plain]
    \frametitle{Conversion \texttt{!=} Conversion}
    \framesubtitle{Converting arbitrary JSON into XML-based format}
    \begin{tikzpicture}
        \visible<1>{\node[draw=none,align=center] at (5,0) (dir1) {unidirectional};}
        \visible<2->{\node[draw=none,align=center] at (5,0) (dir2) {bidirectional};}
        \node[icon] at ( 0, 0) (json) {\jsonlogo};
        \node[icon] at (10, 0)  (xml) {\xmllogo};
        \draw[->] (json.north east) to[bend left] node[midway,above] {1.} (xml.north west);
        \visible<2->{\draw[<-,dashed] (json.south east) to[bend right] node[midway,above] {(2.)} (xml.south west);}
    \end{tikzpicture}
    \visible<3>{\framecrossout{}}
\end{frame}

\begin{frame}[plain]
    \frametitle{Conversion \texttt{!=} Conversion}
    \framesubtitle{Converting XML into JSON-based format}
    \begin{tikzpicture}
        \visible<1>{\node[draw=none,align=center] at (5,0) (dir1) {unidirectional};}
        \visible<2->{\node[draw=none,align=center] at (5,0) (dir2) {bidirectional};}
        \node[icon] at (10, 0) (json) {\jsonlogo};
        \node[icon] at ( 0, 0)  (xml) {\xmllogo};
        \draw[->] (xml.north east) to[bend left] node[midway,above] {1.} (json.north west);
        \visible<2->{\draw[<-,dashed] (xml.south east) to[bend right] node[midway,above] {(2.)} (json.south west);}
    \end{tikzpicture}
    \visible<3>{\framecheckmark{}}
\end{frame}

\begin{frame}[plain]
    \frametitle{Scope of thesis}
    \framesubtitle{}
    \begin{itemize}
        \item{} Find a way to convert arbitrary XML documents into JSON\\
        \item{} Be able to convert the JSON documents back to XML\\
        \item{} The conversion should \ldots\
            \begin{itemize}
                \item{} result in \textbf{well-formed JSON/XML},
                \item{} \textbf{require no additional metadata} (type hints, etc.),\\
                \item{} be \textbf{lossless} and
                    \begin{itemize}
                        \item{} XML documents before and after %
                                $\text{XML}\rightarrow\text{JSON}\rightarrow\text{XML}$ %
                                round-trip should be (logically) equivalent\\
                    \end{itemize}
                \item{} be \textbf{secure}.
                    \begin{itemize}
                        \item{} Not vulnerable to known attacks against parsers\\
                    \end{itemize}
            \end{itemize}
    \end{itemize}
\end{frame}

% TODO: Add content

\section{Basics}

\subsection{XML}

\begin{frame}[plain]
    \frametitle{XML}
    \framesubtitle{Overview}
    \begin{itemize}
        \item{} \emph{\enquote{eXtensible Markup Language}}\\
        \item{} Derived from SGML (ISO 8879)\\
        \item{} First published by W3C in 1998\\
        \item{} Currently two flavors:
            \begin{itemize}
                \item{} XML 1.0 (Fifth Edition), published November 26, 2008\\
                \item{} XML 1.1 (Second Edition), published August 15, 2006\\
            \end{itemize}
    \end{itemize}
\end{frame}

\begin{frame}[plain]
    \frametitle{XML}
    \framesubtitle{Related specifications}
    \begin{itemize}
        \item{} XML Schema Definition (XSD)\\
        \item{} XML Path Language (XPath)\\
        \item{} XML Pointer Language (XPointer)\\
        \item{} XML Query (XQuery)\\
        \item{} XML Signature\\
        \item{} XML Encryption\\
        \item{} XML-RPC\\
        \item{} XML Remote Procedure Call Protocol (XML-RPC)\\
        \item{} Simple Object Access Protocol (SOAP)
        %\item{} Extensible Stylesheet Language Transformations (XSLT)\\
        \item{} and many more\ldots\\
    \end{itemize}
\end{frame}

\subsection{JSON}

\begin{frame}[plain]
    \frametitle{JSON}
    \framesubtitle{Overview}
    \begin{itemize}
        \item{} \emph{\enquote{JavaScript Object Notation}}\\
        \item{} Popularized by Douglas Crockford in the early 2000s\\
        \item{} First specified officially as RFC 4627 (2006)\\
        \item{} Currently defined by:
            \begin{itemize}
                \item{} RFC 7159, published in March 2014\\
                \item{} ECMA-404, published in October 2013\\
            \end{itemize}
    \end{itemize}
\end{frame}

\begin{frame}[plain]
    \frametitle{JSON}
    \framesubtitle{Related specifications}
    \begin{itemize}
        \item{} JSON Schema\\
        \item{} XPath for JSON (JSONPath)\\
        \item{} JSON Pointer\\
        \item{} JSON Query Language (JSONiq)\\
        \item{} JSON Web Signature (JWS)
        \item{} JSON Web Encryption (JWE)
        \item{} JSON Remote Procedure Call Protocol (JSON-RPC)\\
        \item{} SOAP using JSON-RPC (SOAPjr)\\
        %\item{} Internet JSON (I-JSON)\\
        \item{} and many more\ldots{}\\
    \end{itemize}
\end{frame}


\begin{frame}[plain,t]
    \frametitle{XML and JSON}
    \framesubtitle{Related specifications}
    \begin{columns}[t]
        \begin{column}{0.5\textwidth}
            \begin{center}\textbf{\Large XML}\end{center}
            \begin{itemize}
                \item{} Schema Definition (XSD)\\
                \item{} XPath\\
                \item{} XPointer\\
                \item{} XQuery\\
                \item{} XML Signature\\
                \item{} XML Encryption\\
                \item{} XML-RPC\\
                \item{} SOAP
            \end{itemize}
        \end{column}
        \begin{column}{0.5\textwidth}
            \begin{center}\textbf{\Large JSON}\end{center}
            \begin{itemize}
                \item{} JSON Schema\\
                \item{} JSONPath\\
                \item{} JSON Pointer\\
                \item{} JSONiq Query Language\\
                \item{} JSON Web Signature (JWS)
                \item{} JSON Web Encryption (JWE)
                \item{} JSON-RPC\\
                \item{} SOAPjr\\
            \end{itemize}
        \end{column}
    \end{columns}
    \hspace{1cm}\raisebox{0cm}[1cm][0pt]{%
        %\makebox[\textwidth][r]{
        %    \includegraphics[width=2cm]{images/i-see-what-you-did-there.pdf}\newline
        %    {\Large \bf \color{white}\contour{black}{I SEE WHAT YOU DID THERE...}}
        %}
        \begin{tikzpicture}
            \node (fig) {
              \includegraphics[width=2cm]{images/i-see-what-you-did-there}
            };
            \node [anchor=north,inner sep=0]
              at ([yshift=-2mm]fig.south) {\Large \color{white}\contour{black}{\bf I SEE WHAT YOU DID THERE...}};
        \end{tikzpicture}
    }
\end{frame}

\section{XML vs. JSON}

\begin{frame}[plain]
    \frametitle{XML vs. JSON}
    \framesubtitle{Datatypes}
    \begin{columns}[t]
        \begin{column}{0.5\textwidth}
            \begin{center}\textbf{\Large XML}\end{center}
            No datatype support:
            \begin{itemize}
                \item{} Everything is a string\\
                \item{} If you want to specify types, you need a %
                        \emph{schema}\\
            \end{itemize}
        \end{column}
        \begin{column}{0.5\textwidth}
            \begin{center}\textbf{\Large JSON}\end{center}
            Syntactic datatype support for:
            \begin{itemize}
                \item{} Strings\\
                \item{} Numbers
                    \begin{itemize}
                        \item{} Integers\\
                        \item{} Fractions\\
                        \item{} Exponents\\
                    \end{itemize}
                \item{} Arrays
                \item{} Booleans (\texttt{true}, \texttt{false})\\
                \item{} \texttt{null}\\
            \end{itemize}
        \end{column}
    \end{columns}
\end{frame}

\begin{frame}[plain]
    \frametitle{XML vs. JSON}
    \framesubtitle{Datatypes}
    \begin{columns}[t]
        \begin{column}{0.5\textwidth}
            \begin{center}\textbf{\Large XML}\end{center}
            No datatype support:
            \begin{itemize}
                \item{} Everything is a string\\
                \item{} If you want to specify types, you need a %
                        \emph{schema}\\
            \end{itemize}
        \end{column}
        \begin{column}{0.5\textwidth}
            \begin{center}\textbf{\Large JSON}\end{center}
            Syntactic datatype support for:
            \begin{itemize}
                \item{} Strings\\
                \item{} Numbers
                    \begin{itemize}
                        \item{} Integers\\
                        \item{} Fractions\\
                        \item{} Exponents\\
                    \end{itemize}
                \item{} Arrays
                \item{} Booleans (\texttt{true}, \texttt{false})\\
                \item{} \texttt{null}\\
            \end{itemize}
        \end{column}
    \end{columns}
\end{frame}

\begin{frame}[plain]
    \frametitle{XML vs. JSON}
    \framesubtitle{Security Vulnerabilities}
    \begin{columns}[t]
        \begin{column}{0.5\textwidth}
            \begin{center}\textbf{\Large XML}\end{center}
            Various generic attacks on XML Parsers:
            \begin{itemize}
                \item{} Denial of Service Attacks %
                        (\emph{\enquote{Billion Laughs}}, %
                        \emph{Quadratic Blowup Entity Expansion})\\
                \item{} Local/Remote File Inclusion (LFI/RFI) using %
                        \emph{External Entity Expansion}\\
                \item{} Server-Side-Request Forgery (SSRF) via DTD Retrieval\\
            \end{itemize}
        \end{column}
        \begin{column}{0.5\textwidth}
            \begin{center}\textbf{\Large JSON}\end{center}
            Several attacks target JavaScript:
            \begin{itemize}
                \item{} Cross-Site-Scripting (XSS) if JavaScript's %
                        \texttt{eval()} is used instead of%
                        \texttt{JSON.parse()}\\
                \item{} In-browser XSS/CSRF attacks against JSON-P\\
                \item{} XSS if a JSON web service does not set the
                        \texttt{Content-Type}\\
                \item{} \ldots{}but no attacks on JSON parsing itself!\\
            \end{itemize}
        \end{column}
    \end{columns}
\end{frame}



%%% Finally the last slide

\begin{frame}[plain]
\frametitle{Thanks!}
 \begin{center}
 {\bfseries\fontsize{30pt}{1.2em}\selectfont Questions?}
 \end{center}
  \begin{columns}
    \begin{column}{0.5\textwidth}
      \begin{center}
        %\font\endfont = cmss10 at 25.40mm
        %\color{Brown}
        %\endfont
        %\baselineskip 20.0mm
        Reach out via email:
        \begin{itemize}
        \item \textbf{Jan Holthuis}\\
              jan.holthuis@rub.de
        \end{itemize}
      \end{center}
    \end{column}
    \begin{column}{0.5\textwidth}
      \begin{center}
        \pgfimage[width=\textwidth]{questions.jpg}
      \end{center}
    \end{column}
  \end{columns}
\end{frame}

\end{document}
