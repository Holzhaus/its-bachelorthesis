\documentclass[
    alternativetitlepage=bild,
    cornerlogo=hgi_nds_logo2,
    sectionoverview,
]{rubpresentation}
\usepackage[outline]{contour}
\contourlength{0.08em}

\title[XML/JSON conversions]
{Secure XML/JSON conversions}
\subtitle{Bachelor colloquium --- Introduction}

\author[Holthuis]{Jan~Holthuis}

\institute[Ruhr-Uni Bochum]
{%\inst{1}%
Chair for Net- and Data Security\\
Ruhr-University of Bochum
}
\date%[KPT 2004] (optional)
{\today}
\subject{Computer Science}

\titlegraphic{titlepage.png}
\sponsorlogo[height=7.6mm,interpolate=true]{hgi_nds_logo}

\begin{document}

\frame[plain]{\titlepage}

%\begin{frame}{The \texttt{\textbackslash note}-Macro}
%\begin{itemize}[<+->]
%\item normal text for the presentation.
%\note<1-2>[item]{Say something to the audience!}
%\item and text for the presentation.
%\item foo
%\end{itemize}
%\note<2>{Another note for you!}
%\end{frame}

%\note[enumerate]{\item foo \item bar \item baz \item foobar}

\section{Introduction}

\subsection{Scope}

% TODO: Add content

\subsection{Motivation}

% TODO: Add content

\section{Basics}

\subsection{XML}

\begin{frame}[plain]
    \frametitle{XML}
    \framesubtitle{Overview}
    \begin{itemize}
        \item{} \emph{\enquote{eXtensible Markup Language}}\\
        \item{} Derived from SGML (ISO 8879)\\
        \item{} First published by W3C in 1998\\
        \item{} Currently two flavors:
            \begin{itemize}
                \item{} XML 1.0 (Fifth Edition), published November 26, 2008\\
                \item{} XML 1.1 (Second Edition), published August 15, 2006\\
            \end{itemize}
    \end{itemize}
\end{frame}

\begin{frame}[plain]
    \frametitle{XML}
    \framesubtitle{Related specifications}
    \begin{itemize}
        \item{} XML Schema Definition (XSD)\\
        \item{} XML Path Language (XPath)\\
        \item{} XML Pointer Language (XPointer)\\
        \item{} XML Query (XQuery)\\
        \item{} XML Signature\\
        \item{} XML Encryption\\
        \item{} XML-RPC\\
        \item{} XML Remote Procedure Call Protocol (XML-RPC)\\
        \item{} Simple Object Access Protocol (SOAP)
        %\item{} Extensible Stylesheet Language Transformations (XSLT)\\
        \item{} and many more\ldots\\
    \end{itemize}
\end{frame}

\subsection{JSON}

\begin{frame}[plain]
    \frametitle{JSON}
    \framesubtitle{Overview}
    \begin{itemize}
        \item{} \emph{\enquote{JavaScript Object Notation}}\\
        \item{} Popularized by Douglas Crockford in the early 2000s\\
        \item{} First specified officially as RFC 4627 (2006)\\
        \item{} Currently defined by:
            \begin{itemize}
                \item{} RFC 7159, published in March 2014\\
                \item{} ECMA-404, published in October 2013\\
            \end{itemize}
    \end{itemize}
\end{frame}

\begin{frame}[plain]
    \frametitle{JSON}
    \framesubtitle{Related specifications}
    \begin{itemize}
        \item{} JSON Schema\\
        \item{} XPath for JSON (JSONPath)\\
        \item{} JSON Pointer\\
        \item{} JSON Query Language (JSONiq)\\
        \item{} JSON Web Signature (JWS)
        \item{} JSON Web Encryption (JWE)
        \item{} JSON Remote Procedure Call Protocol (JSON-RPC)\\
        \item{} SOAP using JSON-RPC (SOAPjr)\\
        %\item{} Internet JSON (I-JSON)\\
        \item{} and many more\ldots{}\\
    \end{itemize}
\end{frame}


\begin{frame}[plain,t]
    \frametitle{XML and JSON}
    \framesubtitle{Related specifications}
    \begin{columns}[t]
        \begin{column}{0.5\textwidth}
            \begin{center}\textbf{\Large XML}\end{center}
            \begin{itemize}
                \item{} Schema Definition (XSD)\\
                \item{} XPath\\
                \item{} XPointer\\
                \item{} XQuery\\
                \item{} XML Signature\\
                \item{} XML Encryption\\
                \item{} XML-RPC\\
                \item{} SOAP
            \end{itemize}
        \end{column}
        \begin{column}{0.5\textwidth}
            \begin{center}\textbf{\Large JSON}\end{center}
            \begin{itemize}
                \item{} JSON Schema\\
                \item{} JSONPath\\
                \item{} JSON Pointer\\
                \item{} JSONiq Query Language\\
                \item{} JSON Web Signature (JWS)
                \item{} JSON Web Encryption (JWE)
                \item{} JSON-RPC\\
                \item{} SOAPjr\\
            \end{itemize}
        \end{column}
    \end{columns}
    \hspace{1cm}\raisebox{0cm}[1cm][0pt]{%
        %\makebox[\textwidth][r]{
        %    \includegraphics[width=2cm]{images/i-see-what-you-did-there.pdf}\newline
        %    {\Large \bf \color{white}\contour{black}{I SEE WHAT YOU DID THERE...}}
        %}
        \begin{tikzpicture}
            \node (fig) {
              \includegraphics[width=2cm]{images/i-see-what-you-did-there}
            };
            \node [anchor=north,inner sep=0]
              at ([yshift=-2mm]fig.south) {\Large \color{white}\contour{black}{\bf I SEE WHAT YOU DID THERE...}};
        \end{tikzpicture}
    }
\end{frame}

\section{XML vs. JSON}

\begin{frame}[plain]
    \frametitle{XML vs. JSON}
    \framesubtitle{Datatypes}
    \begin{columns}[t]
        \begin{column}{0.5\textwidth}
            \begin{center}\textbf{\Large XML}\end{center}
            No datatype support:
            \begin{itemize}
                \item{} Everything is a string\\
                \item{} If you want to specify types, you need a %
                        \emph{schema}\\
            \end{itemize}
        \end{column}
        \begin{column}{0.5\textwidth}
            \begin{center}\textbf{\Large JSON}\end{center}
            Syntactic datatype support for:
            \begin{itemize}
                \item{} Strings\\
                \item{} Numbers
                    \begin{itemize}
                        \item{} Integers\\
                        \item{} Fractions\\
                        \item{} Exponents\\
                    \end{itemize}
                \item{} Arrays
                \item{} Booleans (\texttt{true}, \texttt{false})\\
                \item{} \texttt{null}\\
            \end{itemize}
        \end{column}
    \end{columns}
\end{frame}

\begin{frame}[plain]
    \frametitle{XML vs. JSON}
    \framesubtitle{Datatypes}
    \begin{columns}[t]
        \begin{column}{0.5\textwidth}
            \begin{center}\textbf{\Large XML}\end{center}
            No datatype support:
            \begin{itemize}
                \item{} Everything is a string\\
                \item{} If you want to specify types, you need a %
                        \emph{schema}\\
            \end{itemize}
        \end{column}
        \begin{column}{0.5\textwidth}
            \begin{center}\textbf{\Large JSON}\end{center}
            Syntactic datatype support for:
            \begin{itemize}
                \item{} Strings\\
                \item{} Numbers
                    \begin{itemize}
                        \item{} Integers\\
                        \item{} Fractions\\
                        \item{} Exponents\\
                    \end{itemize}
                \item{} Arrays
                \item{} Booleans (\texttt{true}, \texttt{false})\\
                \item{} \texttt{null}\\
            \end{itemize}
        \end{column}
    \end{columns}
\end{frame}

\begin{frame}[plain]
    \frametitle{XML vs. JSON}
    \framesubtitle{Security Vulnerabilities}
    \begin{columns}[t]
        \begin{column}{0.5\textwidth}
            \begin{center}\textbf{\Large XML}\end{center}
            Various generic attacks on XML Parsers:
            \begin{itemize}
                \item{} Denial of Service Attacks %
                        (\emph{\enquote{Billion Laughs}}, %
                        \emph{Quadratic Blowup Entity Expansion})\\
                \item{} Local/Remote File Inclusion (LFI/RFI) using %
                        \emph{External Entity Expansion}\\
                \item{} Server-Side-Request Forgery (SSRF) via DTD Retrieval\\
            \end{itemize}
        \end{column}
        \begin{column}{0.5\textwidth}
            \begin{center}\textbf{\Large JSON}\end{center}
            Several attacks target JavaScript:
            \begin{itemize}
                \item{} Cross-Site-Scripting (XSS) if JavaScript's %
                        \texttt{eval()} is used instead of%
                        \texttt{JSON.parse()}\\
                \item{} In-browser XSS/CSRF attacks against JSON-P\\
                \item{} XSS if a JSON web service does not set the
                        \texttt{Content-Type}\\
                \item{} \ldots{}but no attacks on JSON parsing itself!\\
            \end{itemize}
        \end{column}
    \end{columns}
\end{frame}



%%% Finally the last slide

\begin{frame}[plain]
\frametitle{Thanks!}
 \begin{center}
 {\bfseries\fontsize{30pt}{1.2em}\selectfont Questions?}
 \end{center}
  \begin{columns}
    \begin{column}{0.5\textwidth}
      \begin{center}
        %\font\endfont = cmss10 at 25.40mm
        %\color{Brown}
        %\endfont
        %\baselineskip 20.0mm
        Reach out via email:
        \begin{itemize}
        \item \textbf{Jan Holthuis}\\
              jan.holthuis@rub.de
        \end{itemize}
      \end{center}
    \end{column}
    \begin{column}{0.5\textwidth}
      \begin{center}
        \pgfimage[width=\textwidth]{questions.jpg}
      \end{center}
    \end{column}
  \end{columns}
\end{frame}

\end{document}
