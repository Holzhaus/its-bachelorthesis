\begin{abstract}
Bei XML und JSON handelt es sich um -- insbesondere im Mobile- und % chktex 8
Web-Bereich konkurrierende -- menschenlesbare Formate für die % chktex 8
Speicherung und den Austausch hierarchisch strukturierter Daten. Je nach
eingesetzter API, Programmiersprache oder Programmbibliothek kann es dabei
sinnvoll sein, die Daten zunächst in das jeweils andere Format zu überführen.
In der geplanten Bachelorarbeit soll sich mit der verlustfreien Translation von
XML-Daten in JSON-Strukuren befasst werden. Dazu sollen bestehende Verfahren
auf Genauigkeit und Sicherheit hin untersucht werden. Schlussendlich soll in
der Arbeit ein verlustfreies Verfahren zur Übersetzung von XML-Daten in
JSON sowie in Rückrichtung entwickelt und vorgestellt werden.
\end{abstract}

\pagestyle{scrplain} % switch off headers and footers

%% English Version (PO13)
%\section*{Declaration}
%{\selectlanguage{english}
%I hereby declare that this submission is my own work and that, to the best of
%my knowledge and belief, it contains no material previously published or
%written by another person nor material which to a substantial extent has been
%accepted for the award of any other degree or diploma of the university or
%other institute of higher learning, except where due acknowledgment has been
%made in the text.
%}

%% German Version (PO13)
%\section*{Erklärung}
%{\selectlanguage{ngerman}
%Hiermit versichere ich, dass ich die vorliegende Arbeit selbstständig verfasst
%und keine anderen als die angegebenen Quellen und Hilfsmittel benutzt habe,
%dass alle Stellen der Arbeit, die wörtlich oder sinngemäß aus anderen Quellen
%übernommen wurden, als solche kenntlich gemacht sind und dass die Arbeit in
%gleicher oder ähnlicher Form noch keiner Prüfungsbehörde vorgelegt wurde.
%}

%% German Version (vor PO13)
\section*{Eidesstattliche Erklärung}
{\selectlanguage{ngerman}
Ich erkläre, dass ich keine Arbeit in gleicher oder ähnlicher Fassung bereits
für eine andere Prüfung an der Ruhr-Universität Bochum oder einer anderen
Hochschule eingereicht habe.

Ich versichere, dass ich diese Arbeit selbständig verfasst und keine anderen
als die angegebenen Quellen benutzt habe. Die Stellen, die anderen Quellen dem
Wortlaut oder dem Sinn nach entnommen sind, habe ich unter Angabe der Quellen
kenntlich gemacht. Dies gilt sinngemäß auch für verwendete Zeichnungen,
Skizzen, bildliche Darstellungen und dergleichen.
}

\vspace{2cm}
\noindent\begin{tabularx}{\textwidth}{lX}
\makebox[5.5cm]{\hrulefill} & \hrulefill\\
Ort, Datum & Unterschrift\\[8ex]% adds space between the two sets of signatures
\end{tabularx}

\cleardoublepage{}

\section*{Erklärung zum Urheberrecht}
{\selectlanguage{ngerman}
Ich erkläre mich damit einverstanden, dass meine Abschlussarbeit am
Lehrstuhl \emph{Netz- und Datensicherheit (NDS)} dauerhaft in elektronischer
und gedruckter Form aufbewahrt wird und dass die Ergebnisse aus dieser Arbeit
unter Einhaltung guter wissenschaftlicher Praxis in der Forschung
weiterverwendet werden dürfen.
}

\vspace{2cm}
\noindent\begin{tabularx}{\textwidth}{lX}
\makebox[5.5cm]{\hrulefill} & \hrulefill\\
Ort, Datum & Unterschrift\\[8ex]
\end{tabularx}

\cleardoublepage{}

\pagestyle{scrheadings} % reenable headers and footers

%% generate table of contents
\tableofcontents

\cleardoublepage{}
