\chapter{Einleitung} \label{chap:intro}

Für den implementationsunabhängigen Austausch von zugleich menschen- als auch
maschinenlesbaren Daten hat sich die \gls{xml}
bewährt. In bestimmten Bereichen wie Web-\acrshortpl{api} hat die \gls{json}
das bewährte \acrshort{xml}-Format inzwischen jedoch überflügelt.

Dabei kann neben der spezifischen Situation auch die eingesetzte
Programmiersprache, die Unterstützung durch das zugrundeliegende Framework
oder die persönliche Präferenz des Entwicklers den Ausschlag geben, welches
Format ein Webservice oder eine Programmbibliothek unterstützt.

Zwecks Interoperabilität zwischen verschiedenen Teilen einer Anwendung kann es
daher notwendig werden, Daten von einem Format temporär in das jeweils andere
zu überführen und später wieder in das ursprüngliche Format zu bringen.

Dabei sollen Daten, die von der Anwendungslogik nicht verändert wurden, auch
nach der Rückübersetzung ins Ursprungsformat unverändert bleiben. Damit dies
gewährleistet ist, darf das zugrunde liegende Konversionverfahren keine
Informationen bei der Umwandlung verwerfen -- es muss also verlustlos % chktex 8
arbeiten.

Das konkrete Konversionsverfahren ist dabei abhängig vom jeweiligen
Ausgangsformat, d.~h.\ die Umwandlungsverfahren für die beiden Richtungen
\begin{enumerate}
    \item \acrshort{json} $\rightarrow$ \acrshort{xml} $\rightarrow$ \acrshort{json} und
    \item \acrshort{xml} $\rightarrow$ \acrshort{json} $\rightarrow$ \acrshort{xml}
\end{enumerate}
haben jeweils eigene Anforderungen und sind getrennt voneinander zu betrachten.

Während die Abbildung beliebiger \acrshort{json}-Datenstrukturen in \acrshort{xml} -- zumindest bei
oberflächlicher Betrachtung -- trivial erscheint, ist dies beim verlustlosen
Transfer von \acrshort{xml}-Daten ins \acrshort{json}-Format keineswegs der Fall.

Ziel der vorliegenden Arbeit ist es daher, ein sicheres und verlustloses
Verfahren zur Konversion von \acrshort{xml}-Dokumenten in das \acrshort{json}-Format zu finden.

Dazu wird eine Reihe von bereits erhältlichen Programmbibliotheken analysiert,
die \acrshort{xml}-Daten in \acrshort{json} abbilden und aus der resultierenden
\acrshort{json}-Datenstruktur wieder ein \acrshort{xml}-Dokument erstellen können.

Da keines der analysierten Verfahren die zuvor aufgestellten Kriterien
in Gänze erfüllt, wird das vollständigste Konversionsprogramm
weiterentwickelt und im Anschluss erneut evaluiert.

\section{Motivation}
\label{sec:motivation}
Das Aufkommen des sogenannten \emph{Web 2.0} und die zunehmenden Vernetzung
durch das \gls{iot} ging mit einer erhöhten
Verfügbarkeit von öffentlichen Web-\acrshortpl{api} einher. Als Datenformat wird dabei
häufig \acrshort{json} oder \acrshort{xml} verwendet.

Neben \acrshort{xml}-basierten Webservices, die beispielsweise das \acrshort{saml}-Framework, SOAP,
oder \gls{xmlrpc} verwenden, wird \acrshort{xml} auch in einer Vielzahl weiterer
Einsatzbereiche eingesetzt. So dient \acrshort{xml} den Dateiformaten \acrshort{rss}/\acrshort{asf}, \acrshort{mathml},
\gls{svg} oder \gls{xhtml} als Basis. Auch die gängigen
Office-Dateiformate -- das \acrfull{odf}, Microsofts % chktex 8
\acrfull{ooxml} und Apples iWorks -- bauen auf \acrshort{xml} auf. % chktex 8

Inzwischen gewinnt jedoch \acrshort{json} vor allem im Mobile-
und Web-Bereich immer mehr an Bedeutung (vgl. Abb.~\ref{fig:xmljsonapis}). Laut dem \acrshort{api}-Verzeichnis
\emph{ProgrammableWeb}\footnote{\url{https://www.programmableweb.com/}} unterstützten im Jahr 2013 ca. 60 Prozent aller neu
hinzugefügten \acrshortpl{api} das \acrshort{json}-Format, während
\acrshort{xml} im selben Zeitraum lediglich von 37 Prozent der neuen \acrshortpl{api}
unterstützt wurde.~\cite{duvander2013convergence}

\begin{figure}[h!]
    \begin{center}
        \includestandalone[width=\textwidth]{xml-json-apis-2005-to-2013}
    \end{center}
    \caption{Sowohl \acrshort{json} als auch \acrshort{xml} wurden 2013 von jeweils mehr als 45 Prozent der Web-\acrshortpl{api} unterstützt.}
    \label{fig:xmljsonapis}
\end{figure}


\acrshort{json} ist bei bestimmten Aufgaben in Bezug auf Geschwindigkeit und
Ressourcenauslastung deutlich effizienter als \acrshort{xml}~\cite{nurseitov2009comparison}\@.
Inzwischen setzen auch einige populäre NoSQL-Datenbanken wie
\emph{CouchDB} oder \emph{MongoDB} auf \acrshort{json} zur Datenspeicherung. Auch
MySQL verfügt seit Version 5.7.8 über einen nativen \acrshort{json}-Datentyp.

Die Umwandlung zwischen den beiden Formaten kann aus vielen Gründen
notwendig werden. Soll beispielsweise ein SOAP-Webservice als moderne
\acrshort{json}-\acrshort{rest}-Ressource angeboten werden, muss zwischen \acrshort{xml} und \acrshort{json} konvertiert
werden. Auch bei der Speicherung von \acrshort{xml}-Daten in den oben genannten NoSQL-Datenbanken
ist dies der Fall. Zudem ist die Unterstützung der Formate durch
Programmiersprachen, Frameworks und Applikationen nicht immer gleich gut.

\section{Verwandte Arbeiten}

Muschett, Salz und Schenker~\cite{jsonx} schlugen 2011 ein Format zur Repräsentation beliebiger \acrshort{json}-Daten in \acrshort{xml} 1.0 names JSONx vor. In dem inzwischen ausgelaufenen \gls{idraft} werden eine Reihe von Regeln festgelegt um Instanzen der verschiedenen \acrshort{json}-Daten in \acrshort{xml}-Strukturen zu konvertieren. Die vorliegenden Arbeit untersucht hingegen die Konversion in umgekehrter Richtung, d.h. die Umwandlung beliebiger \acrshort{xml}-Dokumente in das \acrshort{json}-Format.

Ein weiteres \acrshort{xml}-basiertes Format, dass das Datenmodell der \acrfull{json} in \acrshort{xml} nachbildet und dadurch die verlustlose Abbildung beliebiger \acrshort{json}-Daten in \acrshort{xml} ermöglicht, ist \gls{jxml}~\cite{jxml}.

Christensen \emph{et.\,al.}~\cite{US7761484} halten ein US-Patent an der Übersetzung zwischen \acrshort{xml} und \enquote{dynamic language data expressions}, womit unter anderem auch \acrshort{json} gemeint ist. Es nennt mehrere verschiedene denkbare Ausführungen, beispielsweise die Konversion von \acrshort{json} in typannotiertes \acrshort{xml} mittels Attributen oder Element-Namen, aber auch die Round-Trip-Umwandlung von \acrshort{xml} zu \acrshort{json} und zurück, ohne diese jedoch im Detail zu beschreiben.

David Lee~\cite{lee2011jxon} entwickelte 2011 ein Verfahren, um anhand annotierter \acrshort{xsd}-Schemata \acrshort{xslt}-Dateien zu erzeugen, die in der Lage sind, bidirektional zwischen \acrshort{xml} und \gls{jxml} zu konvertieren.  Damit ist also -- mit einem Umweg über \gls{jxml} -- die verlustlose Umwandlung von \acrshort{xml} in \acrshort{json} und wieder zurück möglich.
Durch die Nutzung des jeweiligen \acrshort{xsd}-Schemas der zu konvertierenden \acrshort{xml}-Dokumente sind bei der Umwandlung die verwendeten Datentypen bekannt, wodurch ein zuverlässigen Mapping auf die nativen \acrshort{json}-Typen möglich wird.

Im Gegensatz dazu beschränkt sich die vorliegende Arbeit jedoch auf Konverter, die beliebige \acrshort{xml}-Dokumente auch ohne Zuhilfenahme weiterer Metadaten wie XSD-Schemata umwandeln können.

Goessner schlägt ein Mapping zwischen strukturierten \acrshort{xml}-Daten und \acrshort{json} vor, dass möglichst kompakt und simpel sein soll. Dafür werden jedoch Abstriche im Bereich Verlustlosigkeit und Umkehrbarkeit der Konversion hingenommen, d.h. in bestimmten Fällen wie dem Auftreten mehrerer Kindelemente gleichen Namens kann ein Verlust (\emph{hier:} Verlust der Elementreihenfolge) auftreten.~\cite{goessner2006converting}

Ein US-Patent von Williamson~\cite{US20140244692} beschreibt die Überführung von Daten aus einem \acrshort{xml}-Dokument in eine \acrshort{json}-Struktur, indem bestimmte Teile des \acrshort{xml}-Baums selektiert (z.\,B. mittels \acrshort{xpath}-artigen Ausdrücken) und einem Ziel-Schlüssel im resultierenden \acrshort{json}-Objekt zugeordnet werden. Dadurch werden jedoch lediglich Teile der im \acrshort{xml}-Dokument enthaltenen Informationen in die Ausgabe übernommen, sodass verlustloses Round-Tripping unmöglich wird.

In einem Paper vergleicht Wang~\cite{wang2011improving} die beiden Formate und entwickelt einen Übersetzungsalgorithmus zwischen den beiden Formaten. Allerdings wird dabei von datenorientiertem \acrshort{xml} ausgegangen und somit keine Probleme betrachtet, die bei der Umwandlung von spezfischen Strukturen einer dokumentorientierten Auszeichnungssprache wie \acrshort{xml} in das \acrshort{json}-Format entstehen können, beispielsweise die korrekte Abbildung von Mixed Content (vgl. Abschn.~\ref{sec:mixedcontent}). Der von Wang vorgeschlagene Algorithmus ist zudem auch bei datenorientierem \acrshort{xml} unter Umständen verlustbehaftet.\cite[S.~184]{wang2011improving}.

Ein System zur Umstellung des von einer Webapplikation auf JavaScript-Basis genutzten Datenformats wird von Ying und Miller~\cite{ying2013refactoring} vorgeschlagen. Neben der Automatischen Konversion von \acrshort{xml} in \acrshort{json} wird dabei auch der JavaScript-Quelltext der Applikation automatisch an das neue Format angepasst. Der Fokus der Arbeit liegt dabei auf dem Code-Refactoring -- einer genauere Evaluation des eingesetzten Konversionsverfahrens in Bezug auf Sicherheit und Verlustlosgkeit findet nicht statt.

\section{Aufbau der Arbeit}
\label{sec:structure}

Im zweiten Kapitel wird zunächst eine kurze Einführung in die Struktur der Formate \acrshort{xml} und \acrshort{json} gegeben. Zudem werden auch relevante Technologien aus dem Umfeld der Formate beschrieben und die bekannten Angriffe gegen \acrshort{xml}- und \acrshort{json}-Parser erläutert.

Im Anschluss daran wird der Versuchsaufbau beschrieben. Zunächst werden die Anforderungen konkret definiert, d.~h.\ es werden
überprüfbare Kriterien aufgestellt, denen ein Konversionsverfahren genügen
muss. Dazu gehört auch, Verlustlosigkeit genauer zu definieren, d.h.
es wird die Relevanz der durch verschiedene \acrshort{xml}-Strukturen dargestellten
Informationen untersucht sowie möglicherweise zu Datenverlust führende
Probleme bei der Umwandlung beschrieben.

In Abschnitt~\ref{sec:converters} wird die Auswahl der getesteten Konverter vorgestellt und daran anschließend das Überprüfungsverfahren erläutert.

Kapitel~\ref{chap:results} stellt die Ergebnisse der Tests vor und diskutiert diese.
Dabei wird erörtert, ob und welche Verfahren die zuvor genannten Bedingungen
erfüllen.

Das Konversionsverfahren, das bei der Überprüfung die \acrshort{xml}-Strukturen am besten abgebildet hat, wird im folgenden Kapitel dann zu einem vollständig verlustlosen Verfahren weiterentwickelt. Das so verbesserte Verfahren wird mithilfe der Kriterien aus Kapitel~\ref{sec:criteria} einer erneuten Überprüfung unterzogen.

Im Fazit werden die aktuellen Möglichkeiten zur Umwandlung
von \acrshort{xml} in \acrshort{json} und wieder zurück abschließend eingeschätzt und bewertet sowie ein Ausblick auf weitere Forschungfelder gegeben.
