\chapter{Einleitung} \label{chap:intro}
Für den implementationsunabhängigen Austausch von zugleich menschen- als auch
maschinenlesbaren Daten hat sich die \gls{xml}
bewährt. In bestimmten Bereichen wie Web-\acrshortpl{api} hat die \gls{json}
das bewährte \acrshort{xml}-Format inzwischen jedoch überflügelt.

Dabei kann neben der spezifischen Situation auch die eingesetzte
Programmiersprache, die Unterstützung durch das zugrundeliegende Framework
oder die persönliche Präferenz des Entwicklers den Ausschlag geben, welches
Format ein Webservice oder eine Programmbibliothek unterstützt.

Zwecks Interoperabilität zwischen verschiedenen Teilen einer Anwendung kann es
daher notwendig werden, Daten von einem Format temporär in das jeweils andere
zu überführen und später wieder in das ursprüngliche Format zu bringen.

Dabei sollen Daten, die von der Anwendungslogik nicht verändert wurden, auch
nach der Rückübersetzung ins Ursprungsformat unverändert bleiben. Damit dies
gewährleistet ist, darf das zugrunde liegende Konversionverfahren keine
Informationen bei der Umwandlung verwerfen -- es muss also verlustlos % chktex 8
arbeiten.

Das konkrete Konversionsverfahren ist dabei abhängig vom jeweiligen
Ausgangsformat, d.~h.\ die Umwandlungsverfahren für die beiden Richtungen
\begin{enumerate}
    \item $\text{\acrshort{json}} \rightarrow \text{\acrshort{xml}} \rightarrow \text{\acrshort{json}}$ und
    \item $\text{\acrshort{xml}} \rightarrow \text{\acrshort{json}} \rightarrow \text{\acrshort{xml}}$
\end{enumerate}
haben jeweils eigene Anforderungen und sind getrennt voneinander zu betrachten.

Während Abbildung von beliebigen \acrshort{json}-Datenstrukturen in \acrshort{xml} zumindest bei
oberflächlicher Betrachtung trivial erscheint, ist dies beim verlustlosen
Transfer von \acrshort{xml}-Daten ins \acrshort{json}-Format keineswegs der Fall.

Daher sollen in dieser Bachelorarbeit verschiedene Verfahren analysiert
werden, die beliebige \acrshort{xml}-Daten in \acrshort{json} abbilden und aus der resultierenden
\acrshort{json}-Datenstruktur wieder \acrshort{xml}-Dokument erstellen können. Sollte keines
der analysierten Verfahren den vorher aufgestellten Kriterien für eine
zuverlässige und sichere Umwandlung genügen, wird ein eigener
Abbildungsalgorithmus entwickelt, bei dem dies der Fall ist.

\section{Motivation}
\label{sec:motivation}
Das Aufkommen des sogenannten \emph{Web 2.0} und die zunehmenden Vernetzung
durch das \gls{iot} ging mit einer erhöhten
Verfügbarkeit von öffentlichen Web-\acrshortpl{api} einher. Als Datenformat wird dabei
häufig \acrshort{json} oder \acrshort{xml} verwendet.

Neben \acrshort{xml}-basierten Webservices, die beispielsweise das \acrshort{saml}-Framework, SOAP,
oder \gls{xmlrpc} verwenden, wird \acrshort{xml} auch in einer Vielzahl weiterer
Einsatzbereiche eingesetzt. So dient \acrshort{xml} den Dateiformaten \acrshort{rss}/\acrshort{asf}, \acrshort{mathml},
\gls{svg} oder \gls{xhtml} als Basis. Auch die gängigen
Office-Dateiformate -- das \acrfull{odf}, Microsofts % chktex 8
\acrfull{ooxml} und Apples iWorks -- bauen auf \acrshort{xml} auf. % chktex 8

Inzwischen gewinnt jedoch \acrshort{json} vor allem im Mobile-
und Web-Bereich immer mehr an Bedeutung. Laut dem \acrshort{api}-Verzeichnis
\emph{ProgrammableWeb} unterstützten im Jahr 2013 ca. 60\% aller neu
hinzugefügten \acrshortpl{api} das \acrshort{json}-Format, während
\acrshort{xml} im selben Zeitraum lediglich von 37\% der neuen \acrshortpl{api}
unterstützt wurde.~\cite{duvander2013convergence}

\acrshort{json} ist bei bestimmten Aufgaben in puncto Geschwindigkeit und
Ressourcenauslastung deutlich effizienter~\cite{nurseitov2009comparison} als
\acrshort{xml}\@. inzwischen setzen auch einige populäre NoSQL-Datenbanken wie
\emph{CouchDB} oder \emph{MongoDB} auf \acrshort{json} zur Speicherung. Auch
MySQL verfügt seit Version 5.7.8 über einen nativen \acrshort{json}-Datentyp.

Die Umwandlung zwischen den beiden Formaten kann aus vielen Gründen
notwendig werden. Soll beispielsweise ein SOAP-Webservice als moderne
\acrshort{json}-\acrshort{rest}-Ressource angeboten werden, muss zwischen \acrshort{xml} und \acrshort{json} konvertiert
werden. Auch bei der Speicherung von \acrshort{xml}-Daten in den oben genannten NoSQL-Datenbanken
ist dies der Fall. Zudem ist die Unterstützung der Formate durch
Programmiersprachen, Frameworks und Applikationen nicht immer gleich gut.

\section{Verwandte Arbeiten}
% TODO: Add related works
List related work \emph{and} the result of this work\@! What is the relevance of this work concerning your thesis\@? If necessary,\ \emph{emphasize} some words in your text, for example words like \emph{not} or \emph{and} are sometimes crucial for understanding\@.

\section{Zielsetzung}
% TODO: Add contribution
What is your contribution?

\section{Aufbau der Arbeit}
\label{sec:structure}

Nach der Einleitung wird zunächst eine Einführung in die Struktur der
Formate \acrshort{xml} und \acrshort{json} gegeben (wobei die Formate allerdings selbstverständlich
nicht erschöpfend dargestellt werden können, da dies den Rahmen sprengen
würde). Im Zuge dessen werden ebenfalls die wichtigsten Unterschiede zwischen
beiden Formaten aufgezeigt (z.~B.\ \acrshort{json}-Arrays, \acrshort{xml}-Attribute, Duck Typing,
etc.).

Daran anschließend wird zunächst das Ziel konkret definiert, d.~h.\ es werden
überprüfbaren Kriterien aufgestellt, denen ein Konversionsverfahren genügen
muss. Dazu gehört beispielsweise, Verlustlosigkeit genauer zu definieren (vor
allem im Bezug auf Fragen wie \enquote{Sind Kommentare relevant?}). Hier wird
auch ein Abschnitt zu Angriffen (\emph{\gls{xxe}}, \emph{Billion Laughs}, etc.)
eingefügt, gegen den der jeweilige Konverter gewappnet sein soll.

Dann werden die verschiedenen bestehenden Ansätze zur Umformung aufgezählt
und die zugrunde liegenden Konzepte kurz und grob beschrieben.

Der folgende Abschnitt erklärt den Versuchsaufbau (\acrshort{xml}-Testdokumente).
Die verschiedenen Konversionsverfahren werden dann anhand des zuvor
aufgestellten Anforderungskatalogs geprüft.

Anschließend werden die Ergebnisse der Tests vorgestellt und diskutiert.
Dabei wird erörtert, ob und welche Verfahren die zuvor genannten Bedingungen
erfüllen.

Falls keines der Konversionsverfahren den zuvor aufgestellten Kriterien genügt,
folgt der Abschnitt, der einen eigenen Umwandlungsalgorithmus
entwickelt.

Im Diskussions-Abschnitt werden die aktuellen Möglichkeiten zur Umwandlung
von \acrshort{xml} in \acrshort{json} und wieder zurück abschließend eingeschätzt und bewertet.
Dabei soll auch ein Ausblick auf weitere Verbesserungsmöglichkeiten oder
alternative Wege zum Ziel erörtert werden.

Im letzten Kapitel zu verwandten Arbeiten werden weitere Arbeiten zum Thema
beschrieben und Unterschiede zur eigenen Arbeit herausgearbeitet.
