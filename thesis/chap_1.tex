\chapter{Einleitung} \label{chap:intro}
\todo[inline]{TODO: Gesamtes Kapitel nochmal überarbeiten}

Für den implementationsunabhängigen Austausch von zugleich menschen- als auch
maschinenlesbaren Daten hat sich die \gls{xml}
bewährt. In bestimmten Bereichen wie Web-\acrshortpl{api} hat die \gls{json}
das bewährte \acrshort{xml}-Format inzwischen jedoch überflügelt.

Dabei kann neben der spezifischen Situation auch die eingesetzte
Programmiersprache, die Unterstützung durch das zugrundeliegende Framework
oder die persönliche Präferenz des Entwicklers den Ausschlag geben, welches
Format ein Webservice oder eine Programmbibliothek unterstützt.

Zwecks Interoperabilität zwischen verschiedenen Teilen einer Anwendung kann es
daher notwendig werden, Daten von einem Format temporär in das jeweils andere
zu überführen und später wieder in das ursprüngliche Format zu bringen.

Dabei sollen Daten, die von der Anwendungslogik nicht verändert wurden, auch
nach der Rückübersetzung ins Ursprungsformat unverändert bleiben. Damit dies
gewährleistet ist, darf das zugrunde liegende Konversionverfahren keine
Informationen bei der Umwandlung verwerfen -- es muss also verlustlos % chktex 8
arbeiten.

Das konkrete Konversionsverfahren ist dabei abhängig vom jeweiligen
Ausgangsformat, d.~h.\ die Umwandlungsverfahren für die beiden Richtungen
\begin{enumerate}
    \item $\text{\acrshort{json}} \rightarrow \text{\acrshort{xml}} \rightarrow \text{\acrshort{json}}$ und
    \item $\text{\acrshort{xml}} \rightarrow \text{\acrshort{json}} \rightarrow \text{\acrshort{xml}}$
\end{enumerate}
haben jeweils eigene Anforderungen und sind getrennt voneinander zu betrachten.

Während Abbildung von beliebigen \acrshort{json}-Datenstrukturen in \acrshort{xml} zumindest bei
oberflächlicher Betrachtung trivial erscheint, ist dies beim verlustlosen
Transfer von \acrshort{xml}-Daten ins \acrshort{json}-Format keineswegs der Fall.

Ziel ist es daher, ein sicheres und verlustloses Verfahren zur Konversion von
XML-Dokumenten in das JSON-Format zu finden. Dazu werden werden in dieser
Bachelorarbeit verschiedene Konverter anhand zuvor festgelegter Kriterien
analysiert. Da keines der getesteten Konversionsverfahren vollständig
verlustlos arbeitet, wird das vielversprechendste Verfahren angepasst,
sodass alle Anforderungen erfüllt werden.

\section{Motivation}
\label{sec:motivation}
Das Aufkommen des sogenannten \emph{Web 2.0} und die zunehmenden Vernetzung
durch das \gls{iot} ging mit einer erhöhten
Verfügbarkeit von öffentlichen Web-\acrshortpl{api} einher. Als Datenformat wird dabei
häufig \acrshort{json} oder \acrshort{xml} verwendet.

Neben \acrshort{xml}-basierten Webservices, die beispielsweise das \acrshort{saml}-Framework, SOAP,
oder \gls{xmlrpc} verwenden, wird \acrshort{xml} auch in einer Vielzahl weiterer
Einsatzbereiche eingesetzt. So dient \acrshort{xml} den Dateiformaten \acrshort{rss}/\acrshort{asf}, \acrshort{mathml},
\gls{svg} oder \gls{xhtml} als Basis. Auch die gängigen
Office-Dateiformate -- das \acrfull{odf}, Microsofts % chktex 8
\acrfull{ooxml} und Apples iWorks -- bauen auf \acrshort{xml} auf. % chktex 8

Inzwischen gewinnt jedoch \acrshort{json} vor allem im Mobile-
und Web-Bereich immer mehr an Bedeutung (Vgl. Abb.~\ref{fig:xmljsonapis}). Laut dem \acrshort{api}-Verzeichnis
\emph{ProgrammableWeb} unterstützten im Jahr 2013 ca. 60\% aller neu
hinzugefügten \acrshortpl{api} das \acrshort{json}-Format, während
\acrshort{xml} im selben Zeitraum lediglich von 37\% der neuen \acrshortpl{api}
unterstützt wurde.~\cite{duvander2013convergence}

\begin{figure}[h!]
    \begin{center}
        \includestandalone[width=\textwidth]{xml-json-apis-2005-to-2013}
    \end{center}
    \caption{Sowohl JSON als auch XML wurden 2013 von jeweils mehr als 45\% der Web-\acrshortpl{api} unterstützt.}
    \label{fig:xmljsonapis}
\end{figure}


\acrshort{json} ist bei bestimmten Aufgaben im Bezug auf Geschwindigkeit und
Ressourcenauslastung deutlich effizienter~\cite{nurseitov2009comparison} als
\acrshort{xml}\@. Inzwischen setzen auch einige populäre NoSQL-Datenbanken wie
\emph{CouchDB} oder \emph{MongoDB} auf \acrshort{json} zur Datenspeicherung. Auch
MySQL verfügt seit Version 5.7.8 über einen nativen \acrshort{json}-Datentyp.

Die Umwandlung zwischen den beiden Formaten kann aus vielen Gründen
notwendig werden. Soll beispielsweise ein SOAP-Webservice als moderne
\acrshort{json}-\acrshort{rest}-Ressource angeboten werden, muss zwischen \acrshort{xml} und \acrshort{json} konvertiert
werden. Auch bei der Speicherung von \acrshort{xml}-Daten in den oben genannten NoSQL-Datenbanken
ist dies der Fall. Zudem ist die Unterstützung der Formate durch
Programmiersprachen, Frameworks und Applikationen nicht immer gleich gut.

\section{Verwandte Arbeiten}

Muschett, Salz und Schenker schlugen 2011 in ein Format zur Repräsentation beliebiger JSON-Daten in XML 1.0 names JSONx vor. In dem inzwischen ausgelaufenen \gls{idraft} werden eine Reihe von Regeln festgelegt um Instanzen der verschiedenen JSON-Daten in XML-Strukturen zu konvertieren.~\cite{jsonx}

In der vorliegenden Arbeit soll die Konversion in umgekehrter Richtung untersucht werden, d.h. die Umwandlung beliebiger XML-Dokumente in das JSON-Format.

Ein weiteres XML-basiertes Format, dass das Datenmodell der \acrfull{json} in XML nachbildet und dadurch die verlustlose Abbildung beliebiger JSON-Daten in XML ermöglicht, ist JsonXML (JXML)~\cite{jxml}.

David Lee~\cite{lee2011jxon} entwickelte 2011 ein Verfahren, um anhand annotierter XSD-Schemata XSLT-Dateien zu erzeugen, die in der Lage sind, bidirektional zwischen XML und JXML zu konvertieren.  Damit ist also -- mit einem Umweg über JXML -- die verlustlose Umwandlung von XML in JSON und wieder zurück möglich.
Durch die Nutzung des jeweiligen XSD-Schema der zu konvertierenden XML-Dokumente sind bei der Umwandlung die im XML-Dokument genutzten Datentypen bekannt, wodurch ein zuverlässigen Mapping auf die nativen JSON-Typen möglich wird.

Im Gegensatz dazu ist eine Anforderung an das in der vorliegenden Arbeit zu entwickelnde Konversionsverfahren jedoch die Konversion beliebiger XML-Dokumente, ohne Zuhilfenahme weiterer Metadaten wie XSD-Schemata.

Goessner schlägt ein Mapping zwischen strukturierten XML-Daten und JSON vor, dass möglichst kompakte und simpel sein soll. Dafür werden jedoch Abstriche im Bereich Verlustlosigkeit und Umkehrbarkeit der Konversion hingenommen, d.h. in bestimmten Fällen wie dem Auftreten mehrerer Kindelemente gleichen Namens kann ein Verlust (\emph{hier:} Verlust der Elementreihenfolge) auftreten.~\cite{goessner2006converting}

In einem Paper vergleicht Wang die beiden Formate und entwickelt einen Übersetzungsalgorithmus zwischen den beiden Formaten~\cite{wang2011improving}. Allerdings wird dabei von datenorientiertem XML ausgegangen und somit keine Probleme betrachtet, die bei der Umwandlung von spezfischen Strukturen einer dokumentorientierten Auszeichnungssprache wie XML in das JSON-Format entstehen können, beispielsweise die korrekte Abbildung von Mixed Content (vgl. Abschn.~\ref{sec:mixedcontent}). Der von Wang vorgeschlagene Algorithmus zudem auch bei datenorientierem XML unter Umständen verlustbehaftet.\cite[S.~184]{wang2011improving}.

Ein System zur Umstellung des von einer Webapplikation auf JavaScript-Basis genutzten Datenformats wird von Ying und Miller vorgeschlagen~\cite{ying2013refactoring}. Neben der Automatischen Konversion von XML in JSON wird dabei auch der JavaScript-Quelltext der Applikation automatisch an das neue Format angepasst. Der Fokus der Arbeit liegt dabei auf dem Code-Refacoring -- einer genauere Evaluation des eingesetzten Konversionsverfahrens im Bezug auf Sicherheit und Verlustlosgkeit findet nicht statt.

% TODO: Add related works

\todo[inline]{TODO: List related work \emph{and} the result of this work\@! What is the relevance of this work concerning your thesis\@? If necessary,\ \emph{emphasize} some words in your text, for example words like \emph{not} or \emph{and} are sometimes crucial for understanding\@.}

\section{Zielsetzung}
% TODO: Add contribution
\todo[inline]{TODO: What is your contribution?}

\section{Aufbau der Arbeit}
\label{sec:structure}

Im zweiten Kapitel wird zunächst eine kleine Einführung in die Struktur der Formate \acrshort{xml} und \acrshort{json} gegeben. Zudem Werden auch relevante Technologien aus dem Umfeld der Formate beschrieben.  Zudem werden die bekannten Angriffe gegen \acrshort{xml}- und \acrshort{json}-Parser erläutert.

Im Anschluss daran wird der Versuchsbau beschrieben. Zunächst wird die Anforderungen konkret definiert, d.~h.\ es werden
überprüfbaren Kriterien aufgestellt, denen ein Konversionsverfahren genügen
muss. Dazu gehört auch, Verlustlosigkeit genauer zu definieren, d.h.
es wird die Relevanz der durch verschiedene XML-Strukturen dargestellten
Informationen untersucht sowie möglicherweise zu Datenverlust führende
Probleme bei Umwandlung beschrieben.

Danach wird die Auswahl der getesteten Konverter vorgestellt und das Überprüfungsverfahren erläutert.

Anschließend werden die Ergebnisse der Tests vorgestellt und diskutiert.
Dabei wird erörtert, ob und welche Verfahren die zuvor genannten Bedingungen
erfüllen.

Das Konversionsverfahren, das bei der Überprüfung die XML-Strukturen am besten abgebildet hat, wird im folgenden Kapitel dann zu einem vollständig verlustlosen Verfahren weiterentwickelt. Das so verbesserte Verfahren wird mithilfe der Kriterien aus Kapitel~\ref{sec:criteria} einer erneuten Überprüfung unterzogen.

Im Diskussions-Abschnitt werden die aktuellen Möglichkeiten zur Umwandlung
von \acrshort{xml} in \acrshort{json} und wieder zurück abschließend eingeschätzt und bewertet und ein Ausblick auf weitere Forschungfelder gegeben werden.
