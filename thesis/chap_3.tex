\chapter{Versuchsaufbau}
\label{chap:impl}

\begin{samepage}
Im Folgenden wird der Versuchsaufbau, bestehend aus
\begin{enumerate}
    \item{} der Erstellung einer Reihe von Kriterien zur Bewertung der Verlustlosigkeit und Sicherheit von Konversionverfahren,
    \item{} einer Auswahl von zu überprüfenden Konverter, und
    \item{} der Methodik zur Überprüfung der Konversionsverfahren auf Einhaltung der zuvor festgelegten Kriterien
\end{enumerate}
vorgestellt.
\end{samepage}

\section{Bewertungskriterien für Konversionsverfahren}
\label{sec:criteria}

Ziel ist das Finden eines Konversionsverfahrens, welches sowohl \emph{sicher} als auch \emph{verlustlos} arbeitet. Dazu müssen beide Anforderungen zunächst genauer definiert werden.

\subsection{Sicherheit}
\label{sec:criteria-security}

Die Sicherheit eines Konverters hängt in erster Linie von der Sicherheit des eingesetzten \acrshort{xml}-Prozessors ab. Dieser darf gegen keinen der im Abschnitt~\ref{sec:xmlattacks} beschriebenen Angriffe auf \acrshort{xml}-Parser verwundbar sein.

Im Bezug auf Sicherheit stellt die Unterstützung eines \acrshort{xml}-Parsers für Entities das größte Einfallstor für Angriffe dar. Viele der Angriffsvektoren sind nur dann möglich, wenn der Parser General bzw. Parameter Entities auswertet und expandiert.

Das \acrfull{owasp} empfiehlt daher, die Unterstützung für \acrfullpl{dtd} komplett zu deaktivieren, da dies sowohl \acrshort{xxe}-Angriffe als auch \acrshort{dos}-Attacken wie \emph{Billion Laughs} oder \emph{Quadratic Blowup} wirksam verhindert~\cite[Abschn.~1.1]{owasp2017xxeprevention}. Diese Gegenmaßnahme empfiehlt auch Nazim Lala von Microsofts IIS Security Team~\cite{lala2013handlinguntrustedxml}.

Aus Sicherheitsgründen ist die Verarbeitung von \glspl{dtd} für die in der vorliegenden Arbeit überprüften Konverter daher nicht erforderlich.

Dies gilt jedoch nicht für die vordefinierten General Entities \enquote{\mintinline{xml}{&amp;}}, \enquote{\mintinline{xml}{&lt;}}, \enquote{\mintinline{xml}{&gt;}}, \enquote{\mintinline{xml}{&apos;}} und \enquote{\mintinline{xml}{&quot;}}~\cite[Abschn.~4.6]{xml} bzw. Character References~\cite[Abschn.~4.1]{xml}, da diese nicht innerhalb einer \gls{dtd} angegeben werden müssen und sie zum Quotieren von Markup in Text-Knoten und Attributen nötig sind.

\subsection{Verlustlosigkeit}
\label{sec:criteria-lossless}

\begin{samepage}
Damit ein Konversionsprozess als \emph{verlustlos} angesehen werden kann, müssen folgende Bedingungen erfüllt sein:
\begin{enumerate}
    \item{} Eingabedokument und Ausgabedokument müssen logisch äquivalent sein.\label{item:ioequal}
    \item{} Sowohl das \acrshort{json}-Zwischenprodukt als auch das \acrshort{xml}-Ausgabedokument müssen gültige Dokumente sein.\label{item:outputwellf}
\end{enumerate}
\end{samepage}

Logisch äquivalent (vgl. Bed. \ref{item:ioequal}) heißt nicht, dass das \acrshort{xml}-Dokument nach dem Kon\-versions\hyp{}Round\hyp{}Trip bitidentisch mit dem Ursprungsdokument sein muss -- es ist ausreichend, dass die Struktur der beiden Dokumente übereinstimmt. Daher werden beide Dokumente zunächst kanonisiert\footnote{Vgl. Abschnitt \ref{sec:c14n}} und im Anschluss verglichen. Stimmt die kanonische Form beider \acrshort{xml}-Dokumente überein, war die Konversion verlustlos.

\begin{figure}[h!]
\begin{definition}[Verlustlosigkeit von Konversion]\label{def:lossless}
    Sei $V \coloneqq \{x \mid x\text{ ist ein valider }\allowbreak{}\text{\acrshort{json}-Wert}\}$ und $W \coloneqq \{x \mid x\text{ ist ein wohlgeformtes \acrshort{xml}-Dokument}\}$.

    Ein Konversionverfahren $K = (f_{enc}, f_{dec})$ heißt \emph{verlustlos} genau dann wenn
    \begin{align}
        (f_{dec}\mathrel{\circ} f_{enc})(x) \; &\mathbin{\stackrel{\mathmakebox[\widthof{=}]{\mathrm{c14n}}}{=}} \; x\\
        f_{enc}: W &\mapsto V\\
        f_{dec}: V &\mapsto W
    \end{align}
    für alle $x \in W$.
\end{definition}
\end{figure}

Für die Feststellung, dass ein ein Konversionsverfahren verlustbehaftet ist, ist es ausreichend \emph{ein einzelnes} wohlgeformtes \acrshort{xml}-Dokument zu finden, das das Verfahren nicht verlustlos konvertiert. Um jedoch hinreichend zu beweisen, dass ein Konversionsverfahren vollständig verlustlos arbeitet und somit die Bedingungen aus Def.~\ref{def:lossless} erfüllt, müsste die Umwandlung für \emph{alle} wohlgeformten \acrshort{xml}-Dokumente überprüft werden. Da die Menge der möglichen wohlgeformten \acrshort{xml}-Dokumente  -- auch bei Beachtung der Einschränkungen aus Abschnitt~\ref{sec:criteria-security} -- unendlich ist, ist eine deduktive Art der Beweisführung jedoch nicht möglich.

Um zu entscheiden, ob ein Konversionsverfahren verlustlos ist, verfolgt die vorliegende Arbeit daher einen empirisch-induktiven Ansatz: Wenn für eine ausreichend große Anzahl verschiedener Fälle gezeigt wird, dass die Bedingung erfüllt ist\footnote{\emph{hier:} die \acrshort{xml}-Dokumente vor und nach der Konversion sind logisch äquivalent}, kann eine allgemeine Erfüllung der Bedingung induziert werden~\cite[S.~2]{rudner1953judgments}.

Daher wird die Verlustlosigkeit einer Konversion im Folgenden anhand konkreter \acrshort{xml}-Strukturen und Problemstellungen erläutert. Diese bilden die Grundlage für die anschließende Erstellung konkreter Testfälle in Form von \acrshort{xml}-Dokumenten (vgl.~Abschn.~\ref{sec:method-conv}).

\subsubsection{Elemente und Attribute}

Als Kerninhalt jedes \acrshort{xml}\hyp{}Dokuments müssen Elemente und deren Beziehung zueinander erhalten bleiben. Ebenso dürfen Attribute bei der Konversion nicht verloren gehen. Auch die Tag\hyp{}Namen der Elemente transportieren relevante Information, daher müssen sie beibehalten werden -- dies gilt auch für den Tag\hyp{}Namen des Wurzelelements.

\subsubsection{Namespaces}

Namespaces bieten eine Möglichkeit, Teilen eines \acrshort{xml}-Dokuments eine bestimmte Semantik zuzuweisen. Die Zuordnung zwischen Elementen und Namespaces im Dokument darf daher nicht verändert werden.

Obwohl Namespace\hyp{}Prefixe eigentlich frei gewählt werden können, ist es möglich, dass auch sie wichtige Informationen enthalten. Dies wäre beispielsweise dann der Fall, wenn ein im Dokument vorhandener \acrshort{xpath}\hyp{}Ausdruck ein Namenraumprefix referenziert~\cite[Abschn. 4.4]{c14n}. Bei einer Umbenennung von Prefixen wäre die korrekte Evaluation des \acrshort{xpath}\hyp{}Ausdrucks nicht mehr möglich. Die Prefix-Bezeichner müssen also erhalten bleiben.

\subsubsection{Character Data}
\label{sec:cdata}

\emph{Character Data} ist in \acrshort{xml}-Dokumente ein ebenso wichtiger Träger von Information wie Elemente. Dabei kann er im Inhaltsteil von Elementen in zwei Varianten vorkommen: Als normaler Text-Knoten oder als CDATA-Abschnitt.

Mithilfe von CDATA-Abschnitten lässt sich Text, der Markup-Zeichen wie beispielsweise das Kleiner-als-Zeichen\footnote{Unicode-Codepoint \texttt{U+003C}: \texttt{LESS-THAN SIGN}} enthält, direkt in ein \acrshort{xml}-Dokument einbetten, ohne dass diese Zeichen als Markup interpretiert werden.

Dies ist vor allem dann sinnvoll, wenn es unpraktikabel ist, alle Markup-Zeichen im Text einzeln durch die jeweilige Zeichen- oder Entity-Referenz zu ersetzen. CDATA stellt somit eine weitere Möglichkeit dar, Zeichendaten in einem \acrshort{xml}-Dokument anzugeben.~\cite[Abschnitt~2.4]{xml}

Der Unterschied zwischen Zeichendaten aus CDATA-Abschnitten und solchen, bei denen dies nicht der Fall ist, ist jedoch lediglich ein syntaktischer. Daher werden bei der \acrlong{c14n} alle CDATA-Abschnitte im Eingabedokument durch den entsprechenden \emph{Character Content} ersetzt~\cite[Abschnitt~2.1]{c14n}.

Für die Verlustlosigkeit der Konversion ist es somit unerheblich, ob die CDATA-Abschnitte im Ursprungsdokument als solche erhalten bleiben, oder lediglich die Zeichendaten beibehalten werden.

\subsubsection{Kommentare}

\acrshort{xml} verfügt über die Möglichkeit, Dokumente mit Kommentaren zu versehen. Diese sind jedoch nicht Teil der Zeichendaten des \acrshort{xml}-Dokuments. Die Möglichkeit, Kommentare programmatisch auszuwerten, können \acrshort{xml}-Parser zwar bereitstellen, dies ist jedoch optional.

Zudem ist auch bei der Implementierung der \acrlong{c14n} die Unterstützung von \emph{Kanonischem \acrshort{xml} mit Kommentaren} lediglich empfohlen, während die Möglichkeit der Umwandlung in \emph{Kanonisches \acrshort{xml}} ausschließlich aller Kommentare zwingend erforderlich ist.~\cite[Abschnitt~2.1]{c14n}

Folglich ist es nicht nötig, dass sich die Kommentare im \acrshort{xml}-Eingabedokument nach dem \acrshort{xml}$\rightarrow$\acrshort{json}$\rightarrow$\acrshort{xml} Roundtrip auch in der Ausgabe wiederfinden.

\subsubsection{\acrfullpl{pi}}

Wie bereits \acrshort{sgml} unterstützt \acrshort{xml} die Einbettung von Anweisungen, die für die verarbeitetende Applikation bestimmt sind. Diese werden \acrfull{pi} genannt (vgl. Abschn.~\ref{sec:xmlbasics}).

In der Praxis werden \glspl{pi} eher selten eingesetzt. Genutzt werden sie unter anderem um Darstellungsinformationen in Form von Stylesheets mit \acrshort{xml}-Dokumenten zu verknüpfen~\cite[Abschnitt 4]{xmlstylesheet}.

Ein weiteres prominentes Beispiel für eine Praxisanwendung ist die Microsoft Office Suite, die seit der Version 2003 Office-Dokumente als einzelne \acrshort{xml}-Dateien speichern kann. Diese setzen die unspezifische Dateiendung \texttt{*.xml} ein und würden daher mit einem generisch \acrshort{xml}-Anzeigeprogramm geöffnet werden. Mithilfe der Processing Instruction \enquote{\mintinline{xml}{<?mso-application progid="Word.Document"?>}} werden Windows- bzw. Internet Explorer angewiesen, diese als Microsoft Word-Document zu behandeln.~\cite[Abschnitt 3.2]{tverskov2008understandingpi}

Auch die \acrshort{xslt}-Stylesheets des DocBook-Formats nutzen Processing Instructions, um spezielle Formatierungen für die verschiedenen Ausgabeformate festzulegen.~\cite[{Kapitel \enquote{User Reference: PIs}}]{docbookxsl}

Ebenso wie Kommentare sind sie nicht Teil der Zeichendaten eines \acrshort{xml}-Dokuments, müssen jedoch laut Spezifikation zwingend vom \acrshort{xml}-Parser an die verarbeitende Applikation weitergereicht werden~\cite[Abschn.~2.6]{xml}.

Auch bei der \acrlong{c14n} bleiben die \glspl{pi} erhalten.~\cite[Abschnitt 2.3]{c14n}

\subsubsection{Dokumentordnung}

\paragraph{Elementordnung}

Die Spezifikation der \acrshort{c14n} bezieht sich bezüglich der Ordnung auf die \acrshort{w3c}-Empfehlung zur \acrfull{xpath}~\cite[Abschnitt~2.2]{c14n}.

\begin{foreigndisplayquote}{english}[{~\cite[Abschnitt~5]{xpath}}]
    There is an ordering, document order, defined on all the nodes in the document corresponding to the order in which the first character of the \acrshort{xml} representation of each node occurs in the \acrshort{xml} representation of the document after expansion of general entities. Thus, the root node will be the first node. Element nodes occur before their children. Thus, document order orders element nodes in order of the occurrence of their start-tag in the \acrshort{xml} (after expansion of entities).
\end{foreigndisplayquote}

Folglich muss die Reihenfolge der Element-Nodes bei der Konversion beibehalten werden.

\paragraph{Ordnung von Attributen und Namespaces}

Der Reihenfolge, in der die Attribute eines Elements im Start-Tag bzw. Leeres-Element-Tag angegeben wurden, kommt laut \acrshort{xml}-Spezi\-fikation keine Bedeutung zu.~\cite[Abschnitt~3.1]{xml}

Dies wird auch von der \acrshort{xpath}-Spezifikation untermauert, die die Reihenfolge von Name\-space-Deklarationen und Attributen als implementierungsabhängig festlegt.~\cite[Abschnitt~5]{xpath}

Die Reihenfolge der Attribute eines Elements nach einem \acrshort{xml}$\rightarrow$\acrshort{json}$\rightarrow$\acrshort{xml}-Roundtrip ist daher beliebig und muss nicht mit der Reihenfolge vor der Umwandlung identisch sein.

\subsubsection{Whitespace}

\acrshort{xml}-Parser müssen grundsätzlich jeglichen Whitespace -- also Leerzeichen, Tabs und Zeilenumbrüche --  innerhalb des Wurzelelements eines Dokuments an die verarbeitende Applikation weiterreichen~\cite[Abschn.~2.10]{xml}, während Whitespace außerhalb des Wurzelelements keine Bedeutung zukommt~\cite[Abschnitt~2.1]{c14n}.

\begin{samepage}
    Ausgenommen hiervon sind jedoch bestimmte Sonderfälle, in denen Whitespace durch den \acrshort{xml}-Prozessor \emph{normalisiert} wird:
    \begin{enumerate}
        \item{} Zeilenenden, d.h. die verschiedenen Kombinationen der Wagenrücklauf- und \linebreak{}Zeilenvorschub-Zeichen, werden zu einem einfachen Zeilenvorschub umgewandelt ~\cite[Abschn.~2.11]{xml}.
        \item{} In Attributen vorkommende Whitespace-Zeichen (keine Character References) werden zu Leerzeichen umgewandelt~\cite[Abschn.~3.3.3]{xml}
    \end{enumerate}
\end{samepage}

Zwar kann Whitespace durch \glspl{dtd} oder \acrshort{xsd}-Schemata auch als unsignifikant markiert werden~\cite{page2005whitespace}, dies wird jedoch nur von \emph{validierenden} \acrshort{xml}-Prozessoren beachtet und geht somit über den Rahmen der vorliegenden Arbeit hinaus.

\subsubsection{Mixed Content}
\label{sec:mixedcontent}

Eine Besonderheit von \acrshort{xml} und auch \acrshort{sgml} ist, dass die Spezifikation sogenannten \emph{Mixed Content} erlaubt. Dieser liegt vor, wenn ein Element sowohl Character Data als auch Kindelemente enthält~\cite[Abschnitt 3.2.2]{xml}.

\begin{example}[Mixed Content]~
    \begin{minted}{xml}
<mixed>This is an element<br /> containing <emph>mixed content</emph>.<mixed>
    \end{minted}
    \captionof{figure}{Mit Elementen durchsetzte \emph{Character Data} wird \emph{Mixed Content} genannt.}
\end{example}

\emph{Mixed Content} stellt \acrshort{xml}-Parser vor besondere Herausforderungen~\cite{mcgrath2002mixedcontent}. Zudem existiert kein \acrshort{json}-Äquivalent von \emph{Mixed Content}, was eine Konversion erschwert.

\subsubsection{Typinferenz bei der Konversion zu \acrshort{json}}

Zwar ist es möglich, mittels einer \acrfull{xsd} die in einem \acrshort{xml}-Dokument enthaltenten Datentypen genauer festzulegen, direkte syntaktische Unterstützung für Zahlen bietet \acrshort{xml} jedoch im Gegensatz zur \acrlong{json} nicht.

\begin{samepage}
Wird kein Schema verwendet bzw. legt ein Schema den Datentyp eines Elements nicht anderweitig fest, so ist es standardmäßig vom Typ \texttt{xsd:anyType}:
\begin{foreigndisplayquote}{english}[{~\cite[Abschnitt 2.2.1.1]{xmlschema11-1}}]
    A special complex type definition, (referred to in earlier versions of this specification as `the ur-type definition') whose name is \textbf{anyType} in the XSD namespace, is present in each ·\emph{XSD schema}·. The \textbf{definition of anyType} serves as default type definition for element declarations whose \acrshort{xml} representation does not specify one.
\end{foreigndisplayquote}
\end{samepage}

Elemente dieses Typs unterliegen keinen Beschänkungen. Daher ist es ohne Zuhilfenahme eines Schemas nicht möglich, Aussagen über den Wertebereich eines Elements, eines Attributs oder einer Text Node zu treffen.

\begin{example}[Type Inference]~
    Das folgende \acrshort{xml}-Dokument, enthält einen numerischen Wert, wobei in \acrshort{xml} kein syntaktischer Unterschied zwischen Strings und Zahlen besteht.
    \begin{center}
        \mintinline{xml}{<price>5.99</price>}
    \end{center}
Die \acrshort{json}-Entsprechung des \acrshort{xml}-Dokuments \emph{ohne} Typinferenz enthält den Zahlenwert als String:
    \begin{center}
        \mintinline{json}{{ "price": "5.99" }}
    \end{center}
Wird jedoch Typinferenz verwendet, enthält die \acrshort{json}-Struktur den Zahlenwert als \texttt{Number}:
    \begin{center}
        \mintinline{json}{{ "price":  5.99  }}
    \end{center}
\end{example}

Ist das Schema eines \acrshort{xml}-Dokuments nicht bekannt, scheint es daher naheliegend, die Datentypen aus den im Dokument enthaltenen Werten abzuleiten. Enthält eine Text Node oder ein Attribut beispielsweise die Zeichenkette \mintinline{text}{123}, könnte daraus auf einen numerischer Datentyp geschlossen werden. Dieses Prinzip der \emph{Type Inference} nutzt beispielweise Microsoft im Rahmen des \emph{.NET Frameworks} um das \acrshort{xml}-Schema auf Basis von einem oder mehreren \acrshort{xml}-Dokumenten zu \enquote{erraten}.~\cite{msdn2017inferxmlschema}

Im Rahmen der \acrshort{xml} zu \acrshort{json}-Konversion ist es zwar wünschenswert, die nativen Datentypen der \acrshort{json}-Spezifikation voll auszunutzen, ein solches Vorgehen bringt jedoch mehrere Probleme mit sich:

\paragraph{Fehlerkennung von Typen}

Die oben beschriebene Vorgehensweise zur Typableitung kann zur Fehlerkennung von Datentypen führen.

Darf eine Text Node beispielsweise eigentlich beliebige Zeichen enthalten, enthält aber \emph{zufälligerweise} ausschließlich Ziffern, würde fälschlicherweise ein Zahlentyp erkannt werden. Dies könnte zu Problemen mit der verarbeitetenden Applikation führen, die stattdessen eine Zeichenkette erwartet.

\paragraph{Einschränkungen durch Wertebereiche}

Ein weiteres Problem bei der \emph{Type Inference} kann durch die unterschiedlichen Wertebereichsgrenzen der verschiedenen Datentypen entstehen.  Während Zeichenketten in vielen Programmiersprachen mehrere tausend Zeichen lang sein dürfen, ist der Wertebereich von numerischen Datentypen in der Regel deutlich eingeschränkter. Wird ein Zahlenwert vom Konversionsprogramm also in einen nativen Datentyp umgewandelt, für den ein kleinerer Wertebereich gilt als durch Strings darstellbar sind, führt dies zu Fehlern oder Informationsverlust.

Die Programmiersprache \emph{JavaScript} erlaubt beispielsweise Strings mit einer Länge von bis zu $n=2^{53}-1 \approx 9 \cdot 10^{15}$ Zeichen, d.h. als Strings abgelegte Zahlen können rund 9 Billarden Stellen haben.~\cite[Abschnitt 6.1.4]{ecma262} Für Zahlen im Dezimalsystem entspräche dies dem Wertebereich $\left\{x \in \mathbb{Z} \mid -\left(10^{n-1}-1\right) \leq x \leq 10^{n}-1\right\}$.

Für den Datentyp \texttt{Number} nutzt JavaScript 64-bit-Fließkommazahlen nach dem \acrshort{ieee}-754-Standard\footnote{Fließkommazahlen mit doppelter Genauigkeit (\texttt{binary64}-Typ nach \acrshort{ieee}-754)}~\cite[Abschnitt 4.3.20]{ieee754,ecma262}. Für Ganzzahlen gilt daher der \enquote{sichere} Wertebereich $\left\{x \in \mathbb{Z} \mid \abs{x} \leq 2^{53}-1\right\}$~\cite[Abschnitt 20.1.2.5]{ecma262}. Dies entspricht einer Zahl mit lediglich $\floor{\log_{10}\left(2^{53}-1\right)}+1 = 16$ Ziffern im Dezimalsystem.

Zahlen außerhalb dieses Wertebereichs können nicht mehr fehlerfrei eingesetzt werden. Auf den möglicher Informationsverlust bei dem Einsatz von Zahlentypen in \acrshort{json} wird in der \acrshort{ietf}-Spezifikation daher explizit hingewiesen~\cite[Abschn.~6]{rfc7159}.

\begin{figure}[b!]
\begin{example}[Informationsverlust durch Typumwandlung in JavaScript]
    Die Umwandlung von numerischen Zeichenketten in den \texttt{Number}-Typ kann in JavaScript bei Zahlen $> 2^{53}-1$ zu Problemen führen.
    \inputminted[firstline=2,firstnumber=1,mathescape]{javascript}{typeinference.js}
\end{example}
\end{figure}

Ähnliches gilt für Konversionsverfahren, die aus Werten wie \enquote{\mintinline{text}{yes}} und \enquote{\mintinline{text}{true}} den Boole'schen Datentyp ableiten: Nach dem Konversion ist nicht mehr feststellbar, ob der Ursprungswert nun \enquote{\mintinline{text}{yes}} oder \enquote{\mintinline{text}{true}} lautete.

\subsubsection{Unterstützung des Zeichenbereichs}

Die \acrshort{xml}-Spezifikation erlaubt eine große Anzahl verschiedener Unicode-Zeichen weit über den ASCII-Raum hinaus ~\cite[Regel~2]{xml}. Da bei der Konvertierung keine Zeichendaten verloren gehen dürfen (vgl. Abschn.~\ref{sec:cdata}), müssen Konverter in der Lage sein, diese Zeichen zu \acrshort{json} zu übersetzen und ggf. korrekt zu quotieren.

Dies gilt auch für die Zeichen, von deren Einsatz \acrshort{xml}-Spezifikation ausdrücklich abgeraten wird~\cite[Abschn.~2.2]{xml}, da es sich dennoch um wohlgeformtes \acrshort{xml} handelt.

\section{Auswahl der \acrshort{xml}-\acrshort{json}-Konverter}
\label{sec:converters}

\begin{figure}[hb!]
    \begin{threeparttable}
        \captionof{table}{Übersicht der überprüften Konverter}
        \begin{tabularx}{\textwidth}{p{2.65cm}Xp{1.6cm}p{1.5cm}p{2.1cm}}
            \rowcolor{white} % strange hack
            \toprule
            {\fontfamily{rubflama}\selectfont{}\bfseries Konverter} &
            {\fontfamily{rubflama}\selectfont{}\bfseries Autor} &
            {\fontfamily{rubflama}\selectfont{}\bfseries Lizenz} &
            {\fontfamily{rubflama}\selectfont{}\bfseries Sprache} &
            {\fontfamily{rubflama}\selectfont{}\bfseries Version}\\
            \midrule
            Cobra vs\newline Mongoose\tnote{a} & {Paul Battley} & MIT & Ruby & \texttt{0.0.2}\newline 27.06.2006\\
            \rowcolor{rubgray!80}
            GreenCape \acrshort{xml} Converter\tnote{b} & {Niels Braczek} & MIT & PHP & \texttt{a830542}\newline 02.07.2015\\
        Json-lib\tnote{c} & {Andres Almiray\tnote{1}} & Apache~2.0 & Java & \texttt{2.4}\newline 14.12.2010\\
            \rowcolor{rubgray!80}
            \acrshort{jsonml}\tnote{d} & {Stephen M. McKamey} & MIT & JavaScript & \texttt{2.0.0}\newline 09.04.2016\\
            JXON\tnote{e} & {Martin Raifer, Mozilla} & GNU GPL~3.0 & JavaScript & \texttt{2.0.0-beta.4}\newline 22.11.2016\\
            \rowcolor{rubgray!80}
            Json.NET\tnote{f} & {James Newton-King} & MIT & C\# & \texttt{10.0.3}\newline 18.06.2017\\
            org.json.XML\tnote{g} & {Sean Leary / JSON.org} & MIT & Java & \texttt{20160810}\newline 10.08.2016\\
            \rowcolor{rubgray!80}
            Pesterfish\tnote{h} & {Jacob Smullyan} & MIT & Python & \texttt{1578db9}\newline 22.11.2010\\
            x2js\tnote{i} & {Abdulla G. Abdurakh\-manov} & Apache~2.0 & JavaScript & \texttt{185e410}\newline 04.01.2016\\
            \rowcolor{rubgray!80}
            x2js (Fork)\tnote{j} & {Sander Saares / Axinom\tnote{2}} & Apache~2.0 & JavaScript & \texttt{3.1.0}\newline 05.12.2016\\
            xmljson\tnote{k} & {S. Anand} & MIT & Python & \texttt{0.1.7}\newline 09.05.2017\\
            \rowcolor{rubgray!80}
            \bottomrule
        \end{tabularx}
        {\footnotesize
        \begin{tablenotes}[para]
            \item[1] Basiert auf Software von Douglas Crockford.
            \item[2] Fork der \emph{x2js}-Bibliothek von Abdulla G. Abdurakhmanov.
            \item[a] \url{https://rubygems.org/gems/cobravsmongoose}
            \item[b] \url{https://github.com/GreenCape/xml-converter}
            \item[c] \url{http://json-lib.sourceforge.net/}
            \item[d] \url{http://www.jsonml.org/}
            \item[e] \url{https://github.com/tyrasd/jxon}
            \item[f] \url{https://github.com/stleary/JSON-java}
            \item[g] \url{https://github.com/JamesNK/Newtonsoft.Json}
            \item[h] \url{https://bitbucket.org/smulloni/pesterfish/}
            \item[i] \url{https://github.com/abdmob/x2js}
            \item[j] \url{https://github.com/x2js/x2js}
            \item[k] \url{https://github.com/sanand0/xmljson}
        \end{tablenotes}}
    \end{threeparttable}
\end{figure}

Inzwischen sind eine große Anzahl an Applikationen und Programmbibliotheken für die Umwandlung zwischen \acrshort{json} und \acrshort{xml} verfügbar.

\begin{samepage}
Aufgrund des in dieser Arbeit zur Anwendung kommenden Prüfverfahrens konnten jedoch nur solche Konverter betrachtet werden, die
\begin{enumerate}
    \item{} \acrshort{xml}-Dokumente in \acrshort{json} umwandeln können und
    \item{} aus den so erhaltenen \acrshort{json}-Strukuren wieder \acrshort{xml}-Dokumente erstellen können.
\end{enumerate}
\end{samepage}

Da für viele \acrshort{xml}-Strukturen kein Äquivalent in der \acrshort{json}-Spezifikation existiert, gibt es meist verschiedene Möglichkeiten der Abbildung von \acrshort{xml}-Inhalten in \acrshort{json}.
Die gebräuchlichsten Varianten haben sich in Form verschiedener Konvertierungskonventionen herausgebildet, die jeweils beschreiben, wie \acrshort{xml}-Strukturen in \acrshort{json} dargestellt werden~\cite{open311-conventions}. Dabei unterscheidet sich das anhand der verschiedenen Konventionen produzierte \acrshort{json} zum Teil stark, beispielsweise darin, welcher \acrshort{json}-Container\-typ zur Darstellung eines \acrshort{xml}-Elements genutzt wird oder wie Attribute dargestellt werden. Die für die Analyse im Rahmen der vorliegenden Arbeit ausgewählten Konverter sollten daher nach Möglichkeit verschiedene Umwandlungsverfahren und Konventionen implementieren.

Da die Formate insbesondere im Webbereich eingesetzt werden und Webservices sowie \acrshortpl{api} ein relevantes Einsatzfeld für \acrshort{xml}-\acrshort{json}-Konversion darstellen, wurde bei der Auswahl zudem darauf geachtet, ein breites Spektrum verschiedener populärer Programmiersprachen aus diesem Bereich abzudecken. Sowohl die klassischen Programmiersprachen für Web-Applikationen --  Java, PHP und JavaScript -- als auch C\#, Ruby und Python, die im Web in Form von ASP.NET, Ruby-on-Rails bzw. Django oder Flask zum Einsatz kommen, werden durch die Auswahl abgedeckt.

\begin{description}
    \item[Cobra vs Mongoose] Diese Implementierung ist in Form eines Ruby-Gems verfügbar und übersetzt \acrshort{xml}-Dokumente in Ruby-Datenstrukturen (Hashes), kann aber auch \acrshort{json}-Daten ausgeben. Bei der Umwandlung setzt der  Konverter auf die sogenannte \emph{Badgerfish}-Konvention~\cite[Abschn.~3]{cobravsmongoose,open311-conventions}.
    \item[GreenCape \acrshort{xml} Converter] Der in PHP implementierte Konverter kann \acrshort{xml}-Daten in asso\-ziative PHP-Arrays umwandeln. Diese können dann zu \acrshort{json}-Werten serialisiert werden.
    \item[Json-lib] Die Java-Bibliothek \texttt{net.sf.json-lib} baut auf Software des \acrshort{json}-Entwicklers Douglas Crockford auf und wird in über 400 anderen Projekten eingesetzt~\cite{mvn-jsonlib}. Das Paket beinhaltet unter anderem auch eine Klasse zum (De-)Serialisieren von \acrshort{xml}-Daten.
    \item[{\acrshort{jsonml}}] Die \acrfull{jsonml} ist ein \acrshort{json}-basiertes Format zur Speicherung von \acrshort{xml}-Markup. Neben der JavaScript-Bibliothek existieren auch Implementierungen in anderen Programmiersprachen, z.B. in Java\footnote{Das \texttt{org.json}-Paket stellt eine entsprechende Klasse bereit.}  oder PHP\footnote{Als Teil des FluentDOM-Projekts, siehe: \url{https://github.com/FluentDOM/FluentDOM/blob/master/src/FluentDOM/Serializer/Json/JsonML.php}}.
    \item[{\acrshort{jxon}}] Bei dem JavaScript-Modul handelt es sich um eine bidirektionale Bibliothek für die \acrlong{jxon} und eine Weiterentwicklung der ursprünglich von Mozilla veröffentlichten Implementierung.
    \item[Newtonsoft Json.NET] Das \acrshort{json}-Framework für C\#.NET wurde insgesamt mehr als 64 Millionen Mal von Paket-Repository NuGet heruntergeladen~\cite{nugetjsonnet} und war zum Testzeitpunkt das am meisten heruntergeladene .NET-Paket~\cite{nugetstatistics}. Es ermöglicht unter anderem auch die Konversion von \acrshort{xml}-Dokumenten in das \acrshort{json}-Format.
    \item[org.json.XML] Das Paket \texttt{org.json} ist die Referenzimplementierung des \acrshort{json}-Formats für Java und bietet auch ein Konversionsverfahren in Form einer dedizierten \acrshort{xml}-Klasse. Es wird von mehr als 1\,600 Projekten eingesetzt, darunter auch die Google Android Library, das Google Web Toolkit (GWT) oder die Objektserialisierungs\hyp{}Bibliothek XStream~\cite{mvn-orgjsonxml}.
    \item[Pesterfish] Das Python-Modul Pesterfish konvertiert zwischen \acrshort{xml} und Python-\linebreak{}Dictionaries, die dann zu \acrshort{json} serialisiert werden können. Laut dem Autor wurde das Konversionsverfahren als \enquote{Reaktion auf die Badgerfish-Konvention} entwickelt\footnote{Der Autor schreibt dies in einem Kommentar im Quelltext des Moduls.}. Für die Verarbeitung der \acrshort{xml}-Daten baut das Modul auf die \texttt{ElementTree}-\acrshort{api} auf und erlaubt Entwicklern auch die Angabe eines eigenen Parsers, beispielsweise \texttt{defusedxml}.
    \item[x2js] Die JavaScript-Bibliothek ist mit 475 Sternen und 219 Forks auf GitHub recht populär~\cite{githubx2js} und erlaubt eine große Zahl von Einstellungsmöglichkeiten.
    \item[x2js (Fork)] Hierbei handelt es sich um einen für die Verwendung mit NodeJS ausgelegten Fork der xj2s-Bibliothek, der seit der Abspaltung im Oktober 2015 unabhängig weiterentwickelt wird.
    \item[xmljson] Das Python-Paket \emph{xmljson} wurde mehr als 11\,500 mal vom Python-Repository PyPI heruntergeladen~\cite{pypistats-xmljson}. Eine Besonderheit ist, dass es die Konversion von \acrshort{xml}-Daten in Python-Dictionaries/Lists bzw. \acrshort{json} anhand von 6 verschiedenen Konventionen (\emph{Abdera}, \emph{Badgerfish}, \emph{Cobra}, \emph{GData}, \emph{Parker} und \emph{Yahoo}) unterstützt.
\end{description}


\section{Methodik}
\label{sec:method}

\subsection{Überprüfung der Konversionqualität}
\label{sec:method-conv}

\begin{figure}[bp!]
    \begin{center}
        \includestandalone[width=0.8\textwidth]{flowchart_conversiontest}
        \caption{Ablauf der Konversionstests}
    \end{center}
\end{figure}

Zur Überprüfung der Konversionsqualität wurden für alle der in Abschnitt~\ref{sec:criteria-lossless} festgelegten Kriterien und diskutierten Probleme ein oder mehrere Testfälle erstellt. Jeder Testfall besteht aus einer \acrshort{xml}-Datei, anhand der bestimmte \acrshort{xml}-Features oder Problemstellungen nachvollzogen werden können. Die Dateien werden dazu zunächst mit einem Konverter in das \acrshort{json}-Format umgewandelt. Die daraus resultierenden \acrshort{json}-Daten werden im Anschluss wieder zurück in das \acrshort{xml}-Format konvertiert, d.h. es wird ein kompletter Round-Trip vollzogen.

Gemäß Definiton~\ref{def:lossless} in Abschnitt~\ref{sec:criteria-lossless} gibt ein Abgleich zwischen den resultierenden \acrshort{xml}-Daten und dem ursprünglichen \acrshort{xml}-Dokument dann Aufschluss über den eventuellen Verlust signifikanter Informationen. Dazu wird überprüft, ob beide Dokumente logisch äquivalent sind, indem beide in \emph{kanonisches \acrshort{xml}} umgewandelt werden. Ob Informationsverlust aufgetreten ist, kann dann mit simplen Vergleich geprüft werden.

Es muss zudem sichergestellt sein, dass bei der Konversion auch tatsächlich valides \acrshort{json} bzw. wohlgeformtes \acrshort{xml} zurückgegeben wird. Da das von Konverter zurückgelieferte Dokument im Zuge der \acrlong{c14n} geparst wird, werden Verstöße gegen die Wohlgeformtheitsanforderungen der \acrshort{xml}-Spezifikation erkannt und führen zum Nichtbestehen des Tests.

Die \acrshort{json}-Zwischendaten werden ebenfalls geparst und auf Verstöße gegen die \acrshort{json}-Spezifikation untersucht. Zudem werden die Daten vor der Rückkonvertierung neu formatiert, sodass nur Informationen weitergegeben werden, die auch tatsächlich signifikant im Sinne der Spezifikation sind, d.h. dass beispielsweise Whitespace außerhalb von Zeichenketten oder die Reihenfolge von Wertpaaren innerhalb eines \acrshort{json}-Objekts verloren gehen.

\begin{figure}[b!]
    \begin{center}
        \includestandalone[width=0.8\textwidth]{flowchart_securitytest}
        \caption{Ablauf der Sicherheitstests}
    \end{center}
\end{figure}

\subsection{Überprüfung der Sicherheit}
\label{sec:method-sec}

Um die Sicherheit der Konversion zu überprüfen, wird eine modifizierte Vorgehensweise genutzt. Die auf Sicherheitslücken in \acrshort{xml}-Parsern abzielenden Test-Dokumente werden ebenfalls zunächst in das \acrshort{json}-Format und im Anschluss wieder zurück in \acrshort{xml} konvertiert. Allerdings ist dabei ausschließlich von Belang, dass keine der in Abschnitt~\ref{sec:xmlattacks} beschriebenen Angriffe ausgelöst wird -- die Korrektheit der Konversion wird hierbei nicht überprüft.

\begin{samepage}
Stattdessen wird ein Sicherheitstest als Fehlschlag gewerten, wenn eine oder mehrere der folgenden Bedingungen zutrifft:
\begin{enumerate}
    \item{} Der Konversionsvorgang überschreitet zuvor festgelege Obergrenzen für die Allozierung von Arbeitsspeicher, verbraucht zuviel CPU-Zeit oder dauert zu lange.\label{itm:cond-dos}
    \item{} Während der Konversion sendet der Konverter Anfragen an einen Webserver.\label{itm:cond-ssrf}
    \item{} Die vom Konverter ausgegebenen \acrshort{json}- oder \acrshort{xml}-Dateien enthalten eine bestimmte Zeichenkette.\label{itm:cond-fsa}
\end{enumerate}
\end{samepage}

Ist der eingesetzte \acrshort{xml}-Parser verwundbar gegenüber Angriffen aus dem Bereich \acrfull{dos}, führt das Parsen der Testdateien zu einem erhöhten Verbrauch an Systemressourcen. Die Obergrenzen für Arbeitsspeicherbelegung, \acrshort{cpu}-Auslastung und Zeitdauer sind so angelegt, dass ein erfolgreicher Angriff zwangsläufig gegen die erste Bedingung verstößt und damit zum Nichtbestehen des Tests führt.

Um zu Prüfen, ob \acrfull{ssrf} möglich ist, wird ein \acrshort{http}-Server gestartet, der von den entsprechenden Testfällen referenziert wird. Ruft ein Konverter während des Parsings eine der URLs auf, wird die Anfrage vom Server aufgezeichnet. Sind bei dem Server bis zum Abschluss der Konversion keine Anfragen eingegangen (Bed.~\ref{itm:cond-ssrf}), ist der Konverter offenbar nicht anfällig für solche Angriffe.

Die Testfälle für die Suche nach Schwachstellen der Kategorie \acrlong{fsa} verweisen auf Dateien, die eine bestimmte Zeichenkette enthalten. Ist diese Zeichenkette in den vom Konverter ausgegebenen \acrshort{json}- oder \acrshort{xml}-Daten enthalten, wurde vom Parser auf das Dateisystem zugegriffen und der Konverter ist verwundbar.

Alle eingesetzen Sicherheits-Testfälle basieren auf den \acrshort{xml}-Testdokumenten aus dem von Christoph Späth betreuten \enquote{DTD-Attacks}-Repository des Lehrstuhls für Netz- und Datensicherheit der Ruhr-Universität Bochum~\cite{dtdattacksrepo}.

\subsection{Technische Umsetzung}

% SLOC calculated via:
%     git ls-files | tr '\n' '\0' | wc -l --files0-from=- | grep -E ".*py$" | awk '{print $1}' | paste -sd+ | bc

Zur einfacheren Durchführung der Testreihen wurde ein Evaluations-Framework auf Basis von Python 3.5 implementiert, das ca. 2200 SLOC umfasst. Der Testprozess wird dabei weitestgehend automatisiert.

Um die \acrlong{c14n} durchzuführen, nutzt das Framework die~\mintinline{python}{write_c14n()}-Methode aus der \emph{lxml}-Bibliothek\footnote{http://lxml.de/}. Die Kanonisierung erfolgt nicht-exklusiv (vgl. Abschn.~\ref{sec:excc14n}) und ohne Kommentare -- ausgenommen ist dabei ein Testfall, der die Unterstützung für Kommentare überprüft.

Für das \acrshort{xml}-Parsing selbst kommt \emph{defusedxml}\footnote{https://pypi.python.org/pypi/defusedxml} zum Einsatz, das die gängigen Python-APIs zum Parsen von \acrshort{xml}-Dokumenten einem Security-Hardening unterzieht und daher nicht anfällig gegenüber fast allen Angriffsvektoren ist~\cite[Abschn. 9.5]{spaeth2016sok}.

Zur Prüfung und Formatierung der \acrshort{json}-Daten wird das Python-Modul \emph{demjson}\footnote{http://deron.meranda.us/python/demjson/} eingesetzt, das über umfangreiche Möglichkeiten zum Finden von Verstößen gegen die \acrshort{json} (einen sogenannten \emph{Linter}) verfügt. Zusätzlich zum \texttt{strict}-Modus wurden einige besondere Einstellungen vorgenommen: Byte-Order-Marks wurden in Übereinstimmung mit der \acrshort{json}-Spezifikation explizit verboten. Null-Bytes und andere von der Spezifikation nicht verbotene, aber möglicherweise Kompatibilitätsprobleme verursachende Zeichen wurden erlaubt, da es sich hierbei um korrektes \acrshort{json} handelt.

Für die Sicherheitstests wird ein eigener Prozess gestartet. Da als Host-System Linux zum Einsatz kommt, wurde die Begrenzung der Systemressourcen mithilfe des \texttt{setrlimit()}-Syscalls umgesetzt. Werden die so festgelegten Obergrenzen für die zur Verfügung stehende CPU-Zeit und Größe des adressierbaren Speichers des Prozesses überschritten, wird der Prozess vom Kernel durch das Signale \texttt{SIGXCPU} bzw. \texttt{SIGKILL} oder \texttt{SIGSEGV} beendet. Das Framework prüft nach Beendigung des Prozesses, ob der Exitcode auf die Terminierung durch ein Signal hinweist oder ob der Prozess normal beendet wurde.
