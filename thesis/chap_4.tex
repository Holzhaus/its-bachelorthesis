\chapter{Ergebnisse} \label{chap:results}

\begin{figure}[b!]
    \includestandalone[width=\textwidth]{resulttable-basic}
    \captionof{table}{Grundlegende Konversions-Testergebnisse.}
    \label{tbl:results-basic}
\end{figure}

\begin{figure}[t!]
    \includestandalone[width=\textwidth]{resulttable-chars}
    \captionof{table}{Ergebnisse der Tests bezüglich Unterstützung der von der \acrshort{xml}-Spezifikation erlaubten Zeichen.}
    \label{tbl:results-chars}
\end{figure}

\begin{figure}[t!]
    \includestandalone[width=\textwidth]{resulttable-complex}
    \captionof{table}{Testergebnisse der Konverter bei komplexen Dokumenten.}
    \label{tbl:results-complex}
\end{figure}

\begin{figure}[t!]
    \includestandalone[width=\textwidth]{resulttable-sec}
    \captionof{table}{Ergebnisse der Tests auf Sicherheitslücken.}
    \label{tbl:results-sec}
\end{figure}

Es wurden insgesamt 123 Testfälle verwendet und 11 verschiedene Konverter überprüft. Allerdings konnte keiner der Konverter alle Anforderungen aus Abschnitt~\ref{sec:criteria} erfüllen.
Im Folgenden werden die Ergebnisse der verschiedenenen Konverter vorgestellt. Ausgabebeispiele zur Veranschaulichung befinden sich in Anhang~\ref{appx:convexamples}.

\section{Cobra vs Mongoose}
\label{sec:cobravsmongoose}

Die Reihenfolge der Elemente sowie Whitespace werden von \emph{Cobra vs Mongoose} verworfen. Das Auftreten von Mixed Content, \glspl{pi} und Kommentaren im Urspungsdokument führt zu Fehlern bei der Rückkonvertierung von \acrshort{json} zu \acrshort{xml}.

Enthält ein \acrshort{xml}-Dokument Zeichen außerhalb des \acrshort{ascii}-Bereichs, führt dies zum Absturz des Programms.

Bei mehreren verschiedenen Default Namespaces innerhalb eines Dokuments werden diese zusammengefasst und lediglich die zuletzt genannte Namespace-Deklaration beibehalten. Zudem geht die Position der Namespace-Prefix-Deklarationen im Dokument verloren.

\emph{Cobra vs Mongoose} setzt den \acrshort{xml}-Parser \emph{REXML} aus der Ruby\hyp{}Standardbibliothek ein und ist für keinen der überprüften Angriffe verwundbar.

\section{GreenCape \acrshort{xml} Converter}
\label{sec:greencapexml}

Bei der Umwandlung von \acrshort{json} zu \acrshort{xml}-Daten fügt der Konverter hinter allen Elementen automatisch Zeilen\-umbrüche und Einrückungen mit einer Breite von vier Leerzeichen ein. Dies führt dazu, dass der Konverter in unmodifizierter Form keinen der Tests besteht. Um die Überprüfung der anderen, davon unabhängigen Aspekte des Konverters gewährleisten zu können, musste dieses Verhalten durch einen Patch entfernt werden (siehe Anhang~\ref{appx:greencapexml}).

Die Konversion der umfangreichen \acrshort{odf}-Spezifikation in Form einer \texttt{*.fodt}-Datei mithilfe des \emph{GreenCape \acrshort{xml} Converters} schlägt fehl. Der entsprechende Versuch wurde abgebrochen, nachdem der PHP-Prozess seit rund 3 Stunden bei voll ausgelasteter \acrshort{cpu} eingefroren war. Eine Analyse mithilfe des Tools \mintinline{shell}{strace} zeigte, dass sich der PHP-Interpreter in einer Endlosschleife aus aufeinanderfolgenden \mintinline{c}{mmap()}- und \mintinline{c}{mummap()}-Syscalls befand (siehe Abb.~\ref{fig:greencapeloop}).

\begin{figure}[bp!]
    \inputminted{shell-session}{greencapexml-strace.txt}
    \captionof{figure}{Eine Endlosschleife im \emph{GreenCape \acrshort{xml} Converter} musste mittels \texttt{SIGTERM} unterbrochen werden.}
    \label{fig:greencapeloop}
\end{figure}

Abgesehen von diesem möglicherweise für \acrshort{dos}-Attacken ausnutzbaren Verhalten ist der \emph{GreenCape \acrshort{xml} Converter} für keinen der getesteten Angriffe verwundbar.

CDATA-Sektionen in \acrshort{xml}-Dokumenten führen zu Fehlern im von der PHP-Bibliothek selbst implementierten \acrshort{xml}-Parser und bringen das Programm zum Absturz.

Bei der Konversion von Mixed Content wird sämtlicher Textinhalt außerhalb der Kind\-elemente verworfen. Whitespace geht ebenso verloren. \glspl{pi} werden ignoriert. Abgesehen von Whitespace bleiben alle Sonderzeichen bei der Umwandlung jedoch erhalten.

Der \emph{GreenCape \acrshort{xml} Converter} kann -- nach Deaktivierung der automatischen Ein-\linebreak{}rückung -- als einziges Programm neben \acrshort{jsonml} (vgl. Abschn.~\ref{sec:jsonml}) die \acrshort{xml}-Antwort auf eine Anfrage an die MusicBrainz-\acrshort{api} verlustlos umwandeln.

\section{Json-lib}
\label{sec:jsonlib}

Bei Kommentaren oder \glspl{pi} innerhalb des Wurzelelements eines \acrshort{xml}-Dokuments stürzt der Konverter ab. \glspl{pi} außerhalb des Wurzelelements werden von \emph{Json-lib} ignoriert. Bei der Konversion geht zudem der Tag-Name des Wurzelelements verloren. Tritt Mixed Content auf, werden bei der Rückkonversion alle Text-Knoten zusammengefasst.

Druckbare \acrshort{ascii}-Zeichen sowie Leerzeichen werden korrekt umgewandelt, alle anderen Zeichen werden von \emph{Json-lib} verworfen.

Mehrere aufeinanderfolgende Elemente selben Namens werden von \emph{Json-lib} in ein \acrshort{json}-Array konvertiert. Dabei wird jedoch lediglich der Inhalt der Elemente übernommen, während der Tag-Name verloren geht.

Enthält das Wurzelelement eines \acrshort{xml}-Dokuments lediglich \emph{Character Data}, so werden diese bei einem Round-Trip zum Inhalt eines Kindelements des Wurzelelements -- in diesem Fall fügt \emph{Json-lib} also eine Elementebene hinzu, die im Ursprungsdokument nicht existiert. Eine genauere Analyse der ausgegebene Daten zeigt, das dieses Verhalten auch der Grund für das Scheitern der CDATA-Testfälle ist. CDATA-Sektionen werden von \emph{Json-lib} sehr wohl unterstützt.

Leider ist es nicht möglich, \emph{Json-lib} ohne besondere Konfiguration in einem Prozess mit begrenztem virtuellen Adressraum zu verwenden, da die \acrfull{jvm} in diesem Fall nicht gestartet werden kann~\cite{jvmmemlimit}. Um dieses Problem zu umgehen, muss beim Start des Java-Prozesses die maximale Größe des Heap-Adressraums sowie die Größe des für Zeiger auf Metadaten von Java-Klassen zur Verfügung stehenden Adressraums angegeben werden.

Eine sinnvolle Aussage über die Anfälligkeit gegenüber Angriffen aus dem Bereich \acrlong{dos} ist unter diesen Umständen jedoch nicht möglich, sodass die entsprechenden Tests für \emph{Json-lib} manuell wiederholt werden mussten.

Dabei ergab sich eine Verwundbarkeit gegenüber \emph{Billion-Laughs}-Angriffen sowohl mit General Entities als auch mit Parameter Entities. \emph{Json-lib} ist ebenfalls anfällig für \emph{Quadratic\hyp{}Blowup}\hyp{}Attacken. Zudem ist der Parser anfällig für \acrshort{xxe}-Angriffe mittels General Entites bzw. Parameter Entities, die sowohl für \acrlong{fsa} als auch für \acrlong{ssrf} genutzt werden können.
Ebenso erlaubt \emph{Jsonlib} die Einbettung lokaler Dateien mittels Parameter Entities in DTDs, sowohl in der ursprünglichen Version des Angriffs~\cite[S.~10]{morgan2014xml}, als auch in einer modifizierten Variante, die 2016 von Sicherheitsforschern der Ruhr-Universität Bochum vorgestellt wurde.~\cite[Abschn.~5.2]{spaeth2016sok}.

Ein Blick in den Quellcode der Klasse \texttt{net.sf.json.XMLSerializer} zeigt, dass Json-lib den \acrshort{xml}-Parser XOM einsetzt.

\section{\acrshort{jsonml}}
\label{sec:jsonml}

Das Konversionverfahren \acrshort{jsonml} ist vergleichsweise vollständig. Lediglich die in den Ursprungsdokumenten enthaltenen \glspl{pi} werden vom Konverter ignoriert und nicht in die \acrshort{json}-Ausgabe übernommen.

\acrshort{jsonml} ist von allen getesten Konvertern als einziger in der Lage, alle Test-Dokumente verlustlos zu konvertieren, solange diese keine \glspl{pi} enhielten.

\section{Json.NET}
\label{sec:jsondotnet}

Der \emph{Json.NET}-Konverter nutzt die Klasse \mintinline{csharp}{System.Xml.XmlDocument} aus der C\#\hyp{}Standardbibliothek für die Verarbeitung von \acrshort{xml}-Daten.

Probleme hat der Konverter mit der Beibehaltung der ursprünglichen Elementreihenfolge des Dokuments. Bei der Umwandlung von Mixed Content tritt ebenfalls Informationsverlust auf. Whitespace wird bei der Konversion verworfen.

\emph{Json.NET} ist als einziger Konverter im Test in der Lage, \glspl{pi} korrekt umzuwandeln -- zumindest solange sie keinen Whitespace enthalten, da dieser verloren geht. Treten \glspl{pi} außerhalb des Wurzelelements auf, führt dies bei der Rückkonvertierung der \acrshort{json}-Daten zu einem Absturz des Programms.

Im Gegensatz zu allen anderen überprüften Konvertern wandelt \emph{Json.NET} auch \acrshort{xml}-Kommentare um. Der Konverter fügt diese als JavaScript-Blockkommentare, d.h. in der Form \enquote{\mintinline{javascript}{/* Comment */}}, in die \acrshort{json}-Ausgabe ein. Da Kommentare jedoch nicht Teil der \acrshort{json}-Syntax sind (vgl. Def.~\ref{def:json} in Abschn.~\ref{sec:jsontypes}), ist die von \emph{Json.NET} in diesem Fall produzierte Ausgabe kein gültiges \acrshort{json}~\cite{ecma404,rfc7159} und führt bei den meisten Parsern zu Problemen~\cite[Abschn.~2.1,~4.1]{seriot2016minefield}.

Alle Zeichen, die nicht mittels 7-Bit-\acrshort{ascii}-Kodierung darstellbar sind (d.\,h. alle Zeichen, deren Unicode-Codepoint hinter U+00007F liegt) werden von \emph{Json.NET} bereits bei der Konversion in \acrshort{json} durch Fragezeichen ersetzt, deren Anzahl sich an der Anzahl der Bytes bemisst, die für die Kodierung des Zeichens in \acrshort{utf8} nötig sind.

Der Konverter ist anfällig für \acrshort{ssrf}-Angriffe mittels \acrshort{xxe}, sowohl mit General Entities als auch mit Parameter Entities. Ebenso ruft der Konverter externe \glspl{dtd} über den im \texttt{SYSTEM}-Identifiktator angegebene \acrshort{uri} ab.

\section{JXON}
\label{sec:jxon}

\emph{JXON} fehlt die Unterstützung von Mixed Content. Zudem geht die Reihenfolge der Elemente verloren. \acrlongpl{pi} werden bei der Konversion ignoriert.

Auch Whitespace geht bei der Konversion verloren -- neben Tab, Zeilenvorschub, Wagenrücklauf und Leerzeichen umfasst das jedoch auch Unicode-Whitespace wie Ogham-Leerzeichen, mongolische Volkaltrennzeichen oder ideographische Leerzeichen. Eine komplette Liste ist in Anhang~\ref{appx:unicode-whitespace}.

Eine Verwundbarkeit für die getesteten Angriffe auf \acrshort{xml}-Parser wurde bei der Überprüfung nicht gefunden.

\section{org.json.XML}
\label{sec:orgjsonxml}

Da es sich bei \texttt{org.json.XML} um ein Java-Package handelt, ist die Untersuchung von \acrshort{dos}-Angriffen wegen der Beschränkungen der \acrlong{jvm} ebenso problematisch wie bei \emph{Json-lib}~(vgl. Abschn.~\ref{sec:jsonlib}). Auch bei org.json.\acrshort{xml} musste die Verwundbarkeit gegenüber solchen Angriffen daher manuell verifiziert werden. Der \acrshort{xml}-Parser wird von dem Paket selbst implementiert und ist für keinen der überprüften Angriffe verwundbar.

Attribute gehen bei der Konvertierung von \acrshort{xml}-Daten in \acrshort{json} zwar nicht verloren, bei der Rückkonvertierung kann der Konverter jedoch nicht mehr erkennen, dass es sich um Attribute handelt und interpretiert diese stattdessen als Elemente.

\acrshort{xml}-Namespace-Prefixes für Tag-Namen werden zwar grundsätzlich unterstützt, Namespaces können aber aufgrund der fehlerhaften Behandlung von Attributen dennoch nicht genutzt werden.
 Mixed Content wird nicht unterstützt -- enthält ein Element neben einem Kindelement auch Text, wird dieser verworfen.

Ebenfalls gehen bei der Konversion \glspl{pi} und die Ordung der Dokumente verloren. Die Elementreihenfolge ist dabei anscheinend zufällig: Bei der Rückkonvertierung werden Elemente weder in alphabetischer noch in der Reihenfolge ausgegeben, in der sie in der \acrshort{json}-Datei angeordnet sind.

Der Konverter unterstützt lediglich die Konversion von druckbaren \acrshort{ascii}-Zeichen. Unicode-Zeichen außerhalb dieses Bereichs werden bei der Umwandlung in das \acrshort{json}-Format in eine der Byteanzahl ihrer \acrshort{utf8}-Kodierung entsprechnde Menge von Fragezeichen umgewandelt.

Zudem werden Zahlenwerte sowie die Zeichenketten \enquote{\mintinline{json}{true}} und \enquote{\mintinline{json}{false}} in die nativen Java-Datentypen konvertiert, wobei Informationsverlust auftreten kann. Die Zeichenkette \enquote{\texttt{1e-324}} wird beispielweise bei der Übersetzung zu \acrshort{json} als Zahl interpretiert und erscheint daher in der Ausgabe gerundet als \enquote{\texttt{0}}.

\section{Pesterfish}
\label{sec:pesterfish}

Bei der Konversion gehen die Namen der Namespace-Prefixe verloren und werden durch generische Bezeichnungen (\texttt{ns0}, \texttt{ns1}, \dots{}) ersetzt. Diese werden auch dann verwendet, wenn im Ursprungsdokument keinerlei Namespace-Prefixes verwendet wurden, sondern ein eigener Default-Namespace genutzt wurde. Grund dafür ist die\linebreak{}\texttt{ElementTree}-\acrshort{api}, die Namespace-Prefixes beim Parsing von \acrshort{xml}-Dokumenten automatisch zur vollen URI expandiert und den ursprünglichen Prefixnamen verwirft~\cite[Abschn.~20.5.1.7]{pythonetreexmlns}.
Zudem gehen bei der Konvertierung auch \glspl{pi} verloren.

Dennoch ist es mit Pesterfish-Konverter möglich, komplette \acrshort{rss}-Dateien verlustlos in das \acrshort{json}-Format und wieder zurück zu konvertieren.

Bei den Sicherheitstests zeigte sich, dass \emph{Pesterfish} bei Nutzung der Vorgabeeinstellungen verwundbar für die \acrshort{dos}-Angriffe \emph{Billion Laughs} und \emph{Quadratic Blowup} ist. Da die Bibliothek jedoch die Verwendung einer eigenen \texttt{ElementTree}-Implementierung erlaubt, kann diesem Problem durch den Einsatz eines sicheren Ersatzes -- beispielsweise aus der \emph{defusedxml}-Bibliothek -- entgegengewirkt werden.

\section{x2js}
\label{sec:x2js}

Bei der Umwandlung mittels \emph{x2js} geht die Reihenfolge der Elemente verloren. \glspl{pi} werden komplett ignoriert.

Verfügt ein Element über mehrere Kindelemente mit dem gleichen Tag-Namen, tritt bei der Konversion ein Fehler auf, wodurch mehrere \acrshort{json}-Arrays ineinander geschachtelt werden. Bei der Rückkonversion enstehen dann zusätzliche Kindelemente, bei denen die Array-Indizes als Tag-Name verwendet werden.

Mixed Content -- auch durch Einrückungen verursachter -- führt zum Absturz von \emph{x2js}. Sonderzeichen stellen für den Konverter jedoch kein Problem dar.

Bei komplexeren Dokumenten entstehen durch die Konversion mittels \emph{x2js} extrem tief verschachtelte \acrshort{json}-Strukturen mit mehr als 2\,200 Ebenen (vgl. Abb.~\ref{fig:jsondepth} in Abschn.~\ref{sec:jsonml-syntax}). Dies kann bei Parsern zu Problemen und Abstürzen führen, was auch bei dem zur Reformatierung der \acrshort{json}-Daten eingesetzten Parser der Fall war. Die entsprechenden Testfälle mussten daher manuell wiederholt werden.

\section{x2js (Fork)}
\label{sec:x2js-fork}

Die Beibehaltung der Dokumentreihenfolge ist beim Einsatz des \emph{x2js}-Forks nicht gegeben.

Mixed Content führt im Gegensatz zum Ursprungsprojekt nicht zum Absturz, jedoch wird dabei der gesamte Character Content eines Elements zu einem einzelnen Textknoten zusammengefasst.

Whitespace bleibt zwar grundsätzlich erhalten, Einrückungen werden jedoch wie anderer Mixed Content auch zusammengefasst, sodass bei Dokumenten mit Einrückungen nach der Konversion jedes Element mit zuvor eingerücktem Inhalt stattdessen einen einzelnen, nur aus Whitespace bestehenden Textknoten enthält.

Auch \glspl{pi} werden nicht unterstützt -- treten diese innerhalb des Wurzelelements auf, wird anstelle der \gls{pi} ein leeres Element namens \enquote{undefined} eingefügt.

Der Bug, der bei \emph{x2js} Probleme mit Elementen verursacht, die mehrere Kindelemente mit dem gleichen Tag-Namen enhalten, ist in der geforkten Version behoben. Die Kindelemente werden dabei jedoch nach Namen gruppiert, sodass die Reihenfolge beispielsweise bei alternierenden Tag-Namen verloren geht.

\section{xmljson}
\label{sec:xmljson}

Im Gegensatz zu anderen Konvertern verfügt das Python-Package \emph{xmljson} über keine Möglichkeit, einen \acrshort{xml}-Daten enthaltenden String direkt zu konvertieren, sondern akzeptiert lediglich bereits geparste \texttt{ElementTree}-Objekte. Da der Nutzer selbst für das Parsen des \acrshort{xml}-Dokuments zuständig ist und es im Unterschied zu Pesterfish (Vgl.~Abschn.~\ref{sec:pesterfish}) auch keine Voreinstellung gibt, wird eine Sicherheitsanalyse dieses Konverters hinfällig. Zum Durchführen der Tests wurde \texttt{defusedxml.lxml} verwendet.

Der \emph{xmljson}-Konverter kann mehrere verschiedene Konvertierungskonventionen nutzen (vgl.~Abschn.~\ref{sec:converters}), die unabhängig voneinander getestet wurden.

Durch den Einsatz der \texttt{ElementTree}-\acrshort{api} hat \emph{xmljson} keinen Zugriff auf die im \acrshort{xml}-Dokument verwendeten Prefixnamen, sodass diese wie beim \emph{Pesterfish}-Konverter durch generische Namen ersetzt werden~(vgl.~Abschn.~\ref{sec:pesterfish}).
\glspl{pi} werden von allen Konventionen ignoriert.

Lediglich die \emph{Cobra}- und \emph{Yahoo}-Konventionen (Abschn.~\ref{sec:xmljson-cobra} und~\ref{sec:xmljson-yahoo}) nutzen keine Typinferenz. Alle anderen Konverter wandeln die Zeichenketten \enquote{\texttt{true}} und \enquote{\texttt{false}} in boolsche Werte um. Auch Zahlenwerte werden als numerische Datentypen interpretiert, wodurch beispielsweise Rundungsfehler auftreten können oder Formatierung verloren geht (z.B. \mintinline{json}{1e+39} anstatt \mintinline{json}{"1E39"}). Sehr große Zahlen werden als \mintinline{js}{Infinity}-Literal dargestellt, der zwar valides JavaScript wäre, von der \acrshort{json}-Spezifikation jedoch nicht erlaubt ist.

Ähnlich wie auch \emph{JXON} (vgl.~\ref{sec:jxon}) entfernt die \emph{Badgerfish}- bzw. \emph{GData}-Konvention bei der Konversion Unicode-Whitespace-Zeichen (vgl. Anhang~\ref{appx:unicode-whitespace}). Zudem werden bei den Konventionen \emph{Badgerfish}, \emph{GData} und \emph{Parker} einige dezimale Zahlzeichen aus der BMP-Ebene des Unicode-Standards~\cite[S.~49]{unicode9} in ihr ASCII-Äquivalenta umgewandelt -- aus einer bengalischen Ziffer Acht\footnote{Unicode-Codepoint \texttt{U+09EE}: \texttt{BENGALI DIGIT EIGHT}} wird so beispielsweise die lateinische \enquote{\texttt{8}}\footnote{Unicode-Codepoint \texttt{U+0038}: \texttt{DIGIT EIGHT}}. Die vollständige Liste der veränderten Zahlzeichen befindet sich in Anhang~\ref{appx:unicode-digits}.

\subsection{Abdera}
\label{sec:xmljson-abdera}

Attribute werden bei der \acrshort{json}-Konvertierung mithilfe der \emph{Abdera}-Konvention zwar in die Ausgabe übernommen, allerdings ist \emph{xmljson} bei der Rückkonvertierung nicht in der Lage, diese von normalen Elementen zu unterscheiden. Daher befinden sich Attribute nach der Rückkonvertierung nicht mehr an der ursprünglichen Stelle im Dokument, sondern in einem \enquote{\mintinline{xml}{<attributes>}}-Element, das als Kindelement des Ursprungselements eingefügt wird. Ein zusätzliches \enquote{\mintinline{xml}{<children>}}-Element enthält die ursprünglichen Kindelemente des Elements.

Außerdem werden teilweise Elemente und Textknoten zu Attributwerten umgedeutet. So wird aus \enquote{\mintinline{xml}{<a><b>hello</b></a>}} durch einen Round-Trip \enquote{\mintinline{xml}{<a b="hello" />}}

Da der Konverter nicht in der Lage ist, anhand der \acrshort{json}-Daten verlässlich zwischen Elementen und Attributen zu unterscheiden und diese dadurch später nicht mehr korrekt rekonstruieren kann, schlagen auch Tests fehl, die der Konverter eigentlich bestehen könnte. So nutzt der Konverter bei mehreren Kindelementen \acrshort{json}-Arrays, d.h. die Elementreihenfolge ist auch nach der Konversion noch ersichtlich.
Auch Tag-Name und Attribute des Wurzelelements gehen eigentlich nicht verloren.

Mixed Content wird nicht unterstützt, es ist lediglich der erste Textknoten im Dokument auffindbar.
Führender oder anhängender Whitespace wird bei der Konversion verworfen.

\subsection{Badgerfish}
\label{sec:xmljson-badgerfish}

Die Ergebnisse stimmen im Wesentlichen mit denen des \emph{Cobra-vs-Mongoose}-Konverters (vgl.~Abschn.~\ref{sec:cobravsmongoose}) überein, der ebenfalls auf die sogenannte Badgerfish-Konvention zur Darstellung von \acrshort{xml}-Strukturen in \acrshort{json} setzt. Nicht unterstützte Features wie Mixed Content, \glspl{pi} oder Kommentare führten bei \emph{xmljson} jedoch nicht zu einem Absturz des Programms. Zudem kommt es zu Problemen durch den Einsatz von Typinferenz und den Verlust der Prefixnamen von Namespaces.

\subsection{Cobra}
\label{sec:xmljson-cobra}

Bei \emph{Cobra} handelt es sich um eine modifizierte Variante der \emph{Abdera}-Konvention, die sich hauptsächlich in der Frage, welche Schlüssel in der \acrshort{json}-Objekt-Repräsentation eines \acrshort{xml}-Dokuments optional sind, von \emph{Abdera} unterscheidet. Daher hat auch \emph{Cobra} große Probleme mit der Rückkonvertierung zu \acrshort{xml} und der Unterscheidung zwischen Elementen und Attributen.

Ein weiterer Unterschied ist, dass bei Cobra keine Datentypen erraten werden, sondern alles als String behandelt wird.

\subsection{GData}
\label{sec:xmljson-gdata}

Durch die Nutzung von ungeordneten \acrshort{json}-Objekten geht bei dem Einsatz der \emph{GData}-Konvention die Reihenfolge der im \acrshort{xml}-Dokument enthaltenen Elemente verloren. Bei Mixed Content wird lediglich der erste Textknoten übernommen, alle weiteren werden verworfen.

Bei der Konversion kann es zu Informationsverlust durch Typinferenz kommen.

\subsection{Parker}
\label{sec:xmljson-parker}

In der Standardeinstellung verwirft diese Konvention das Wurzelelement -- dieses Verhalten ist jedoch über einen Parameter abschaltbar.

Alle Attribute gehen bei der Konversion verloren, ebenso wie \glspl{pi}. Als einzige Konvention des \emph{xmljson}-Konverters wurde bei ausschließlich Text enhaltenden Elementen führender oder nachfolgender Whitespace nicht verworfen -- Mixed Content wird jedoch trotzdem nicht unterstützt.

\subsection{Yahoo}
\label{sec:xmljson-yahoo}

Die \emph{Yahoo} hat ebenfalls Probleme bei der Unterscheidung zwischen Elementen und Attributen, fügt aber im Gegensatz zu \emph{Abdera} und \emph{Cobra} nicht neue, im Originaldokument nicht existierende Elemente in das Dokument ein.

Die Elementreihenfolge geht bei der Konvertierung verloren.

Im \acrshort{xml}-Dokument enthaltener Character Content wird in allen getesteten Fällen zu \acrshort{json}-Strings konvertiert, eine Typumwandlung findet nicht statt.

\chapter{Weiterentwicklung eines Konversionsverfahrens}
\label{chap:jsonml}

Mit insgesamt 112 von 123 bestandenen Testfällen erfüllt die \acrfull{jsonml} von allen Konversionverfahren die meisten Kriterien im Test.

In diesem Kapitel wird daher zunächst ein Überblick über die von \acrshort{jsonml} eingesetzte Syntax gegeben.
Darauf aufbauend werden die notwendigen Modifikationen am \acrshort{jsonml}-Verfahren beschrieben, die im Zuge dieser Arbeit entwickelt wurden, um das Ziel eines vollständig verlustlosen Konversionsverfahrens zu erreichen.

\section{Syntax}
\label{sec:jsonml-syntax}

Im Gegensatz zu anderen Konvertern nutzt \acrshort{jsonml} ungeordnete \acrshort{json}-Objekte ausschließlich für Attributlisten~\cite{jsonmlsyntax}. Die Baumstruktur eines \acrshort{xml}-Dokuments wird mittels \acrshort{json}-Arrays dargestellt, wobei ein Array immer genau ein Element repräsentiert. Textknoten bzw. CDATA-Sektionen werden zu einfachen \acrshort{json}-Strings umgewandelt.

\begin{figure}[H]
    \begin{definition}[{Formale Syntax der \acrfull{jsonml}}]
        \label{def:jsonml}
        Sowohl \synt{tag-name} als auch \synt{attribute-name} sind \acrshort{json}-Werte vom Typ String. Die Whitespace-Regeln sind identisch mit denen von \acrshort{json} (vgl. Def.~\ref{def:json}).
        \begin{grammar}
            <element> ::= \[[
        \begin{stack}
            `[' <tag-name>
                \begin{stack}
                    `,' <attributes>\\
                \end{stack}
                \begin{stack}
                    `,' \begin{rep}
                            <element>\\
                            `,'
                        \end{rep}\\
                \end{stack}
            `]'\\
            \tok{\color{black} \acrshort{json}-String}
        \end{stack}
    \]]

<attributes> ::= \[[
    `\{' \begin{stack}
            \begin{rep}
                <attribute-name> `:' <attribute-value>\\
                `,'
            \end{rep}\\
        \end{stack} `\}'
    \]]

<attribute-value> ::= \[[
        \begin{stack}
            \tok{\color{black} \acrshort{json}-String}\\
            \tok{\color{black} \acrshort{json}-Number}\\
            `true'\\
            `false'\\
            `null'
        \end{stack}
    \]]

        \end{grammar}
    \end{definition}
\end{figure}

\begin{figure}[h]
    \begin{example}[{\acrshort{jsonml}-Dokument}]
        In der \acrshort{jsonml}-Repräsentation des \acrshort{xml}-Dokuments aus Beispiel~\ref{ex:xmldoc} fehlt lediglich die \acrlong{pi}.
        \inputminted{json}{xmltree.json}
        \label{fig:xmltreejsonml}
    \end{example}
\end{figure}

\acrshort{jsonml} sieht keine gesonderte Verarbeitung von Namespace-Deklarationen oder mit Namespace-Prefixes versehenen Tag-Namen vor, sondern behandelt diese wie normale Attribute bzw. wie einen Teil des Tag-Namens.

\acrshort{jsonml} stellt die \acrshort{xml}-Inhalte recht effizient dar: Die \acrshort{json}-Repräsentation eines umfangreichen Office-Dokuments im \acrshort{fodt}-Format benötigt rund $6,6$ Prozent weniger Speicherplatz als die äquivalente Darstellung durch kanonisches \acrshort{xml}.

Im Vergleich zu anderen Konvertern ist der Overhead bei \acrshort{jsonml} deutlich geringer. So waren die vom Pesterfish-Konverter ausgegebenen Daten trotz Informationsverlust auch nach Entfernung von optionalem Whitespace und unnötiger Quotierung von Unicode-Zeichen mehr als dreieinhalb Mal so groß wie bei \acrshort{jsonml} (vgl. Abb.~\ref{fig:sizecomparison}).

\begin{center}
    \begin{threeparttable}
        \caption{Größenvergleich von \acrshort{jsonml} ggü. \acrshort{xml} und Pesterfish anhand der Spezifikation des \acrlong{odf} im \acrshort{fodt}-Format.}
        \label{fig:sizecomparison}
        \begin{tabular}{lrrr}
            \toprule
            \rowcolor{white}     & \multicolumn{1}{c}{\fontfamily{rubflama}\selectfont\textbf{Größe (in bytes)}} & \multicolumn{2}{c}{\fontfamily{rubflama}\selectfont\textbf{Verhältnis zu \acrshort{xml} (in \%)}}\\
                                 &                  & \multicolumn{1}{c}{\fontfamily{rubflama}\selectfont\textbf{Größe}}   & \multicolumn{1}{c}{\fontfamily{rubflama}\selectfont\textbf{Veränderung}}\\
            \midrule
            \rowcolor{rubgray!50} Kanonisches \acrshort{xml} &  $5787196$ & $100,0$ &      $0$ \\
                                  \acrshort{jsonml}\tnote{a} &  $5405329$ &  $93,4$ &   $-6,6$ \\
            \rowcolor{rubgray!50} Pesterfish\tnote{a}        & $15061634$ & $260,3$ & $+160,3$ \\
                                  Pesterfish\tnote{b}        & $14480612$ & $250,2$ & $+150,2$ \\
            \bottomrule
        \end{tabular}
        \begin{tablenotes}
            \item[a] \acrshort{json} unverändert
            \item[b] Optionaler \acrshort{json}-Whitespace entfernt
        \end{tablenotes}
    \end{threeparttable}
\end{center}

\begin{figure}[t]
    \includestandalone[width=\textwidth]{nestingdepth}
    \captionof{figure}{Verschachtelungstiefe der aus einem Word-\acrshort{xml}-Dokument erzeugten \acrshort{json}-Dateien (weniger ist besser).}
    \label{fig:jsondepth}
\end{figure}

Neben der Dateigröße ist auch die Verschachtelungstiefe der \acrshort{json}-Container ein guter Indikator für Overhead. Meist wird für die Verarbeitung der verschiedenen Ebenen von \acrshort{json}-Objekten und -Arrays ein Stapelspeicher (Stack) eingesetzt, der bei jeder zusätzlichen Verschachtelungsebene anwächst. Damit der Speicherverbrauch des Stacks im Rahmen bleibt, erlaubt es die \acrshort{ietf} daher die Verschachtelungstiefe von \acrshort{json} zu begrenzen ~\cite[Abschn.~9]{rfc7159}. So hat die \mintinline{php}{json_decode()}-Funktion von PHP standardmäßig eine Obergrenze von 512 Ebenen~\cite{php-json} und Ruby limitiert die maximale Tiefe sogar auf nur 100 Ebenen~\cite{ruby-json}.

Bei einem Vergleich der Verschachtelungstiefe anhand eines Microsoft\hyp{}Office\hyp{}Dokuments kann \acrshort{jsonml} ebenfalls gut abschneiden (vgl. Abb.~\ref{fig:jsondepth}). Mit 17 Ebenen ist die Komplexität von \acrshort{jsonml} am geringsten und unterschreitet damit den Medianwert um rund 26 Prozent.

\section{Unterstützung von \acrlongpl{pi}}

Probleme hat der Konverter jedoch mit der Umwandlung von \acrfullpl{pi}. Diese werden bei der Umwandlung in \acrshort{json} vollständig ignoriert. Stephen McKamey, der Entwickler von \acrshort{jsonml}, begründet dies damit, das es keine sinnvolle Entsprechung von \acrshortpl{pi} in \acrshort{json} gäbe~\cite{mckamey2006xml}.

Zwar bietet \acrshort{json} tatsächlich keinen vergleichbaren Mechanismus, eine Unterstützung von \glspl{pi} kann für bestimmte Einsatzzwecke aber sinnvoll sein, beispielsweise wenn sonst die Verknüpfung mit \acrshort{xml}-Stylesheets oder Formatierungsinformationen in DocBook\hyp{}Dateien verloren gehen könnte. Im Rahmen der vorliegenden Arbeit wurde die \acrshort{jsonml}-Syntax daher um Unterstützung von \acrlongpl{pi} ergänzt.

\glspl{pi} bestehen aus einem \emph{Ziel} und \emph{Daten} (Vgl. Abschn. \ref{sec:xmlbasics}), bilden also das 2-Tupel\linebreak{}$P \coloneqq \langle target, data \rangle$. Der Datenteil kann dabei auch leer sein.

Das Ziel muss ein gültiger Name im Sinne der \acrshort{xml}-Spezifikation sein~\cite[{Regel~[17]}]{xml}. Das heißt, dass der Name einer \gls{pi}\ ebenso wie auch der Tag-Name von Elementen~\cite[{Regel~[40]}]{xml} mit einem sog. \texttt{NameStartChar} beginnen muss. Dadurch wird ausgeschlossen, dass Tag-Namen mit bestimmten Zeichen beginnen -- darunter auch das Fragezeichen, da dies dazu führen würde, dass Start-Tags mit \glspl{pi} verwechselt werden könnten. Insbesondere in \acrshort{sgml} -- zu dem \acrshort{xml} vollständig kompatibel sein soll -- wären solche Tags nicht mehr von \glspl{pi} zu unterscheiden, da laut \acrshort{sgml}-Spezifikation lediglich ein einfaches Größerzeichen anstatt der Kombination aus Fragezeichen und Größerzeichen (\enquote{\texttt{?>}}) zum Schließen der \gls{pi} ausreicht.

Dadurch wird es möglich, \glspl{pi} in \acrshort{jsonml} eindeutig in Form eines \acrshort{json}-Arrays\linebreak{}\enquote{\mintinline{json}{["?target", "data"]}} darzustellen~(vgl. Definition \ref{def:jsonmlpi}), das dem 2-Tupel $P$ (s.o.) entspricht. Die Repräsentation von \glspl{pi} ähnelt damit der eines Elementknotens, der einen einzelnen Textknoten (\emph{Character Data}) enthält. Eine Verwechslung ist jedoch durch das dem Zielnamen vorangestellte Fragezeichen ausgeschlossen -- ein Tag-Name darf nicht mit einem Fragezeichen beginnen, wodurch die Kategorisierung als \gls{pi} eindeutig ist.  \begin{figure}[h]
    \begin{definition}[{Formale Syntax der \acrfull{jsonml} mit \emph{\glspl{pi}}}]
        \label{def:jsonmlpi}

        Die um Unterstützung von \emph{\glspl{pi}} erweiterte Syntax ist mit Ausnahme der Produktionsregeln für \synt{element} identisch mit der Syntax aus Definition~\ref{def:jsonml}.
        \synt{tag-name}, \synt{pi-target} und \synt{pi-data} sind \acrshort{json}-Werte vom Typ String.

        \begin{grammar}
            <element> ::= \[[
        \begin{stack}
            `[' \begin{stack}
                    <tag-name>
                    \begin{stack}
                        `,' <attributes>\\
                    \end{stack}
                    \begin{stack}
                        `,' \begin{rep}
                                <element>\\
                                `,'
                            \end{rep}\\
                    \end{stack}\\
                    `?' <pi-target> `,' <pi-value>
                \end{stack} `]'\\
            \tok{\color{black} \acrshort{json}-String}
        \end{stack}
    \]]

        \end{grammar}

        Enthält das Dokument \emph{\glspl{pi}} auf Dokument-Ebene (d.h. als Top-Level-Konstrukt), dann ist das \acrshort{jsonml}-Wurzelelement ein \synt{element} mit einem leeren String als \synt{tag-name}, das die Kindknoten des Dokuments (d.h. \emph{\glspl{pi}} auf Dokumentebene und das Wurzelelement des Dokuments) als Unterelemente enthält.
    \end{definition}
\end{figure}

\section{Überprüfung der Änderungen}

Die syntaktischen Änderungen aus Definition~\ref{def:jsonmlpi} wurden in die JavaScript-Referenz\-implementierung von Stephen McKamey eingearbeitet. Entsprechende \emph{Unittests} zur Sicherstellung der korrekten Umwandlung von \glspl{pi} wurden ebenfalls hinzugefügt.

\begin{figure}[h!]

    \begin{example}[{\acrshort{jsonml}-Dokument mit \glspl{pi}}]
        Die \acrshort{jsonml}-Repräsentation des \acrshort{xml}-Dokuments aus Beispiel \ref{ex:xmltree} kann nun die \gls{pi} darstellen -- auch solche, die sich außerhalb des Wurzelelements befinden.
        \begin{minted}[autogobble]{json}
            ["", "\n",
                [ "?xml-stylesheet", "href=\"style.css\"" ],"\n","\n",
                ["albums", "\n  ",
                    ["album", {"catno": "ARGO LP-628"}, "\n    ",
                        ["artist", "Ahmad Jamal Trio"], "\n    ",
                        ["title", "At The Pershing"], "\n    ",
                        ["recording", "Recorded ",
                            ["date", "January 16, 1958"], "."
                        ], "\n  "
                    ], "\n"
                ]
            ]
        \end{minted}
    \end{example}
\end{figure}

Bei einer erneuten Überprüfung des \acrshort{jsonml}-Konverters unter Berücksichtigung der o. g. Änderungen wurde deren Korrektheit bestätigt: Alle Testdokumente, auch die zuvor fehlgeschlagenen, lassen sich nun verlustlos von \acrshort{xml} nach \acrshort{json} und wieder zurück konvertieren (vgl. Tab.~\ref{tbl:results-basic} bis~\ref{tbl:results-sec}).

Alle Änderungen wurden dem \acrshort{jsonml}-Projekt zur Verfügung\footnote{Vgl.~\url{https://github.com/mckamey/jsonml/pull/14}} gestellt (siehe Anhang \ref{appx:jsonmlpi}).
