\chapter{Ergebnisse} \label{chap:results}

In diesem Abschnitt werden die Ergebnisse der verschiedenene Textfälle vorgestellt. Es wurden 101 Testfälle und 11 verschiedene Konverter überprüft.

\begin{figure}[b!]
    \label{fig:results-basic}
    \includestandalone[width=\textwidth]{resulttable-basic}
    \caption{Konversions-Testergebnisse bezüglich verschiedener Konversionprobleme.}
\end{figure}

\begin{figure}[b!]
    \label{fig:results-chars}
    \includestandalone[width=\textwidth]{resulttable-chars}
    \caption{Ergebnisse der Tests bezüglich Unterstützung der von der XML-Spezifikation erlaubten Zeichen.}
\end{figure}

\begin{figure}[b!]
    \label{fig:results-complex}
    \includestandalone[width=\textwidth]{resulttable-complex}
    \caption{Testergebnisse der Konverter bei komplexen Dokumenten.}
\end{figure}

\begin{figure}[b!]
    \label{fig:results-sec}
    \includestandalone[width=\textwidth]{resulttable-sec}
    \caption{Ergebnisse der Tests auf Sicherheitlücken}
\end{figure}


\chapter{\acrfull{jsonml}}

Mit insgesamt XX von 101 bestandenen Testfällen ist die \acrfull{jsonml} durchaus vielversprechend.

\begin{definition}Formale Syntax der \acrfull{jsonml}
\label{def:jsonml}

Sowohl \synt{tag-name} als auch \synt{attribute-name} sind \acrshort{json}-Werte vom Typ String. Die Whitespace-Regeln sind identisch wie bei \acrshort{json} (Vgl. Definition~\ref{def:json}).

\begin{grammar}
    <element> ::= \[[
        \begin{stack}
            `[' <tag-name>
                \begin{stack}
                    `,' <attributes>\\
                \end{stack}
                \begin{stack}
                    `,' \begin{rep}
                            <element>\\
                            `,'
                        \end{rep}\\
                \end{stack}
            `]'\\
            \tok{\color{black} \acrshort{json}-String}
        \end{stack}
    \]]

<attributes> ::= \[[
    `\{' \begin{stack}
            \begin{rep}
                <attribute-name> `:' <attribute-value>\\
                `,'
            \end{rep}\\
        \end{stack} `\}'
    \]]

<attribute-value> ::= \[[
        \begin{stack}
            \tok{\color{black} \acrshort{json}-String}\\
            \tok{\color{black} \acrshort{json}-Number}\\
            `true'\\
            `false'\\
            `null'
        \end{stack}
    \]]

\end{grammar}
\end{definition}

\section{Unterstützung von Processing Instructions}

Probleme hatte der Konverter jedoch mit der Umwandlung von \acrfullpl{pi}. Diese werdem in der aktuellen Version der einfach ignoriert. Stephen McKamey, der Entwickler von \acrshort{jsonml} begründet damit, das es keine sinnvolle Entsprechung von \acrshortpl{pi} in \acrshort{json} gäbe~\cite{mckamey2006xml}.

Zwar ist bietet JSON tatsächlich keinen vergleichbaren Mechanismus, eine Unterstützung von \glspl{pi} für bestimmte Einsatzzwecke aber sinnvoll sein, da sonst beispielsweise Formatierungsinformationen in DocBook-Dokumente verloren gehen könnten. Daher wurde die \acrshort{jsonml}-Syntax um \acrshort{pi}-Unterstützung ergänzt.

Processing Instructions bestehen aus einem \emph{Ziel} und \emph{Daten} (Vgl. Abschn. \ref{sec:xmltree}), bilden also das 2-Tupel $P \coloneqq \langle target, data \rangle$. Der Datenteil kann kann dabei auch der leer sein.

Das Ziel muss ein gültiger Name im Sinne der \acrshort{xml}-Spezifikation sein~\cite[{Regel~[17]}]{maler2008xml}. Das heißt, dass der Name einer Processing-Instruction\ ebenso wie auch der Tag-Name vom Elementen~\cite[{Regel~[40]}]{maler2008xml} mit einem sog. \texttt{NameStartChar} beginnen muss. Dadurch wird ausgeschlossen, dass Tag-Namen mit bestimmten Zeichen beginnen -- darunter auch das Fragezeichen, da dies dazu führen würde, dass sich Start-Tags mit \glspl{pi} verwechselt werden könnten. Insbesondere in \acrshort{sgml} -- zu dem \acrshort{xml} vollständig kompatibel sein soll -- wären solche Tags nicht mehr von \glspl{pi} zu unterscheiden, da laut \acrshort{sgml}-Spezifikation lediglich ein einfaches Größerzeichen anstatt der Kombination aus Fragezeichen und Größerzeiche (\texttt{?>}) zum Schließen der \gls{pi} ausreicht.


Dadurch wird es möglich, Processing Instuctions in \acrshort{jsonml} eindeutig in Form eines \acrshort{json}-Arrays \mintinline{json}{["?target", "data"]} darzustellen~(vgl. Definition \ref{def:jsonmlpi}), das dem 2-Tupel $P$ (s.o.) entspricht. Die Repräsentation von \glspl{pi} ähnelt damit der eines Elementknotens, der einen einzelnen Textknoten (\emph{Character Data}) enthält. Eine Verwechslung ist jedoch durch das dem Zielnamen vorangestellte Fragezeichen ausgeschlossen -- ein Tagname darfnicht mit einem Fragezeichen beginnen, wodurch die Kategorisierung als \gls{pi} eindeutig ist.

\begin{figure}[h]
    \begin{definition} Formale Syntax der \acrfull{jsonml} mit \emph{Processing Instructions}
        \label{def:jsonmlpi}

        Die um Unterstützung von \emph{Processing Instructions} erweitere Syntax ist mit Ausname der Produktionsregeln für \synt{element} identisch zu der Syntax aus Definition~\ref{def:jsonml}.
        \synt{tag-name}, \synt{pi-target} und \synt{pi-data} sind \acrshort{json}-Werte vom Typ String.

        \begin{grammar}
            <element> ::= \[[
        \begin{stack}
            `[' \begin{stack}
                    <tag-name>
                    \begin{stack}
                        `,' <attributes>\\
                    \end{stack}
                    \begin{stack}
                        `,' \begin{rep}
                                <element>\\
                                `,'
                            \end{rep}\\
                    \end{stack}\\
                    `?' <pi-target> `,' <pi-value>
                \end{stack} `]'\\
            \tok{\color{black} \acrshort{json}-String}
        \end{stack}
    \]]

        \end{grammar}

        Enthält das Dokument \emph{Processing Instructions} auf Dokument-Ebene (d.h. als Top-Level-Konstrukt), dann ist das \acrshort{jsonml}-Wurzelelement ein \synt{element} mit einem leeren String als \synt{tag-name}, das die Child-Nodes des Dokuments (d.h. \emph{Processing Instructions} auf Dokumentebene und das Wurzelelement des Dokuments) als Unterelemente enthält.
    \end{definition}
\end{figure}

\section{Überprüfung der Änderungen}

Die syntaktischen Änderungen aus Definition~\ref{def:jsonmlpi} wurden in die JavaScript-Referenz\-implementierung von Stephen McKamey eingearbeitet. Entsprechende \emph{Unittests} zur Sicherstellung der korrekten Umwandlung von \glspl{pi} wurden ebenfalls hinzugefügt.

\begin{figure}[h!]
    \begin{example}[Processing Instructions] Die \acrshort{jsonml}-Repräsentation des \acrshort{xml}-Dokuments aus Beispiel \ref{ex:xmltree} kann nun die \gls{pi} darstellen -- auch solche, die sich außerhalb des Wurzelelements befinden.
        \begin{minted}[autogobble]{json}
            ["", "\n",
                [ "?xml-stylesheet", "href=\"style.css\"" ],"\n","\n",
                ["albums", "\n  ",
                    ["album", {"catno": "ARGO LP-628"}, "\n    ",
                        ["artist", "Ahmad Jamal Trio"], "\n    ",
                        ["title", "At The Pershing"], "\n    ",
                        ["recording", "Recorded ",
                            ["date", "January 16, 1958"], "."
                        ], "\n  "
                    ], "\n"
                ]
            ]
        \end{minted}
    \end{example}
\end{figure}

Bei einer erneuten Überprüfung des \acrshort{jsonml}-Konverters unter Berücksichtigung der o. g. Änderungen wurden deren Korrektheit bestätigt: Alle Testdokumente, auch die zuvor fehlgeschlagenen, lassen sich nun verlustlos von \acrshort{xml} nach \acrshort{json} und wieder zurück konvertieren.

Alle Änderungen wurdem dem \acrshort{jsonml}-Projekt zur Verfügung gestellt.\footnote{Vgl.~\url{https://github.com/mckamey/jsonml/pull/14}}
