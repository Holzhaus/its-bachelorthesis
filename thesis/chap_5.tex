\chapter{Fazit und Ausblick} \label{chap:conclusion}
Ziel der vorliegenden Arbeit war das Finden eines sicheren und verlustlosen Verfahrens zur Konversion von beliebigen \acrshort{xml}-Dokumenten in \acrshort{json}-Datenstrukturen. Dazu wurden die Anforderungen \enquote{Sicherheit} und \enquote{Verlustlosigkeit} zunächst näher bestimmt und ein Kriterienkatalog für ein sicheres und verlustloses Konversionsverfahren erstellt.

Auf dieser Basis wurden dann eine Reihe von Konversionsprogrammen analysiert. Dazu wurde ein Test-Framework implementiert, das die Durchführung der Überprüfung weitestgehen automatisiert.

Es zeigte sich, das keines der überprüften Konversionsverfahren bisher in der Lage ist, \acrshort{xml}-Strukturen verlustlos abzubilden. Teile der \acrshort{xml}-Spezifikation werden von vielen Konversionsverfahren nicht oder nur ungenügend unterstützt. Darunter sind eher selten genutzte Features wie \glspl{pi}, aber auch grundlegende \acrshort{xml}-Eigenschaften wie die Beibehaltung der Elementreihenfolge oder die Möglichkeit zur Nutzung von Mixed Content.

Die \acrfull{jsonml} erfüllte bei der Analyse die meisten der zuvor aufgestellten Kritieren. Probleme hatte es nur bei der Unterstützung von \acrlongpl{pi}. Durch eine Weiterentwicklung des Konversionsverfahrens konnte dieser Mangel behoben werden, sodass das nun alle Anforderungen vollumfänglich erfüllt werden. Auch die Umwandlung von komplexen \acrshort{xml}-basierten Formaten wie \gls{ooxml}, Flat \gls{odf} oder \gls{svg} in \acrshort{json} ist so verlustlos und sicher möglich.

Das hier gezeigte Prüfverfahren vergleicht die \acrshort{xml}-Dokumente vor und nach einem Konversions-Round-Trip. Viele der Konversionsverfahren sind jedoch für eine Analyse dieser Art schon allein deshalb nicht geeignet, da ihnen die Möglichkeit der Rückkonvertierung zu \acrshort{xml} fehlt. Ohne eine solche Funktionalität ist eine Prüfung der Verlustlosigkeit in der hier beschriebenen Art jedoch nicht möglich. Projekte wie die PHP-Bibliothek \emph{xml2jsonphp}~\cite{xml2jsonphp} oder das JavaScript-Modul \emph{xmlToJSON}~\cite{metatribalxmltojson} konnten daher nicht berücksichtigt werden.

Ansatzpunkt weiterer Forschung könnte neben der Evaluierung von weiteren Konversionverfahren und -programmen auch die Einbeziehung zusätzlicher Metadaten wie etwa \acrshortpl{dtd} oder Schema-Informationen sein. Dies könnte Verbesserungen bei dem Verwendung von nativen \acrshort{json}-Datentypen bringen.

Auch die Überprüfung von Verfahren, die beliebige \acrshort{json}-Daten in ein \acrshort{xml}-basiertes Format übersetzen, könnte Gegenstand weiterer Betrachtungen sein, inbesondere im Hinblick auf die Sicherheit gegenüber Angriffen auf \acrshort{xml}-Parser.

Das Aufkommen von Technologien wie \acrshort{json} Schema, \acrshort{json} Reference oder \acrshort{json} Include könnte die Komplexität von \acrshort{json}-Parsern in Zukunft deutlich ansteigen lassen und diese für Angriffe verwundbar machen, für die bislang typischerweise eher \acrshort{xml}-Parser anfällig sind. Eine Evaluierung der Verwundbarkeit von Konvertern gegenüber solchen Angriffen erscheint daher sinnvoll.
