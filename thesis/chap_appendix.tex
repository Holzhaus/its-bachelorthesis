\chapter{Ausgabebeispiele der Konverter}
\label{appx:convexamples}
\begin{figure}[h!]
    \begin{tabular}[t]{cc}
\subfloat[Cobra vs Mongoose]{
    \begin{minipage}[t]{.475\linewidth}
        \inputminted[fontsize=\footnotesize]{json}{examples/cobravsmongoose.json}
    \end{minipage}
} &
\subfloat[GreenCape \acrshort{xml} Converter]{
    \begin{minipage}[t]{.475\linewidth}
    \inputminted[fontsize=\footnotesize]{json}{examples/greencapexml.json}
    \end{minipage}
} \\
\subfloat[Json-lib]{
    \begin{minipage}[t]{.475\linewidth}
    \inputminted[fontsize=\footnotesize]{json}{examples/jsonlib.json}
    \end{minipage}
} &
\subfloat[\acrshort{jsonml}]{
    \begin{minipage}[t]{.475\linewidth}
    \inputminted[fontsize=\footnotesize]{json}{examples/jsonml.json}
    \end{minipage}
}
\end{tabular}
\end{figure}
\begin{figure}[h!]\ContinuedFloat
    \begin{tabular}[t]{cc}
\subfloat[org.json.XML]{
    \begin{minipage}[t]{.475\linewidth}
    \inputminted[fontsize=\footnotesize]{json}{examples/orgjsonxml.json}
    \end{minipage}
} &
\subfloat[x2js (Fork)]{
    \begin{minipage}[t]{.475\linewidth}
    \inputminted[fontsize=\footnotesize]{json}{examples/x2js.json}
    \end{minipage}
}\\
\subfloat[JXON]{
    \begin{minipage}[t]{.475\linewidth}
    \inputminted[fontsize=\footnotesize]{json}{examples/jxon.json}
    \end{minipage}
} &
\subfloat[Json.NET]{
    \begin{minipage}[t]{.475\linewidth}
    \inputminted[fontsize=\footnotesize]{json}{examples/newtonsoftjson.json}
    \end{minipage}
}
\end{tabular}
\end{figure}
\begin{figure}[h!]\ContinuedFloat
    \begin{tabular}[t]{cc}
\multirow{2}{*}[1.5ex]{
\subfloat[Pesterfish]{
    \begin{minipage}[t]{.475\linewidth}
    \inputminted[fontsize=\footnotesize]{json}{examples/pesterfish.json}
    \end{minipage}
}
} &
\subfloat[xmljson (Abdera)]{
    \begin{minipage}[t]{.475\linewidth}
    \inputminted[fontsize=\footnotesize]{json}{examples/xmljson-abdera.json}
    \end{minipage}
}\\
    &
\subfloat[xmljson (Yahoo)]{
    \begin{minipage}[t]{.475\linewidth}
    \inputminted[fontsize=\footnotesize]{json}{examples/xmljson-yahoo.json}
    \end{minipage}
}
\end{tabular}
\end{figure}

\begin{figure}[h!]\ContinuedFloat
    \begin{tabular}[t]{cc}
        \begin{minipage}[t]{.475\linewidth}
\subfloat[xmljson (Badgerfish)]{
    \begin{minipage}{\linewidth}
    \inputminted[fontsize=\footnotesize]{json}{examples/xmljson-badgerfish.json}
    \end{minipage}
}\\
\subfloat[xmljson (GData)]{
    \begin{minipage}{\linewidth}
    \inputminted[fontsize=\footnotesize]{json}{examples/xmljson-gdata.json}
    \end{minipage}
}
\end{minipage}
\hfill{}
\begin{minipage}[t]{.475\linewidth}
\subfloat[xmljson (Cobra)]{
    \begin{minipage}{\linewidth}
    \inputminted[fontsize=\footnotesize]{json}{examples/xmljson-cobra.json}
    \end{minipage}
}\\
\subfloat[xmljson (Parker)]{
    \begin{minipage}{\linewidth}
    \inputminted[fontsize=\footnotesize]{json}{examples/xmljson-parker.json}
    \end{minipage}
}
\end{minipage}
\end{tabular}
\end{figure}

\chapter{Problematische Unicode-Zeichen}
\label{appx:unicode}

\section{Whitespace}
\label{appx:unicode-whitespace}

Bei der Konversion mittels JXON und XMLJSON mit Badgerfish- oder GData-Konversion gehen die folgenden Unicode-Whitespace-Zeichen verloren:

\begin{figure}[h!]
    \begin{center}
        \begingroup
        \footnotesize
        \begin{threeparttable}
        \begin{tabular}{lr}
            \toprule
            {\fontfamily{rubflama}\selectfont\textbf{Name}} & {\fontfamily{rubflama}\selectfont\textbf{Unicode-Codepoint}}\\
            \midrule
            \rowcolor{rubgray!50}\texttt{CHARACTER TABULATION}               & \texttt{000009}\\
                                 \texttt{LINE FEED (LF)}                     & \texttt{00000A}\\
            \rowcolor{rubgray!50}\texttt{CARRIAGE RETURN (CR)}               & \texttt{00000D}\\
                                 \texttt{SPACE}                              & \texttt{000020}\\
            \rowcolor{rubgray!50}\texttt{NEXT LINE (NEL)}\tnote{1}           & \texttt{U+0085}\\
                                 \texttt{NO-BREAK SPACE}                     & \texttt{U+00A0}\\
            \rowcolor{rubgray!50}\texttt{OGHAM SPACE MARK}                   & \texttt{U+1680}\\
                                 \texttt{MONGOLIAN VOWEL SEPARATOR}          & \texttt{U+180E}\\
            \rowcolor{rubgray!50}\texttt{EN QUAD}                            & \texttt{U+2000}\\
                                 \texttt{EM QUAD}                            & \texttt{U+2001}\\
            \rowcolor{rubgray!50}\texttt{EN SPACE}                           & \texttt{U+2002}\\
                                 \texttt{EM SPACE}                           & \texttt{U+2003}\\
            \rowcolor{rubgray!50}\texttt{THREE-PER-EM SPACE}                 & \texttt{U+2004}\\
                                 \texttt{FOUR-PER-EM SPACE}                  & \texttt{U+2005}\\
            \rowcolor{rubgray!50}\texttt{SIX-PER-EM SPACE}                   & \texttt{U+2006}\\
                                 \texttt{FIGURE SPACE}                       & \texttt{U+2007}\\
            \rowcolor{rubgray!50}\texttt{PUNCTUATION SPACE}                  & \texttt{U+2008}\\
                                 \texttt{THIN SPACE}                         & \texttt{U+2009}\\
            \rowcolor{rubgray!50}\texttt{HAIR SPACE}                         & \texttt{U+200A}\\
                                 \texttt{LINE SEPARATOR}                     & \texttt{U+2028}\\
            \rowcolor{rubgray!50}\texttt{PARAGRAPH SEPARATOR}                & \texttt{U+2029}\\
                                 \texttt{NARROW NO-BREAK SPACE}              & \texttt{U+202F}\\
            \rowcolor{rubgray!50}\texttt{MEDIUM MATHEMATICAL SPACE}          & \texttt{U+205F}\\
                                 \texttt{IDEOGRAPHIC SPACE}                  & \texttt{U+3000}\\
            \rowcolor{rubgray!50}\texttt{ZERO WIDTH NO-BREAK SPACE}\tnote{2} & \texttt{U+FEFF}\\
            \bottomrule
            \end{tabular}
            \begin{tablenotes}
                \item[1] Nur bei XMLJSON mit Badgerfish- und XMLJSON-GData-Konvention
                \item[2] Nur bei JXON
            \end{tablenotes}
        \end{threeparttable}
        \endgroup
    \end{center}
\end{figure}

\newpage{}
\section{Zahlen}
\label{appx:unicode-digits}

Bei der Konversion mittels XMLJSON mit Badgerfish-, GData- oder Parker-Konvention wurden folgende Unicode-Zahlenbereiche in ihr ASCII-Äquivalent im Bereich \texttt{U+0030} bis \texttt{U+0039} umgewandelt:

\begin{figure}[hb!]
    \begin{center}
        \begingroup
        \footnotesize
        \begin{tabular}{lrr}
            \toprule
            {\fontfamily{rubflama}\selectfont\textbf{Name}} & \multicolumn{2}{c}{\fontfamily{rubflama}\selectfont\textbf{Unicode-Bereich}}\\
                                                            & {\fontfamily{rubflama}\selectfont\textbf{Beginn}} & {\fontfamily{rubflama}\selectfont\textbf{Ende}}\\
            \midrule
\texttt{ARABIC-INDIC DIGITS} & \texttt{U+0669} & \texttt{U+0660}\\
\texttt{EXTENDED ARABIC-INDIC DIGITS} & \texttt{U+06F9} &  \texttt{U+06F0}\\
\texttt{NKO DIGITS} & \texttt{U+07C9} &  \texttt{U+07C0}\\
\texttt{DEVANAGARI DIGITS} & \texttt{U+096F} &  \texttt{U+0966}\\
\texttt{BENGALI DIGITS} & \texttt{U+09EF} &  \texttt{U+09E6}\\
\texttt{GURMUKHI DIGITS} & \texttt{U+0A6F} &  \texttt{U+0A66}\\
\texttt{GUJARATI DIGITS} & \texttt{U+0AEF} &  \texttt{U+0AE6}\\
\texttt{ORIYA DIGITS} & \texttt{U+0B6F} &  \texttt{U+0B66}\\
\texttt{TAMIL DIGITS} & \texttt{U+0BEF} &  \texttt{U+0BE6}\\
\texttt{TELUGU DIGITS} & \texttt{U+0C6F} &  \texttt{U+0C66}\\
\texttt{KANNADA DIGITS} & \texttt{U+0CEF} &  \texttt{U+0CE6}\\
\texttt{MALAYALAM DIGITS} & \texttt{U+0D6F} &  \texttt{U+0D66}\\
\texttt{SINHALA LITH DIGITS} & \texttt{U+0DEF} &  \texttt{U+0DE6}\\
\texttt{THAI DIGITS} & \texttt{U+0E59} &  \texttt{U+0E50}\\
\texttt{LAO DIGITS} & \texttt{U+0ED9} &  \texttt{U+0ED0}\\
\texttt{TIBETAN DIGITS} & \texttt{U+0F29} &  \texttt{U+0F20}\\
\texttt{MYANMAR DIGITS} & \texttt{U+1049} &  \texttt{U+1040}\\
\texttt{MYANMAR SHAN DIGITS} & \texttt{U+1099} &  \texttt{U+1090}\\
\texttt{KHMER DIGITS} & \texttt{U+17E9} &  \texttt{U+17E0}\\
\texttt{MONGOLIAN DIGITS} & \texttt{U+1819} &  \texttt{U+1810}\\
\texttt{LIMBU DIGITS} & \texttt{U+194F} &  \texttt{U+1946}\\
\texttt{NEW TAI LUE DIGITS} & \texttt{U+19D9} &  \texttt{U+19D0}\\
\texttt{TAI THAM HORA DIGITS} & \texttt{U+1A89} &  \texttt{U+1A80}\\
\texttt{TAI THAM THAM DIGITS} & \texttt{U+1A99} &  \texttt{U+1A90}\\
\texttt{BALINESE DIGITS} & \texttt{U+1B59} &  \texttt{U+1B50}\\
\texttt{SUNDANESE DIGITS} & \texttt{U+1BB9} &  \texttt{U+1BB0}\\
\texttt{LEPCHA DIGITS} & \texttt{U+1C49} &  \texttt{U+1C40}\\
\texttt{OL CHIKI DIGITS} & \texttt{U+1C59} &  \texttt{U+1C50}\\
\texttt{VAI DIGITS} & \texttt{U+A629} &  \texttt{U+A620}\\
\texttt{SAURASHTRA DIGITS} & \texttt{U+A8D9} &  \texttt{U+A8D0}\\
\texttt{KAYAH LI DIGITS} & \texttt{U+A909} &  \texttt{U+A900}\\
\texttt{JAVANESE DIGITS} & \texttt{U+A9D9} &  \texttt{U+A9D0}\\
\texttt{MYANMAR TAI LAING DIGITS} & \texttt{U+A9F9} &  \texttt{U+A9F0}\\
\texttt{CHAM DIGITS} & \texttt{U+AA59} &  \texttt{U+AA50}\\
\texttt{MEETEI MAYEK DIGITS} & \texttt{U+ABF9} &  \texttt{U+ABF0}\\
\texttt{FULLWIDTH DIGITS} & \texttt{U+FF19} &  \texttt{U+FF10}\\
            \bottomrule
        \end{tabular}
        \endgroup
    \end{center}
\end{figure}

\chapter{Patches}
\label{appx:patches}

\section{Entfernung des Whitespace im GreenCape XML Konverter}
\label{appx:greencapexml}

\inputminted[breakautoindent=false,fontsize=\footnotesize]{udiff}{patches/greencapexml-noindent.patch}

\newpage{}
\section{Unterstützung für \acrshortpl{pi} in \acrshort{jsonml}}
\label{appx:jsonmlpi}

\inputminted[breakautoindent=false,fontsize=\footnotesize]{udiff}{patches/jsonml-pi.patch}
