\documentclass[border=10pt]{standalone}
\usepackage{pgfplotstable}
\usepackage{colortbl}
\usepackage{makecell}
\usepackage{booktabs}
\usepackage{threeparttable}
\usepackage{pifont}
\newcommand{\cmark}{\ding{51}}%
\newcommand{\xmark}{\ding{55}}%
\usepackage[T1]{fontenc}
\usepackage{rubfonts2009}
% FIXME: Blue should be #003560, but that'd conflict with the front page colors
\definecolor{rubblue}{HTML}{003660}
\definecolor{rubgreen}{HTML}{8dae10}
\definecolor{rubgray}{HTML}{e7e7e7}
\pgfplotstableread[col sep=comma,comment chars={\#},header=false]{nestingdepth.csv}\datatable
\begin{document}
\begin{tikzpicture}
    \begin{axis}[
            xbar,
            y=0.5cm,
            xmin=0,
            xmax=80,
            ytick=data,
            xlabel={Verschachtelungstiefe der JSON-Container},
            yticklabel style={
                font={\fontfamily{rubscala}\selectfont},
            },
            xticklabel style={
                font={\fontfamily{rubscala}\selectfont},
            },
            yticklabels from table={\datatable}{0},
            restrict x to domain*=0:90,
            visualization depends on=rawx\as\rawx, % Save the unclipped values
            after end axis/.code={ % Draw line indicating break
                \draw [line width=3pt, white, decoration={snake, amplitude=1pt}, decorate] (rel axis cs:1.05,0) -- (rel axis cs:1.05,1.05);
            },
            nodes near coords={%
                \pgfmathprintnumber{\rawx}% Print unclipped values
            },
            /pgf/number format/.cd,
            use comma,
            1000 sep={\,},
            axis x line=bottom,
            y axis line style={draw=none},
            y tick style={draw=none},
            clip=false,
        ]
        \addplot[rubblue,fill=rubblue] table[y expr=\coordindex, x=1] {\datatable};
  \end{axis}
\end{tikzpicture}
\end{document}
