%%% DOCUMENT-INFO
%% AUTHOR		: Jan Holthuis [jan.holthuis@ruhr-uni-bochum.de]
%% VERSION		: 0.1

\documentclass[german]{rubthesis}
\usepackage[mode=buildnew]{standalone}
\usepackage{tabularx}
\usepackage{threeparttable}
\usepackage{colortbl}
\usepackage{keyval}
\usepackage[useregional]{datetime2}
\DTMsetup{showdow=true}

\usepackage[nounderscore,rounded]{syntax}
\usepackage{todonotes}
\usepackage{hyphenat}
\usepackage{subfig}
\usepackage{multirow}
\usepackage{icomma}

\usepackage{mathtools}
\DeclarePairedDelimiter\abs{\lvert}{\rvert}
\DeclarePairedDelimiter\ceil{\lceil}{\rceil}
\DeclarePairedDelimiter\floor{\lfloor}{\rfloor}

\renewcommand{\sdsize}{\scriptsize\color{rubgreen}}

\renewcommand{\syntleft}{\color{rubblue}$\langle$\normalfont\itshape}
\renewcommand{\litleft}{\color{black}"\bgroup\ulitleft}
\renewcommand{\litright}{\ulitright\egroup"}
\renewcommand{\sdlengths}{%
  \setlength{\sdstartspace}{1em}%
  \setlength{\sdendspace}{1em}%
  \setlength{\sdmidskip}{0.5em}%
  \setlength{\sdtokskip}{0.25em}%
  \setlength{\sdfinalskip}{0.5em plus 10000fil}%
  \setlength{\sdrulewidth}{0.2pt}%
  \setlength{\sdcirclediam}{12pt}%
  \setlength{\sdindent}{0pt}%
}

\hyphenation{Java}
\hyphenation{Ruby}

\addbibresource{literature.bib}

\renewcommand{\thesistype}{Bacherlorarbeit}
\renewcommand{\thesisadvisor}{B.~Sc.~Paul~Rösler}
\renewcommand{\thesiskeywords}{XML, JSON, Konversion, Umwandlung, Abbildung}

% Remove theorem type name from list of theorems
\makeatletter
\def\ll@mdexample{%
  \protect\numberline{\csname the\thmt@envname\endcsname}%
  \ifx\@empty\thmt@shortoptarg
    \thmt@thmname
  \else
    \thmt@shortoptarg
  \fi}
\def\l@thmt@mdexample{}
\makeatother

% Set date here
%\day=6 \month=6 \year=2012

% Set name and title
\author{Jan~Holthuis}
\newcommand{\matrikelnummer}{108\,009\.215\,809}
\title{\nohyphens{%
Bridging the Gap: Verlustfreie und sichere Umwandlung von XML-Datenstrukturen
ins JSON-Format
}}
\date{\today}

% Load acronyms in preamble
%\loadglsentries{glossary}
\loadglsentries{acronyms}

% Remove parentheses from citations
\renewcommand{\mkcitation}[1]{ #1}

\makeatletter
\define@key{convtool}{url}{\def\conv@url{#1}}
\define@key{convtool}{license}{\def\conv@license{#1}}
\define@key{convtool}{author}{\def\conv@author{#1}}
\define@key{convtool}{version}{\def\conv@version{#1}}
\define@key{convtool}{language}{\def\conv@language{#1}}
\define@key{convtool}{version}{\def\conv@version{#1}}
\define@key{convtool}{versiondate}{\def\conv@versiondate{#1}}
\setkeys{convtool}{url={},license={},author={},language={},version={Unbekannt}, versiondate={}}%
\newcommand{\convtool}[2][]{%
    \begingroup%
    \setkeys{convtool}{#1}% Set new keys
    \begin{tabular}{p{3.6cm}l}
    \ifx\conv@author\@empty
    \else
        {\fontfamily{rubflama}\selectfont{}\textbf{Autor:}} & \conv@author\\
    \fi
    \ifx\conv@url\@empty
    \else
        {\fontfamily{rubflama}\selectfont{}\textbf{URL:}} & \url{\conv@url}\\
    \fi
    \ifx\conv@license\@empty
    \else
        {\fontfamily{rubflama}\selectfont{}\textbf{Lizenz:}} & \conv@license\\
    \fi
    \ifx\conv@language\@empty
    \else
        {\fontfamily{rubflama}\selectfont{}\textbf{Programmiersprache:}} & \conv@language\\
    \fi
    \ifx\conv@version\@empty
    \else
        {\fontfamily{rubflama}\selectfont{}\textbf{Gestete Version:}} & \conv@version
        \ifx\conv@versiondate\@empty
        \else
            ~(\DTMdate{\conv@versiondate})
        \fi\\
    \fi
    \end{tabular}
    \endgroup%
}
\makeatother


\begin{document}

%% switch to roman paginating for the acknowledgements, table of contents etc.
\pagenumbering{roman} % uncomment this if you like it

%% title page --- made out of expressions defined above
\frontpage{}

\begin{abstract}
Insert abstract here.
\end{abstract}

\pagestyle{scrplain} % switch off headers and footers
\section*{Declaration}
I hereby declare that this submission is my own work and that,\ to the best of my
knowledge and belief,\ it contains no material previously published or written by another
person nor material which to a substantial extent has been accepted for the award of any
other degree or diploma of the university or other institute of higher learning,\ except
where due acknowledgment has been made in the text\@.

\section*{Erkl\"{a}rung}
{\selectlanguage{ngerman}
Hiermit versichere ich,\ dass ich die vorliegende Arbeit selbstst\"{a}ndig verfasst und keine
anderen als die angegebenen Quellen und Hilfsmittel benutzt habe,\ dass alle Stellen
der Arbeit,\ die w\"{o}rtlich oder sinngem\"{a}\ss{} aus anderen Quellen \"{u}bernommen wurden,\ als
solche kenntlich gemacht sind und dass die Arbeit in gleicher oder \"{a}hnlicher Form noch
keiner Pr\"{u}fungsbeh\"{o}rde vorgelegt wurde\@.}

\vspace{2cm}
\rule{4cm}{0.1pt} \hfill \rule{7cm}{0.1pt} \\
\hspace*{1.75cm} \textsc{Date} \hspace*{6.8cm} \textsc{\@author}

\cleardoublepage
\pagestyle{scrheadings} % reenable headers and footers

%% generate table of contents
\tableofcontents

\cleardoublepage

\pagenumbering{arabic} %switches to arabic numbers for the rest of the text
\setcounter{page}{1}

%%
%% include all your chapters as .tex files,
%% each file contains sections \section{name of section},
%% subsections \subsections{...} and so on...
%%

\chapter{Einleitung} \label{chap:intro}
Für den implementationsunabhängigen Austausch von zugleich menschen- als auch
maschinenlesbaren Daten hat sich die \gls{xml}
bewährt. In bestimmten Bereichen wie Web-\acrshortpl{api} hat die \gls{json}
das bewährte \acrshort{xml}-Format inzwischen jedoch überflügelt.

Dabei kann neben der spezifischen Situation auch die eingesetzte
Programmiersprache, die Unterstützung durch das zugrundeliegende Framework
oder die persönliche Präferenz des Entwicklers den Ausschlag geben, welches
Format ein Webservice oder eine Programmbibliothek unterstützt.

Zwecks Interoperabilität zwischen verschiedenen Teilen einer Anwendung kann es
daher notwendig werden, Daten von einem Format temporär in das jeweils andere
zu überführen und später wieder in das ursprüngliche Format zu bringen.

Dabei sollen Daten, die von der Anwendungslogik nicht verändert wurden, auch
nach der Rückübersetzung ins Ursprungsformat unverändert bleiben. Damit dies
gewährleistet ist, darf das zugrunde liegende Konversionverfahren keine
Informationen bei der Umwandlung verwerfen -- es muss also verlustlos % chktex 8
arbeiten.

Das konkrete Konversionsverfahren ist dabei abhängig vom jeweiligen
Ausgangsformat, d.~h.\ die Umwandlungsverfahren für die beiden Richtungen
\begin{enumerate}
    \item $\text{JSON} \rightarrow \text{XML} \rightarrow \text{JSON}$ und
    \item $\text{XML} \rightarrow \text{JSON} \rightarrow \text{XML}$
\end{enumerate}
haben jeweils eigene Anforderungen und sind getrennt voneinander zu betrachten.

Während Abbildung von beliebigen \acrshort{json}-Datenstrukturen in \acrshort{xml} zumindest bei
oberflächlicher Betrachtung trivial erscheint, ist dies beim verlustlosen
Transfer von \acrshort{xml}-Daten ins \acrshort{json}-Format keineswegs der Fall.

Daher solleb in dieser Bachelorarbeit verschiedene Verfahren analysiert
werden, die beliebige \acrshort{xml}-Daten in \acrshort{json} abbilden und aus der resultieren
\acrshort{json}-Datenstruktur wieder \acrshort{xml}-Dokument erstellen können. Sollte keines
der analysierten Verfahren den vorher aufgestellten Kriterien für eine
zuverlässige und sichere Umwandlung genügen, wird ein eigener
Abbildungsalgorithmus entwickelt, bei dem dies der Fall ist.

\section{Motivation}
\label{sec:motivation}
Das Aufkommen des sogenannten \emph{Web 2.0} und die zunehmenden Vernetzung
durch das \gls{iot} ging mit einer erhöhten
Verfügbarkeit von öffentlichen Web-\acrshortpl{api} einher. Als Datenformat wird dabei
häufig \acrshort{json} oder \acrshort{xml} verwendet.

Neben \acrshort{xml}-basierten Webservices, die beispielsweise das \acrshort{saml}-Framework, SOAP,
oder \gls{xmlrpc} verwenden, wird \acrshort{xml} auch in einer Vielzahl weiterer
Einsatzbereiche eingesetzt. So dient \acrshort{xml} den Dateiformaten \acrshort{rss}/\acrshort{asf}, \acrshort{mathml},
\gls{svg} oder \gls{xhtml} als Basis. Auch die gängigen
Office-Dateiformate -- das \acrfull{odf}, Microsofts % chktex 8
\acrfull{ooxml} und Apples iWorks -- bauen auf \acrshort{xml} auf. % chktex 8

Inzwischen gewinnt jedoch \acrshort{json} vor allem im Mobile-
und Web-bereich immer mehr an Bedeutung. Laut dem \acrshort{api}-Verzeichnis
\emph{ProgrammableWeb} unterstützten im Jahr 2013 ca. 60\% aller neu
hinzugefügten \acrshortpl{api} das \acrshort{json}-Format, während
\acrshort{xml} im selben Zeitraum lediglich von 37\% der neuen \acrshortpl{api}
unterstützt wurde.\cite{duvander2013convergence}

\acrshort{json} ist bei bestimmten Aufgaben in puncto Geschwindigkeit und
Ressourcenauslastung deutlich effizienter\cite{nurseitov2009comparison} als
\acrshort{xml}\@. inzwischen setzen auch einige populäre NoSQL-Datenbanken wie
\emph{CouchDB} oder \emph{MongoDB} auf \acrshort{json} zur Speicherung. Auch
MySQL verfügt seit Version 5.7.8 über einen nativen \acrshort{json}-Datentyp.

Die Umwandlung zwischen den beiden Formaten kann aus vielen Gründen
notwendig werden. Soll beispielsweise ein SOAP-Webservice als moderne
\acrshort{json}-\acrshort{rest}-Ressource angeboten werden, muss zwischen \acrshort{xml} und \acrshort{json} konvertiert
werden. Auch bei der Speicherung von \acrshort{xml}-Daten in den oben genannten NoSQL-Datenbanken
ist dies der Fall. Zudem ist die Unterstützung der Formate durch
Programmiersprachen, Frameworks und Applikationen nicht immer gleich gut.

\section{Verwandte Arbeiten}
% TODO: Add related works
List related work \emph{and} the result of this work\@! What is the relevance of this work concerning your thesis\@? If necessary,\ \emph{emphasize} some words in your text, for example words like \emph{not} or \emph{and} are sometimes crucial for understanding\@.

\section{Zielsetzung}
% TODO: Add contribution
What is your contribution?

\section{Aufbau dieser Arbeit}
\label{sec:structure}

Nach der Einleitung wird zunächst eine Einführung in die Struktur der
Formate \acrshort{xml} und \acrshort{json} gegeben (wobei die Formate allerdings selbstverständlich
nicht erschöpfend dargestellt werden können, da dies den Rahmen sprengen
würde). Im Zuge dessen werden ebenfalls die wichtigsten Unterschiede zwischen
beiden Formaten aufgezeigt (z.~B.\ \acrshort{json}-Arrays, \acrshort{xml}-Attribute, Duck Typing,
etc.).

Daran anschließend wird zunächst das Ziel konkret definiert, d.~h.\ es werden
überprüfbaren Kriterien aufgestellt, denen ein Konversionsverfahren genügen
muss. Dazu gehört beispielsweise, Verlustlosigkeit genauer zu definieren (vor
allem im Bezug auf Fragen wie \enquote{Sind Kommentare relevant?}). Hier wird
auch ein Abschnitt zu Angriffen (\emph{\gls{xxe}}, \emph{Billion Laughs}, etc.)
eingefügt, gegen den der jeweilige Konverter gewappnet sein soll.

Dann werden die verschiedenen bestehenden Ansätze zur Umformung aufgezählt
und die zugrunde liegenden Konzepte kurz und grob beschrieben.

Der folgende Abschnitt erklärt den Versuchsaufbau (\acrshort{xml}-Testdokumente).
Die verschiedenen Konversionsverfahren werden dann anhand des zuvor
aufgestellten Anforderungskatalogs geprüft.

Anschließend werden die Ergebnisse der Tests vorgestellt und diskutiert.
Dabei wird erörtert, ob und welche Verfahren die zuvor genannten Bedingungen
erfüllen.

Falls keines der Konversionsverfahren den zuvor aufgestellten Kriterien genügt,
folgt der Abschnitt, der einen eigenen Umwandlungsalgorithmus
entwickelt.

Im Diskussions-Abschnitt werden die aktuellen Möglichkeiten zur Umwandlung
von \acrshort{xml} in \acrshort{json} und wieder zurück abschließend eingeschätzt und bewertet.
Dabei soll auch ein Ausblick auf weitere Verbesserungsmöglichkeiten oder
alternative Wege zum Ziel erörtert werden.

Im letzten Kapitel zu verwandten Arbeiten werden weitere Arbeiten zum Thema
beschrieben und Unterschiede zur eigenen Arbeit herausgearbeitet.


\chapter{Background} \label{chap:background}

Lorem ipsum dolor sit amet consectetuer parturient ac pulvinar magna porttitor. Accumsan vel ac eros laoreet Nulla leo Nulla vel Pellentesque Quisque. Adipiscing penatibus Phasellus egestas leo id neque nec quis est orci. Porta tellus ligula ut ridiculus eros eget ut Vivamus dictum nulla. Dui wisi enim vitae nulla Fusce Curabitur congue consectetuer urna Quisque. Felis Vestibulum Quisque sed Vestibulum et malesuada ac id tristique vitae. Aliquam Suspendisse mattis et libero et tincidunt quis tellus eget consectetuer. Libero Morbi cursus augue eget dapibus tincidunt nunc parturient id arcu. Donec sapien enim Aenean convallis Donec elit tincidunt dolor vitae tellus. Ac consectetuer at tortor malesuada ac dui ligula habitant habitasse congue. 

Mauris interdum sit sem Sed pharetra enim pede nulla auctor ipsum. Tincidunt habitasse auctor Integer non ut laoreet facilisis nibh nunc dictum. Iaculis wisi Praesent hendrerit magnis accumsan massa arcu senectus Vestibulum nibh. Congue et volutpat ipsum pretium in vitae justo ipsum Mauris Pellentesque. Volutpat malesuada parturient ligula Nulla Donec nunc sed tempus urna wisi. Hendrerit metus cursus et auctor nonummy nisl libero mauris purus mollis. Sem justo fringilla pharetra In elit Donec habitasse et semper congue. Pulvinar Curabitur et Integer et convallis convallis in fames mollis Curabitur. Convallis et sapien risus ut Sed rhoncus ac Sed et interdum. Non fermentum justo Nullam nunc habitant ut malesuada nec id Nullam. Et quis lobortis massa volutpat eleifend interdum wisi auctor pede orci. Urna.

Id malesuada eu est nec nec tellus pretium pede condimentum nunc. Malesuada quam enim Curabitur lorem nulla Quisque semper Fusce auctor in. Id wisi pretium parturient tempus Aenean accumsan interdum pellentesque adipiscing scelerisque. Et Nunc tempus Sed quis et semper amet nibh et tempor. Pretium neque consectetuer dignissim ut Sed Proin Ut diam fringilla leo. Condimentum gravida aliquet turpis Sed eu non adipiscing Suspendisse cursus pretium. Urna semper eget Nulla enim Sed tincidunt consequat pellentesque tortor iaculis. Laoreet orci arcu a adipiscing justo Phasellus Sed Nullam mollis tincidunt. Nam Lorem elit condimentum tincidunt elit enim Curabitur Nulla In pretium. Fermentum odio suscipit nibh nec pede justo nec ipsum leo Aliquam. Adipiscing wisi interdum Pellentesque sem ipsum Nulla.

\section{Some text}
Quis convallis fermentum accumsan Ut Nulla libero Morbi quis Nam at. Mi suscipit cursus mus eu Curabitur elit commodo at volutpat turpis. Tristique Sed orci id Aenean tempus quis Nunc ligula lacinia sagittis. In vitae ipsum quis tincidunt id Vivamus tincidunt volutpat ut venenatis. Nullam Nunc accumsan pellentesque et augue sem Integer auctor Nam ac. Hendrerit laoreet tellus urna faucibus pellentesque Nulla turpis wisi pede porta. Felis malesuada Vestibulum amet Curabitur lacinia wisi cursus habitasse Nam massa. Porttitor consequat Sed tempus interdum Donec Vestibulum a Curabitur ante eget. Et nec et Nullam nunc non ligula dignissim sit velit Curabitur. Rhoncus wisi et pretium vestibulum metus fringilla pede nibh volutpat vel. Pellentesque mattis Nullam dolor Vestibulum elit sapien tellus tristique aliquam ligula. Ac sed amet ac Aenean urna amet.

Et pharetra tristique Sed laoreet a risus Suspendisse nibh Mauris ac. Porta vitae Curabitur id quis Curabitur malesuada sed leo urna ac. Augue id augue pulvinar vitae lorem montes Aenean Pellentesque wisi consectetuer. Nec amet mauris Donec condimentum ac eu dapibus semper hendrerit augue. Ante Integer in leo adipiscing et Proin tortor sodales lorem Quisque. Sagittis at ac Integer justo sociis In elit id Vestibulum at. Auctor sociis est faucibus Phasellus quam ac vitae quis tempor ligula. Nec massa ante aliquam elit Aenean scelerisque id leo Nulla ullamcorper. Elit Cum tellus morbi nibh consectetuer Curabitur pellentesque at Aenean ornare. Quis risus Morbi lobortis ullamcorper auctor nisl sit eget lorem eget. Ante Fusce egestas id Sed Phasellus laoreet cursus ridiculus eros adipiscing. Pretium Phasellus tellus laoreet congue condimentum venenatis tempus In pellentesque tempor. Et sed ante id tincidunt porta facilisi Suspendisse congue justo.

Suspendisse dolor sed et feugiat Nunc malesuada justo pulvinar tortor nunc. Turpis nunc mauris interdum id Curabitur Sed elit metus id nibh. Cras elit Praesent metus id platea augue nunc volutpat wisi et. Auctor pellentesque elit Vivamus leo Lorem sollicitudin vitae est Phasellus habitasse. Malesuada In condimentum dolor vitae eget elit Phasellus urna justo elit. Integer Curabitur Sed vitae Nam Nullam ac tempor pretium Morbi id. A venenatis volutpat Pellentesque eleifend Quisque Vestibulum faucibus elit Curabitur semper. Feugiat mattis Maecenas laoreet hendrerit diam Lorem ultrices platea quis consequat. Laoreet ante augue id Vestibulum laoreet Vivamus lacinia libero urna Sed. Tempus Ut Vestibulum Curabitur leo Curabitur laoreet egestas Sed ac cursus. Ante justo tempus dapibus penatibus.

Dui felis tellus quis et vel Aenean Donec adipiscing nibh natoque. Porttitor pellentesque turpis faucibus ante orci Nunc sagittis libero Sed mauris. Phasellus magna a dignissim Pellentesque Sed in quis id lobortis quis. Nulla ut urna nunc nibh et lobortis Sed scelerisque et ante. Dolor Maecenas lacinia nunc fringilla lorem malesuada congue ut sagittis Sed. Feugiat cursus Maecenas vitae vitae semper cursus at et Vivamus elit. Et Morbi elit Quisque lacinia sed magna mauris dapibus Aenean odio. Vestibulum quis eu sit lacinia est penatibus Integer nibh Quisque enim. Velit tristique massa Cras nisl ligula Phasellus quis auctor In eu. Leo Donec elit convallis tincidunt aliquam quis ac neque urna Maecenas. Consectetuer semper dolor wisi pede Vestibulum lacinia condimentum tempor In cursus. Mattis In felis metus porta et Integer congue elit Nam Donec. Sed tempus porta accumsan orci Curabitur augue laoreet Aenean augue.

\section{Some more text}
Aenean id mauris a tellus interdum condimentum tellus interdum volutpat condimentum. Semper pretium justo tincidunt dolor vitae vel Cum Nam auctor auctor. Nulla libero Curabitur auctor pede pretium lorem porttitor laoreet tincidunt adipiscing. Fringilla accumsan vitae montes pede Integer semper cursus justo vel tincidunt. Duis tellus pretium convallis Sed magna molestie faucibus tortor et Nam. Porttitor platea nibh wisi Maecenas dignissim lacus in lobortis orci interdum. Sollicitudin orci elit consequat a ac ante amet quis at pede. Sed ac orci auctor ultrices laoreet id ullamcorper pretium leo semper. Parturient id Vestibulum porttitor fringilla Curabitur Curabitur at ligula turpis ligula. Donec sapien at pretium faucibus et parturient eros volutpat eget a. Augue a Curabitur iaculis quis Sed id Aenean Mauris Nullam odio. Arcu pretium semper Aliquam sed dignissim ipsum malesuada massa Aenean urna. Consectetuer sed tincidunt sagittis Ut platea semper consequat faucibus nibh Quisque. 

Pharetra vel augue magna montes Nam nec eros Vestibulum dui mi. Non mauris Nam nibh Nulla Lorem netus nulla ipsum adipiscing ut. Laoreet fringilla In orci eros magna dui Vestibulum Maecenas dictumst cursus. Lorem tristique Nam volutpat ac nunc lorem commodo id Vestibulum ligula. Nunc velit congue velit eget orci dictum congue laoreet et tortor. Sapien tellus id arcu Sed molestie nascetur gravida dolor dolor pede. Orci orci consequat mattis id Morbi dignissim lorem tincidunt nulla orci. Justo ac vestibulum et Mauris quis adipiscing venenatis ullamcorper Maecenas wisi. Pede fermentum ipsum metus leo fames Nunc Nam et adipiscing sociis. Eget justo lorem at lorem at non cursus dictum tellus Nunc. Tempus felis elit faucibus Vestibulum pede volutpat in Integer commodo tortor. Et dis nec Curabitur scelerisque pede dui libero urna facilisi massa. Elit nibh vitae Nulla venenatis et justo leo Vestibulum nec venenatis. 

Et eros tristique sem elit condimentum Fusce enim tortor pellentesque neque. Tristique massa sit montes Lorem dignissim justo eros eu dictum Nulla. Dictum augue dictum Vestibulum urna elit rhoncus purus facilisi id lacinia. Consectetuer facilisi elit Vivamus et platea pellentesque lorem pretium Aenean eros. Ipsum id ut pharetra Quisque consequat feugiat dignissim id tortor pretium. Praesent Quisque cursus ut et felis ut quis sed wisi parturient. Lorem Donec lobortis sociis consequat sed hendrerit tincidunt tincidunt id rutrum. Quisque sociis aliquet pretium penatibus nec consequat at libero eget habitasse. Fringilla congue accumsan tempor malesuada eleifend aliquam ligula In et sit. Nunc laoreet sociis neque consectetuer nibh massa neque amet elit et. Porttitor Cras at ligula amet tellus morbi porta.


Lorem ipsum dolor sit amet consectetuer parturient ac pulvinar magna porttitor. Accumsan vel ac eros laoreet Nulla leo Nulla vel Pellentesque Quisque. Adipiscing penatibus Phasellus egestas leo id neque nec quis est orci. Porta tellus ligula ut ridiculus eros eget ut Vivamus dictum nulla. Dui wisi enim vitae nulla Fusce Curabitur congue consectetuer urna Quisque. Felis Vestibulum Quisque sed Vestibulum et malesuada ac id tristique vitae. Aliquam Suspendisse mattis et libero et tincidunt quis tellus eget consectetuer. Libero Morbi cursus augue eget dapibus tincidunt nunc parturient id arcu. Donec sapien enim Aenean convallis Donec elit tincidunt dolor vitae tellus. Ac consectetuer at tortor malesuada ac dui ligula habitant habitasse congue. 

Mauris interdum sit sem Sed pharetra enim pede nulla auctor ipsum. Tincidunt habitasse auctor Integer non ut laoreet facilisis nibh nunc dictum. Iaculis wisi Praesent hendrerit magnis accumsan massa arcu senectus Vestibulum nibh. Congue et volutpat ipsum pretium in vitae justo ipsum Mauris Pellentesque. Volutpat malesuada parturient ligula Nulla Donec nunc sed tempus urna wisi. Hendrerit metus cursus et auctor nonummy nisl libero mauris purus mollis. Sem justo fringilla pharetra In elit Donec habitasse et semper congue. Pulvinar Curabitur et Integer et convallis convallis in fames mollis Curabitur. Convallis et sapien risus ut Sed rhoncus ac Sed et interdum. Non fermentum justo Nullam nunc habitant ut malesuada nec id Nullam. Et quis lobortis massa volutpat eleifend interdum wisi auctor pede orci. Urna.

Id malesuada eu est nec nec tellus pretium pede condimentum nunc. Malesuada quam enim Curabitur lorem nulla Quisque semper Fusce auctor in. Id wisi pretium parturient tempus Aenean accumsan interdum pellentesque adipiscing scelerisque. Et Nunc tempus Sed quis et semper amet nibh et tempor. Pretium neque consectetuer dignissim ut Sed Proin Ut diam fringilla leo. Condimentum gravida aliquet turpis Sed eu non adipiscing Suspendisse cursus pretium. Urna semper eget Nulla enim Sed tincidunt consequat pellentesque tortor iaculis. Laoreet orci arcu a adipiscing justo Phasellus Sed Nullam mollis tincidunt. Nam Lorem elit condimentum tincidunt elit enim Curabitur Nulla In pretium. Fermentum odio suscipit nibh nec pede justo nec ipsum leo Aliquam. Adipiscing wisi interdum Pellentesque sem ipsum Nulla.

\section{Even more text}
Quis convallis fermentum accumsan Ut Nulla libero Morbi quis Nam at. Mi suscipit cursus mus eu Curabitur elit commodo at volutpat turpis. Tristique Sed orci id Aenean tempus quis Nunc ligula lacinia sagittis. In vitae ipsum quis tincidunt id Vivamus tincidunt volutpat ut venenatis. Nullam Nunc accumsan pellentesque et augue sem Integer auctor Nam ac. Hendrerit laoreet tellus urna faucibus pellentesque Nulla turpis wisi pede porta. Felis malesuada Vestibulum amet Curabitur lacinia wisi cursus habitasse Nam massa. Porttitor consequat Sed tempus interdum Donec Vestibulum a Curabitur ante eget. Et nec et Nullam nunc non ligula dignissim sit velit Curabitur. Rhoncus wisi et pretium vestibulum metus fringilla pede nibh volutpat vel. Pellentesque mattis Nullam dolor Vestibulum elit sapien tellus tristique aliquam ligula. Ac sed amet ac Aenean urna amet.

Et pharetra tristique Sed laoreet a risus Suspendisse nibh Mauris ac. Porta vitae Curabitur id quis Curabitur malesuada sed leo urna ac. Augue id augue pulvinar vitae lorem montes Aenean Pellentesque wisi consectetuer. Nec amet mauris Donec condimentum ac eu dapibus semper hendrerit augue. Ante Integer in leo adipiscing et Proin tortor sodales lorem Quisque. Sagittis at ac Integer justo sociis In elit id Vestibulum at. Auctor sociis est faucibus Phasellus quam ac vitae quis tempor ligula. Nec massa ante aliquam elit Aenean scelerisque id leo Nulla ullamcorper. Elit Cum tellus morbi nibh consectetuer Curabitur pellentesque at Aenean ornare. Quis risus Morbi lobortis ullamcorper auctor nisl sit eget lorem eget. Ante Fusce egestas id Sed Phasellus laoreet cursus ridiculus eros adipiscing. Pretium Phasellus tellus laoreet congue condimentum venenatis tempus In pellentesque tempor. Et sed ante id tincidunt porta facilisi Suspendisse congue justo.

Suspendisse dolor sed et feugiat Nunc malesuada justo pulvinar tortor nunc. Turpis nunc mauris interdum id Curabitur Sed elit metus id nibh. Cras elit Praesent metus id platea augue nunc volutpat wisi et. Auctor pellentesque elit Vivamus leo Lorem sollicitudin vitae est Phasellus habitasse. Malesuada In condimentum dolor vitae eget elit Phasellus urna justo elit. Integer Curabitur Sed vitae Nam Nullam ac tempor pretium Morbi id. A venenatis volutpat Pellentesque eleifend Quisque Vestibulum faucibus elit Curabitur semper. Feugiat mattis Maecenas laoreet hendrerit diam Lorem ultrices platea quis consequat. Laoreet ante augue id Vestibulum laoreet Vivamus lacinia libero urna Sed. Tempus Ut Vestibulum Curabitur leo Curabitur laoreet egestas Sed ac cursus. Ante justo tempus dapibus penatibus.

Dui felis tellus quis et vel Aenean Donec adipiscing nibh natoque. Porttitor pellentesque turpis faucibus ante orci Nunc sagittis libero Sed mauris. Phasellus magna a dignissim Pellentesque Sed in quis id lobortis quis. Nulla ut urna nunc nibh et lobortis Sed scelerisque et ante. Dolor Maecenas lacinia nunc fringilla lorem malesuada congue ut sagittis Sed. Feugiat cursus Maecenas vitae vitae semper cursus at et Vivamus elit. Et Morbi elit Quisque lacinia sed magna mauris dapibus Aenean odio. Vestibulum quis eu sit lacinia est penatibus Integer nibh Quisque enim. Velit tristique massa Cras nisl ligula Phasellus quis auctor In eu. Leo Donec elit convallis tincidunt aliquam quis ac neque urna Maecenas. Consectetuer semper dolor wisi pede Vestibulum lacinia condimentum tempor In cursus. Mattis In felis metus porta et Integer congue elit Nam Donec. Sed tempus porta accumsan orci Curabitur augue laoreet Aenean augue.

Aenean id mauris a tellus interdum condimentum tellus interdum volutpat condimentum. Semper pretium justo tincidunt dolor vitae vel Cum Nam auctor auctor. Nulla libero Curabitur auctor pede pretium lorem porttitor laoreet tincidunt adipiscing. Fringilla accumsan vitae montes pede Integer semper cursus justo vel tincidunt. Duis tellus pretium convallis Sed magna molestie faucibus tortor et Nam. Porttitor platea nibh wisi Maecenas dignissim lacus in lobortis orci interdum. Sollicitudin orci elit consequat a ac ante amet quis at pede. Sed ac orci auctor ultrices laoreet id ullamcorper pretium leo semper. Parturient id Vestibulum porttitor fringilla Curabitur Curabitur at ligula turpis ligula. Donec sapien at pretium faucibus et parturient eros volutpat eget a. Augue a Curabitur iaculis quis Sed id Aenean Mauris Nullam odio. Arcu pretium semper Aliquam sed dignissim ipsum malesuada massa Aenean urna. Consectetuer sed tincidunt sagittis Ut platea semper consequat faucibus nibh Quisque. 

Pharetra vel augue magna montes Nam nec eros Vestibulum dui mi. Non mauris Nam nibh Nulla Lorem netus nulla ipsum adipiscing ut. Laoreet fringilla In orci eros magna dui Vestibulum Maecenas dictumst cursus. Lorem tristique Nam volutpat ac nunc lorem commodo id Vestibulum ligula. Nunc velit congue velit eget orci dictum congue laoreet et tortor. Sapien tellus id arcu Sed molestie nascetur gravida dolor dolor pede. Orci orci consequat mattis id Morbi dignissim lorem tincidunt nulla orci. Justo ac vestibulum et Mauris quis adipiscing venenatis ullamcorper Maecenas wisi. Pede fermentum ipsum metus leo fames Nunc Nam et adipiscing sociis. Eget justo lorem at lorem at non cursus dictum tellus Nunc. Tempus felis elit faucibus Vestibulum pede volutpat in Integer commodo tortor. Et dis nec Curabitur scelerisque pede dui libero urna facilisi massa. Elit nibh vitae Nulla venenatis et justo leo Vestibulum nec venenatis. 

Et eros tristique sem elit condimentum Fusce enim tortor pellentesque neque. Tristique massa sit montes Lorem dignissim justo eros eu dictum Nulla. Dictum augue dictum Vestibulum urna elit rhoncus purus facilisi id lacinia. Consectetuer facilisi elit Vivamus et platea pellentesque lorem pretium Aenean eros. Ipsum id ut pharetra Quisque consequat feugiat dignissim id tortor pretium. Praesent Quisque cursus ut et felis ut quis sed wisi parturient. Lorem Donec lobortis sociis consequat sed hendrerit tincidunt tincidunt id rutrum. Quisque sociis aliquet pretium penatibus nec consequat at libero eget habitasse. Fringilla congue accumsan tempor malesuada eleifend aliquam ligula In et sit. Nunc laoreet sociis neque consectetuer nibh massa neque amet elit et. Porttitor Cras at ligula amet tellus morbi porta.


\chapter{Versuchsaufbau}
\label{chap:impl}

\begin{samepage}
Im Folgenden wird der Versuchsaufbau, bestehend aus
\begin{enumerate}
    \item{} der Erstellung einer Reihe von Kriterien zur Bewertung der Verlustlosigkeit und Sicherheit von Konversionverfahren,
    \item{} einer Auswahl von zu überprüfenden Konverter, und
    \item{} der Methodik zur Überprüfung der Konversionsverfahren auf Einhaltung der zuvor festgelegten Kriterien
\end{enumerate}
vorgestellt.
\end{samepage}

\section{Bewertungskriterien für Konversionsverfahren}
\label{sec:criteria}

Ziel ist das Finden eines Konversionsverfahrens, welches sowohl \emph{sicher} als auch \emph{verlustlos} arbeitet. Dazu müssen beide Anforderungen zunächst genauer definiert werden.

\subsection{Sicherheit}
\label{sec:criteria-security}

Die Sicherheit eines Konverters hängt in erster Linie von der Sicherheit des eingesetzten \acrshort{xml}-Prozessors ab. Dieser darf gegen keinen der im Abschnitt~\ref{sec:xmlattacks} beschriebenen Angriffe auf \acrshort{xml}-Parser verwundbar sein.

Im Bezug auf Sicherheit stellt die Unterstützung eines \acrshort{xml}-Parsers für Entities das größte Einfallstor für Angriffe dar. Viele der Angriffsvektoren sind nur dann möglich, wenn der Parser General bzw. Parameter Entities auswertet und expandiert.

Das \acrfull{owasp} empfiehlt daher, die Unterstützung für \acrfullpl{dtd} komplett zu deaktivieren, da dies sowohl \acrshort{xxe}-Angriffe als auch \acrshort{dos}-Attacken wie \emph{Billion Laughs} oder \emph{Quadratic Blowup} wirksam verhindert~\cite[Abschn.~1.1]{owasp2017xxeprevention}. Diese Gegenmaßnahme empfiehlt auch Nazim Lala von Microsofts IIS Security Team~\cite{lala2013handlinguntrustedxml}.

Aus Sicherheitsgründen ist die Verarbeitung von \glspl{dtd} für die in der vorliegenden Arbeit überprüften Konverter daher nicht erforderlich.

Dies gilt jedoch nicht für die vordefinierten General Entities \enquote{\mintinline{xml}{&amp;}}, \enquote{\mintinline{xml}{&lt;}}, \enquote{\mintinline{xml}{&gt;}}, \enquote{\mintinline{xml}{&apos;}} und \enquote{\mintinline{xml}{&quot;}}~\cite[Abschn.~4.6]{maler2008xml} bzw. Character References~\cite[Abschn.~4.1]{maler2008xml}, da diese nicht innerhalb einer \gls{dtd} angegeben werden müssen und sie zum Quotieren von Markup in Text-Knoten und Attributen nötig sind.

\subsection{Verlustlosigkeit}
\label{sec:criteria-lossless}

\begin{samepage}
Damit ein Konversionsprozess als \emph{verlustlos} angesehen werden kann, müssen folgende Bedingungen erfüllt sein:
\begin{enumerate}
    \item{} Eingabedokument und Ausgabedokument müssen logisch äquivalent sein.\label{item:ioequal}
    \item{} Sowohl das \acrshort{json}-Zwischenprodukt als auch das \acrshort{xml}-Ausgabedokument müssen gültige Dokumente sein.\label{item:outputwellf}
\end{enumerate}
\end{samepage}

Logisch äquivalent (vgl. Bed. \ref{item:ioequal}) heißt nicht, dass das \acrshort{xml}-Dokument nach dem Kon\-versions\hyp{}Round\hyp{}Trip bitidentisch mit dem Ursprungsdokument sein muss -- es ist ausreichend, dass die Struktur der beiden Dokumente übereinstimmt. Daher werden beide Dokumente zunächst kanonisiert\footnote{Vgl. Abschnitt \ref{sec:c14n}} und im Anschluss verglichen. Stimmt die kanonische Form beider \acrshort{xml}-Dokumente überein, war die Konversion verlustlos.

\begin{figure}[h!]
\begin{definition}[Verlustlosigkeit von Konversion]\label{def:lossless}
    Sei $V \coloneqq \{x \mid x\text{ ist ein valider }\allowbreak{}\text{\acrshort{json}-Wert}\}$ und $W \coloneqq \{x \mid x\text{ ist ein wohlgeformtes \acrshort{xml}-Dokument}\}$.

    Ein Konversionverfahren $K = (f_{enc}, f_{dec})$ heißt \emph{verlustlos} genau dann wenn
    \begin{align}
        (f_{dec}\mathrel{\circ} f_{enc})(x) \; &\mathbin{\stackrel{\mathmakebox[\widthof{=}]{\mathrm{c14n}}}{=}} \; x\\
        f_{enc}: W &\mapsto V\\
        f_{dec}: V &\mapsto W
    \end{align}
    für alle $x \in W$.
\end{definition}
\end{figure}

Für die Feststellung, dass ein ein Konversionsverfahren verlustbehaftet ist, ist es ausreichend \emph{ein einzelnes} wohlgeformtes XML-Dokument zu finden, das das Verfahren nicht verlustlos konvertiert. Um jedoch hinreichend zu beweisen, dass ein Konversionsverfahren vollständig verlustlos arbeitet und somit die Bedingungen aus Def.~\ref{def:lossless} erfüllt, müsste die Umwandlung für \emph{alle} wohlgeformten XML-Dokumente überprüft werden. Da die Menge der möglichen wohlgeformten XML-Dokumente  -- auch bei Beachtung der Einschränkungen aus Abschnitt~\ref{sec:criteria-security} -- unendlich ist, ist eine deduktive Art der Beweisführung jedoch nicht möglich.

Um zu entscheiden, ob ein Konversionsverfahren verlustlos ist, verfolgt die vorliegende Arbeit daher einen empirisch-induktiven Ansatz: Wenn für eine ausreichend große Anzahl verschiedener Fälle gezeigt wird, dass die Bedingung erfüllt ist\footnote{\emph{hier:} die XML-Dokumente vor und nach der Konversion sind logisch Äquivalent}, kann eine allgemeine Erfüllung der Bedingung induziert werden~\cite[S.~2]{rudner1953judgments}.

Daher wird die Verlustlosigkeit einer Konversion im Folgenden anhand konkreter XML-Strukturen und Problemstellungen erläutert. Diese bilden die Grundlage für die anschließende Erstellung konkreter Testfälle in Form von XML-Dokumenten (vgl.~Abschn.~\ref{sec:method-conv}).

\subsubsection{Elemente und Attribute}

Als Kerninhalt jedes \acrshort{xml}\hyp{}Dokuments müssen Elemente und deren Beziehung zueinander erhalten bleiben. Ebenso dürfen Attribute bei der Konversion nicht verloren gehen. Auch die Tag\hyp{}Namen der Elemente transportieren relevante Information, daher müssen sie beibehalten werden -- dies gilt auch für den Tag\hyp{}Namen des Wurzelelements.

\subsubsection{Namespaces}

Namespaces bieten eine Möglichkeit, Teilen eines \acrshort{xml}-Dokuments eine bestimmte Semantik zuzuweisen. Die Zuordnung zwischen Elementen und Namespaces im Dokument darf daher nicht verändert werden.

Obwohl Namespace\hyp{}Prefixe eigentlich frei gewählt werden können, ist es möglich, dass auch sie wichtige Informationen enthalten. Dies wäre beispielsweise dann der Fall, wenn ein im Dokument vorhandener \acrshort{xpath}\hyp{}Ausdruck ein Namenraumprefix referenziert~\cite[Abschn. 4.4]{boyer2001c14n}. Bei einer Umbenennung von Prefixen wäre die korrekte Evaluation des \acrshort{xpath}\hyp{}Ausdrucks nicht mehr möglich. Die Prefix-Bezeichner müssen also erhalten bleiben.

\subsubsection{Character Data}
\label{sec:cdata}

\emph{Character Data} ist in \acrshort{xml}-Dokumente ein ebenso wichtiger Träger von Information wie Elemente. Dabei kann er im Inhaltsteil von Elementen in zwei Varianten vorkommen: Als normaler Text-Knoten oder als CDATA-Abschnitt.

Mithilfe von CDATA-Abschnitten lässt sich Text, der Markup-Zeichen wie beispielsweise das Kleiner-als-Zeichen\footnote{Unicode-Codepoint \texttt{U+003C}: \texttt{LESS-THAN SIGN}} enthält, direkt in ein \acrshort{xml}-Dokument einbetten, ohne dass diese Zeichen als Markup interpretiert werden.

Dies ist vor allem dann sinnvoll, wenn es unpraktikabel ist, alle Markup-Zeichen im Text einzeln durch die jeweilige Zeichen- oder Entity-Referenz zu ersetzen. CDATA stellt somit eine weitere Möglichkeit dar, Zeichendaten in einem \acrshort{xml}-Dokument anzugeben.~\cite[Abschnitt~2.4]{maler2008xml}

Der Unterschied zwischen Zeichendaten aus CDATA-Abschnitten und solchen, bei denen dies nicht der Fall ist, ist jedoch lediglich ein syntaktischer. Daher werden bei der \acrlong{c14n} alle CDATA-Abschnitte im Eingabedokument durch den entsprechenden \emph{Character Content} ersetzt~\cite[Abschnitt~2.1]{boyer2001c14n}.

Für die Verlustlosigkeit der Konversion ist es somit unerheblich, ob die CDATA-Abschnitte im Ursprungsdokument als solche erhalten bleiben, oder lediglich die Zeichendaten beibehalten werden.

\subsubsection{Kommentare}

\acrshort{xml} verfügt über die Möglichkeit, Dokumente mit Kommentaren zu versehen. Diese sind jedoch nicht Teil der Zeichendaten des \acrshort{xml}-Dokuments. Die Möglichkeit, Kommentare programmatisch auszuwerten, können \acrshort{xml}-Parser zwar bereitstellen, dies ist jedoch optional.

Zudem ist auch bei der Implementierung der \acrlong{c14n} die Unterstützung von \emph{Kanonischem \acrshort{xml} mit Kommentaren} lediglich empfohlen, während die Möglichkeit der Umwandlung in \emph{Kanonisches \acrshort{xml}} ausschließlich aller Kommentare zwingend erforderlich ist.~\cite[Abschnitt~2.1]{boyer2001c14n}

Folglich ist es nicht nötig, dass sich die Kommentare im \acrshort{xml}-Eingabedokument nach dem $\acrshort{xml}\rightarrow{}JSON\rightarrow{}\acrshort{xml}$ Roundtrip auch in der Ausgabe wiederfinden.

\subsubsection{\acrfullpl{pi}}

Wie bereits \acrshort{sgml} unterstützt \acrshort{xml} die Einbettung von Anweisungen, die für die verarbeitetende Applikation bestimmt sind. Diese werden \acrfull{pi} genannt (vgl. Abschn.~\ref{sec:xmlbasics}).

In der Praxis werden \glspl{pi} eher selten eingesetzt. Genutzt werden sie unter anderem um Darstellungsinformationen in Form von Stylesheets mit \acrshort{xml}-Dokumenten zu verknüpfen~\cite[Abschnitt 4]{xmlstylesheet}.

Ein weiteres prominentes Beispiel für eine Praxisanwendung ist die Microsoft Office Suite, die seit der Version 2003 Office-Dokumente als einzelne \acrshort{xml}-Dateien speichern kann. Diese setzen die unspezifische Dateiendung \texttt{*.xml} ein und würden daher mit einem generisch \acrshort{xml}-Anzeigeprogramm geöffnet werden. Mithilfe der Processing Instruction \enquote{\mintinline{xml}{<?mso-application progid="Word.Document"?>}} werden Windows- bzw. Internet Explorer angewiesen, diese als Microsoft Word-Document zu behandeln.~\cite[Abschnitt 3.2]{tverskov2008understandingpi}

Auch die XSLT-Stylesheets des DocBook-Formats nutzen Processing Instructions, um spezielle Formatierungen für die verschiedenen Ausgabeformate festzulegen.~\cite[{Kapitel \enquote{User Reference: PIs}}]{docbookxsl}

Wie auch Kommentare sind sie nicht Teil der Zeichendaten eines \acrshort{xml}-Dokuments, die Unterstützung durch \acrshort{xml}-Parser ist laut Spezifikation jedoch zwingend vorgeschrieben.

Auch bei der \acrlong{c14n} bleiben die \glspl{pi} erhalten.~\cite[Abschnitt 2.3]{boyer2001c14n}

\subsubsection{Dokumentordnung}

\paragraph{Elementordnung}

Die Spezifikation der \acrshort{c14n} bezieht sich bezüglich der Ordnung auf die \acrshort{w3c}-Empfehlung zur \acrfull{xpath}~\cite[Abschnitt~2.2]{boyer2001c14n}.

\begin{foreigndisplayquote}{english}[{~\cite[Abschnitt~5]{clark1999xpath1}}]
    There is an ordering, document order, defined on all the nodes in the document corresponding to the order in which the first character of the \acrshort{xml} representation of each node occurs in the \acrshort{xml} representation of the document after expansion of general entities. Thus, the root node will be the first node. Element nodes occur before their children. Thus, document order orders element nodes in order of the occurrence of their start-tag in the \acrshort{xml} (after expansion of entities).
\end{foreigndisplayquote}

Folglich muss die Reihenfolge der Element-Nodes bei der Konversion beibehalten werden.

\paragraph{Ordnung von Attributen und Namespaces}

Der Reihenfolge, in der die Attribute eines Elements im Start-Tag bzw. Leeres-Element-Tag angegeben wurden, kommt laut \acrshort{xml}-Spezi\-fikation keine Bedeutung zu.~\cite[Abschnitt~3.1]{maler2008xml}

Dies wird auch von der \acrshort{xpath}-Spezifikation untermauert, die die Reihenfolge von Name\-space-Deklarationen und Attributen als implementierungsabhängig festlegt.~\cite[Abschnitt~5]{clark1999xpath1}

Die Reihenfolge der Attribute eines Elements nach einem $\text{\acrshort{xml}}\rightarrow{}\text{\acrshort{json}}\rightarrow{}\text{\acrshort{xml}}$-Roundtrip ist daher beliebig und muss nicht mit der Reihenfolge vor der Umwandlung identisch sein.

\subsubsection{Whitespace}

XML-Parser müssen grundsätzlich jeglichen Whitespace -- also Leerzeichen, Tabs und Zeilenumbrüche --  innerhalb des Wurzelelements eines Dokuments an die verarbeitende Applikation weiterreichen~\cite[Abschn.~2.10]{maler2008xml}, während Whitespace außerhalb des Wurzelelements keine Bedeutung zukommt~\cite[Abschnitt~2.1]{boyer2001c14n}.

\begin{samepage}
    Ausgenommen hiervon sind jedoch bestimmte Sonderfälle, in denen Whitespace durch den \acrshort{xml}-Prozessor \emph{normalisiert} wird:
    \begin{enumerate}
        \item{} Zeilenenden, d.h. die verschiedenen Kombinationen der Wagenrücklauf- und \linebreak{}Zeilenvorschub-Zeichen, werden zu einem einfachen Zeilenvorschub umgewandelt ~\cite[Abschn.~2.11]{maler2008xml}.
        \item{} In Attributen vorkommende Whitespace-Zeichen (keine Character References) werden zu Leerzeichen umgewandelt~\cite[Abschn.~3.3.3]{maler2008xml}
    \end{enumerate}
\end{samepage}

Zwar kann Whitespace durch \glspl{dtd} oder \acrshort{xsd}-Schemata auch als unsignifikant markiert werden~\cite{page2005whitespace}, dies wird jedoch nur von \emph{validierenden} XML-Prozessoren beachtet und geht somit über den Rahmen der vorliegenden Arbeit hinaus.

\subsubsection{Mixed Content}
\label{sec:mixedcontent}

Eine Besonderheit von \acrshort{xml} und auch \acrshort{sgml} ist, dass die Spezifikation sogenannten \emph{Mixed Content} erlaubt. Dieser liegt vor, wenn ein Element sowohl Character Data als auch Kindelemente enthält~\cite[Abschnitt 3.2.2]{maler2008xml}.

\begin{example}[Mixed Content]~
    \begin{minted}{xml}
<mixed>This is an element<br /> containing <emph>mixed content</emph>.<mixed>
    \end{minted}
    \captionof{figure}{Mit Elementen durchsetzte \emph{Character Data} wird \emph{Mixed Content} genannt.}
\end{example}

\emph{Mixed Content} stellt \acrshort{xml}-Parser vor besondere Herausforderungen~\cite{mcgrath2002mixedcontent}. Zudem existiert kein \acrshort{json}-Äquivalent von \emph{Mixed Content}, was eine Konversion erschwert.

\subsubsection{Typinferenz bei der Konversion zu JSON}

Zwar ist es möglich, mittels einer \acrfull{xsd} die in einem \acrshort{xml}-Dokument enthaltenten Datentypen genauer festzulegen, direkte syntaktische Unterstützung für Zahlen bietet \acrshort{xml} jedoch im Gegensatz zur \acrlong{json} nicht.

\begin{samepage}
Wird kein Schema verwendet bzw. legt ein Schema den Datentyp eines Elements nicht anderweitig fest, so ist es standardmäßig vom Typ \texttt{xsd:anyType}:
\begin{foreigndisplayquote}{english}[{~\cite[Abschnitt 2.2.1.1]{xmlschema11-1}}]
    A special complex type definition, (referred to in earlier versions of this specification as `the ur-type definition') whose name is \textbf{anyType} in the XSD namespace, is present in each ·\emph{XSD schema}·. The \textbf{definition of anyType} serves as default type definition for element declarations whose \acrshort{xml} representation does not specify one.
\end{foreigndisplayquote}
\end{samepage}

Elemente dieses Typs unterliegen keinen Beschänkungen. Daher ist es ohne Zuhilfenahme eines Schemas nicht möglich, Aussagen über den Wertebereich eines Elements, eines Attributs oder einer Text Node zu treffen.

\begin{example}[Type Inference]~
    Das folgende \acrshort{xml}-Dokument, enthält einen numerischen Wert, wobei in \acrshort{xml} kein syntaktischer Unterschied zwischen Strings und Zahlen besteht.
    \begin{center}
        \mintinline{xml}{<price>5.99</price>}
    \end{center}
Die \acrshort{json}-Entsprechung des \acrshort{xml}-Dokuments \emph{ohne} Typinferenz enthält den Zahlenwert als String:
    \begin{center}
        \mintinline{json}{{ "price": "5.99" }}
    \end{center}
Wird jedoch Typinferenz verwendet, enthält die \acrshort{json}-Struktur den Zahlenwert als \texttt{Number}:
    \begin{center}
        \mintinline{json}{{ "price":  5.99  }}
    \end{center}
\end{example}

Ist das Schema eines \acrshort{xml}-Dokuments nicht bekannt, scheint es daher naheliegend, die Datentypen aus den im Dokument enthaltenen Werten abzuleiten. Enthält eine Text Node oder ein Attribut beispielsweise die Zeichenkette \mintinline{text}{123}, könnte daraus auf einen numerischer Datentyp geschlossen werden. Dieses Prinzip der \emph{Type Inference} nutzt beispielweise Microsoft im Rahmen des \emph{.NET Frameworks} um das \acrshort{xml}-Schema auf Basis von einem oder mehreren \acrshort{xml}-Dokumenten zu \enquote{erraten}.~\cite{msdn2017inferxmlschema}

Im Rahmen der \acrshort{xml} zu \acrshort{json}-Konversion ist es zwar wünschenswert, die nativen Datentypen der \acrshort{json}-Spezifikation voll auszunutzen, ein solches Vorgehen bringt jedoch mehrere Probleme mit sich:

\paragraph{Fehlerkennung von Typen}

Die oben beschriebene Vorgehensweise zur Typableitung kann zur Fehlerkennung von Datentypen führen.

Darf eine Text Node beispielsweise eigentlich beliebige Zeichen enthalten, enthält aber \emph{zufälligerweise} ausschließlich Ziffern, würde fälschlicherweise ein Zahlentyp erkannt werden. Dies könnte zu Problemen mit der verarbeitetenden Applikation führen, die stattdessen eine Zeichenkette erwartet.

\paragraph{Einschränkungen durch Wertebereiche}

Ein weiteres Problem bei der \emph{Type Inference} kann durch die unterschiedlichen Wertebereichsgrenzen der verschiedenen Datentypen entstehen.  Während Zeichenketten in vielen Programmiersprachen mehrere tausend Zeichen lang sein dürfen, ist der Wertebereich von numerischen Datentypen in der Regel deutlich eingeschränkter. Wird ein Zahlenwert vom Konversionsprogramm also in einen nativen Datentyp umgewandelt, für den ein kleinerer Wertebereich gilt als durch Strings darstellbar sind, führt dies zu Fehlern oder Informationsverlust.

Die Programmiersprache \emph{JavaScript} erlaubt beispielsweise Strings mit einer Länge von bis zu $n=2^{53}-1 \approx 9 \cdot 10^{15}$ Zeichen, d.h. als Strings abgelegte Zahlen können rund 9 Billarden Stellen haben.~\cite[Abschnitt 6.1.4]{ecma262} Für Zahlen im Dezimalsystem entspräche dies dem Wertebereich $\left\{x \in \mathbb{Z} \mid -\left(10^{n-1}-1\right) \leq x \leq 10^{n}-1\right\}$.

Für den Datentyp \texttt{Number} nutzt JavaScript 64-bit-Fließkommazahlen nach dem \acrshort{ieee}-754-Standard\footnote{Fließkommazahlen mit doppelter Genauigkeit (\texttt{binary64}-Typ nach \acrshort{ieee}-754)}~\cite[Abschnitt 4.3.20]{ieee754,ecma262}. Für Ganzzahlen gilt daher der \enquote{sichere} Wertebereich $\left\{x \in \mathbb{Z} \mid \abs{x} \leq 2^{53}-1\right\}$~\cite[Abschnitt 20.1.2.5]{ecma262}. Dies entspricht einer Zahl mit lediglich $\floor{\log_{10}\left(2^{53}-1\right)}+1 = 16$ Ziffern im Dezimalsystem.

Zahlen außerhalb dieses Wertebereichs können nicht mehr fehlerfrei eingesetzt werden. Auf den möglicher Informationsverlust bei dem Einsatz von Zahlentypen in \acrshort{json} wird in der \acrshort{ietf}-Spezifikation daher explizit hingewiesen~\cite[Abschn.~6]{rfc7159}.

\begin{figure}[b!]
\begin{example}[Informationsverlust durch Typumwandlung in JavaScript]
    Die Umwandlung von numerischen Zeichenketten in den \texttt{Number}-Typ kann in JavaScript bei Zahlen $> 2^{53}-1$ zu Problemen führen.
    \inputminted[firstline=2,firstnumber=1,mathescape]{javascript}{typeinference.js}
\end{example}
\end{figure}

Ähnliches gilt für Konversionsverfahren, die aus Werten wie \enquote{\mintinline{text}{yes}} und \enquote{\mintinline{text}{true}} den Boole'schen Datentyp ableiten: Nach dem Konversion ist nicht mehr feststellbar, ob der Ursprungswert nun \enquote{\mintinline{text}{yes}} oder \enquote{\mintinline{text}{true}} lautete.

\subsubsection{Unterstützung des Zeichenbereichs}

Die \acrshort{xml}-Spezifikation erlaubt eine große Anzahl verschiedener Unicode-Zeichen weit über den ASCII-Raum hinaus ~\cite[Regel~2]{maler2008xml}. Da bei der Konvertierung keine Zeichendaten verloren gehen dürfen (vgl. Abschn.~\ref{sec:cdata}), müssen Konverter in der Lage sein, diese Zeichen zu \acrshort{json} zu übersetzen und ggf. korrekt zu quotieren.

Dies gilt auch für die Zeichen, von deren Einsatz \acrshort{xml}-Spezifikation ausdrücklich abgeraten wird~\cite[Abschn.~2.2]{maler2008xml}, da es sich dennoch um wohlgeformtes \acrshort{xml} handelt.

\section{Auswahl der \acrshort{xml}-\acrshort{json}-Konverter}
\label{sec:converters}

\begin{figure}[hb!]
    \begin{threeparttable}
        \captionof{table}{Übersicht der überprüften Konverter}
        \begin{tabularx}{\textwidth}{p{2.65cm}Xp{1.6cm}p{1.5cm}p{2.1cm}}
            \rowcolor{white} % strange hack
            \toprule
            {\fontfamily{rubflama}\selectfont{}\bfseries Konverter} &
            {\fontfamily{rubflama}\selectfont{}\bfseries Autor} &
            {\fontfamily{rubflama}\selectfont{}\bfseries Lizenz} &
            {\fontfamily{rubflama}\selectfont{}\bfseries Sprache} &
            {\fontfamily{rubflama}\selectfont{}\bfseries Version}\\
            \midrule
            Cobra vs\newline Mongoose\tnote{a} & {Paul Battley} & MIT & Ruby & \texttt{0.0.2}\newline 27.06.2006\\
            \rowcolor{rubgray!80}
            GreenCape \acrshort{xml} Converter\tnote{b} & {Niels Braczek} & MIT & PHP & \texttt{a830542}\newline 02.07.2015\\
        Json-lib\tnote{c} & {Andres Almiray\tnote{1}} & Apache~2.0 & Java & \texttt{2.4}\newline 14.12.2010\\
            \rowcolor{rubgray!80}
            \acrshort{jsonml}\tnote{d} & {Stephen M. McKamey} & MIT & JavaScript & \texttt{2.0.0}\newline 09.04.2016\\
            JXON\tnote{e} & {Martin Raifer, Mozilla} & GNU GPL~3.0 & JavaScript & \texttt{2.0.0-beta.4}\newline 22.11.2016\\
            \rowcolor{rubgray!80}
            Json.NET\tnote{f} & {James Newton-King} & MIT & C\# & \texttt{10.0.3}\newline 18.06.2017\\
            org.json.XML\tnote{g} & {Sean Leary / JSON.org} & MIT & Java & \texttt{20160810}\newline 10.08.2016\\
            \rowcolor{rubgray!80}
            Pesterfish\tnote{h} & {Jacob Smullyan} & MIT & Python & \texttt{1578db9}\newline 22.11.2010\\
            x2js\tnote{i} & {Abdulla G. Abdurakh\-manov} & Apache~2.0 & JavaScript & \texttt{185e410}\newline 04.01.2016\\
            \rowcolor{rubgray!80}
            x2js (Fork)\tnote{j} & {Sander Saares / Axinom\tnote{2}} & Apache~2.0 & JavaScript & \texttt{3.1.0}\newline 05.12.2016\\
            xmljson\tnote{k} & {S. Anand} & MIT & Python & \texttt{0.1.7}\newline 09.05.2017\\
            \rowcolor{rubgray!80}
            \bottomrule
        \end{tabularx}
        {\footnotesize
        \begin{tablenotes}[para]
            \item[1] Basiert auf Software von Douglas Crockford.
            \item[2] Fork der \emph{x2js}-Bibliothek von Abdulla G. Abdurakhmanov.
            \item[a] \url{https://rubygems.org/gems/cobravsmongoose}
            \item[b] \url{https://github.com/GreenCape/xml-converter}
            \item[c] \url{http://json-lib.sourceforge.net/}
            \item[d] \url{http://www.jsonml.org/}
            \item[e] \url{https://github.com/tyrasd/jxon}
            \item[f] \url{https://github.com/stleary/JSON-java}
            \item[g] \url{https://github.com/JamesNK/Newtonsoft.Json}
            \item[h] \url{https://bitbucket.org/smulloni/pesterfish/}
            \item[i] \url{https://github.com/abdmob/x2js}
            \item[j] \url{https://github.com/x2js/x2js}
            \item[k] \url{https://github.com/sanand0/xmljson}
        \end{tablenotes}}
    \end{threeparttable}
\end{figure}

Inzwischen sind eine große Anzahl an Applikationen und Programmbibliotheken für die Umwandlung zwischen \acrshort{json} und \acrshort{xml} verfügbar.

\begin{samepage}
Aufgrund des in dieser Arbeit zur Anwendung kommenden Prüfverfahrens konnten jedoch nur solche Konverter betrachtet werden, die
\begin{enumerate}
    \item{} \acrshort{xml}-Dokumente in \acrshort{json} umwandeln können und
    \item{} aus den so erhaltenen \acrshort{json}-Strukuren wieder \acrshort{xml}-Dokumente erstellen können.
\end{enumerate}
\end{samepage}

Da für viele XML-Strukturen kein Äquivalent in der JSON-Spezifikation existiert, gibt es meist verschiedene Möglichkeiten der Abbildung von XML-Inhalten in JSON.
Die gebräuchlichsten Varianten haben sich in Form verschiedener Konvertierungskonventionen herausgebildet, die jeweils beschreiben, wie XML-Strukturen in JSON dargestellt werden~\cite{open311-conventions}. Dabei unterscheidet sich das anhand der verschiedenen Konventionen produzierte JSON zum Teil stark, beispielsweise darin, welcher JSON-Container\-typ zur Darstellung eines XML-Elements genutzt wird oder wie Attribute dargestellt werden. Die für die Analyse im Rahmen der vorliegenden Arbeit ausgewählten Konverter sollten daher nach Möglichkeit verschiedene Umwandlungsverfahren und Konventionen implementieren.

Da die Formate insbesondere im Webbereich eingesetzt werden und Webservices sowie \acrshortpl{api} ein relevantes Einsatzfeld für \acrshort{xml}-\acrshort{json}-Konversion darstellen, wurde bei der Auswahl zudem darauf geachtet, ein breites Spektrum verschiedener populärer Programmiersprachen aus diesem Bereich abzudecken. Sowohl die klassischen Programmiersprachen für Web-Applikationen --  Java, PHP und JavaScript -- als auch C\#, Ruby und Python, die im Web in Form von ASP.NET, Ruby-on-Rails bzw. Django oder Flask zum Einsatz kommen, werden durch die Auswahl abgedeckt.

\begin{description}
    \item[Cobra vs Mongoose] Diese Implementierung ist in Form eines Ruby-Gems verfügbar und übersetzt \acrshort{xml}-Dokumente in Ruby-Datenstrukturen (Hashes), kann aber auch \acrshort{json}-Daten ausgeben. Bei der Umwandlung setzt der  Konverter auf die sogenannte \emph{Badgerfish}-Konvention~\cite[Abschn.~3]{cobravsmongoose,open311-conventions}.
    \item[GreenCape \acrshort{xml} Converter] Der in PHP implementierte Konverter kann \acrshort{xml}-Daten in asso\-ziative PHP-Arrays umwandeln. Diese können dann zu \acrshort{json}-Werten serialisiert werden.
    \item[Json-lib] Die Java-Bibliothek \texttt{net.sf.json-lib} baut auf Software des \acrshort{json}-Entwicklers Douglas Crockford auf und wird in über 400 anderen Projekten eingesetzt~\cite{mvn-jsonlib}. Das Paket beinhaltet unter anderem auch eine Klasse zum (De-)Serialisieren von \acrshort{xml}-Daten.
    \item[{\acrshort{jsonml}}] Die \acrfull{jsonml} ist ein \acrshort{json}-basiertes Format zur Speicherung von \acrshort{xml}-Markup. Neben der JavaScript-Bibliothek existieren auch Implementierungen in anderen Programmiersprachen, z.B. in Java\footnote{Das \texttt{org.json}-Paket stellt eine entsprechende Klasse bereit.}  oder PHP\footnote{Als Teil des FluentDOM-Projekts, siehe: \url{https://github.com/FluentDOM/FluentDOM/blob/master/src/FluentDOM/Serializer/Json/JsonML.php}}.
    \item[{\acrshort{jxon}}] Bei dem JavaScript-Modul handelt es sich um eine bidirektionale Bibliothek für die \acrlong{jxon} und eine Weiterentwicklung der ursprünglich von Mozilla veröffentlichten Implementierung.
    \item[Newtonsoft Json.NET] Das JSON-Framework für C\#.NET wurde insgesamt mehr als 64 Millionen Mal von Paket-Repository NuGet heruntergeladen~\cite{nugetjsonnet} und war zum Testzeitpunkt das am meisten heruntergeladene .NET-Paket~\cite{nugetstatistics}. Es ermöglicht unter anderem auch die Konversion von XML-Dokumenten in das JSON-Format.
    \item[org.json.XML] Das Paket \texttt{org.json} ist die Referenzimplementierung des \acrshort{json}-Formats für Java und bietet auch ein Konversionsverfahren in Form einer dedizierten \acrshort{xml}-Klasse. Es wird von mehr als 1\,600 Projekten eingesetzt, darunter auch die Google Android Library, das Google Web Toolkit (GWT) oder die Objektserialisierungs\hyp{}Bibliothek XStream~\cite{mvn-orgjsonxml}.
    \item[Pesterfish] Das Python-Modul Pesterfish konvertiert zwischen \acrshort{xml} und Python-\linebreak{}Dictionaries, die dann zu \acrshort{json} serialisiert werden können. Laut dem Autor wurde das Konversionsverfahren als \enquote{Reaktion auf die Badgerfish-Konvention} entwickelt\footnote{Der Autor schreibt dies in einem Kommentar im Quelltext des Moduls.}. Für die Verarbeitung der \acrshort{xml}-Daten baut das Modul auf die \texttt{ElementTree}-\acrshort{api} auf und erlaubt Entwicklern auch die Angabe eines eigenen Parsers, beispielsweise \texttt{defusedxml}.
    \item[x2js] Die JavaScript-Bibliothek ist mit 475 Sternen und 219 Forks auf GitHub recht populär~\cite{githubx2js} und erlaubt eine große Zahl von Einstellungsmöglichkeiten.
    \item[x2js (Fork)] Hierbei handelt es sich um einen für die Verwendung mit NodeJS ausgelegten Fork der xj2s-Bibliothek, der seit der Abspaltung im Oktober 2015 unabhängig weiterentwickelt wird.
    \item[xmljson] Das Python-Paket \emph{xmljson} wurde mehr als 11\,500 mal vom Python-Repository PyPI heruntergeladen~\cite{pypistats-xmljson}. Eine Besonderheit ist, dass es die Konversion von \acrshort{xml}-Daten in Python-Dictionaries/Lists bzw. \acrshort{json} anhand von 6 verschiedenen Konventionen (\emph{Abdera}, \emph{Badgerfish}, \emph{Cobra}, \emph{GData}, \emph{Parker} und \emph{Yahoo}) unterstützt.
\end{description}


\section{Methodik}
\label{sec:method}

\subsection{Überprüfung der Konversionqualität}
\label{sec:method-conv}

\begin{figure}[bp!]
    \begin{center}
        \includestandalone[width=0.8\textwidth]{flowchart_conversiontest}
        \caption{Ablauf der Konversionstests}
    \end{center}
\end{figure}

Zur Überprüfung der Konversionsqualität wurden für alle der in Abschnitt~\ref{sec:criteria-lossless} festgelegten Kriterien und diskutierten Probleme ein oder mehrere Testfälle erstellt. Jeder Testfall besteht aus einer \acrshort{xml}-Datei, anhand der bestimmte \acrshort{xml}-Features oder Problemstellungen nachvollzogen werden können. Die Dateien werden dazu zunächst mit einem Konverter in das \acrshort{json}-Format umgewandelt. Die daraus resultierenden \acrshort{json}-Daten werden im Anschluss wieder zurück in das \acrshort{xml}-Format konvertiert, d.h. es wird ein kompletter Round-Trip vollzogen.

Gemäß Definiton~\ref{def:lossless} in Abschnitt~\ref{sec:criteria-lossless} gibt ein Abgleich zwischen den resultierenden \acrshort{xml}-Daten und dem ursprünglichen \acrshort{xml}-Dokument dann Aufschluss über den eventuellen Verlust signifikanter Informationen. Dazu wird überprüft, ob beide Dokumente logisch äquivalent sind, indem beide in \emph{kanonisches \acrshort{xml}} umgewandelt werden. Ob Informationsverlust aufgetreten ist, kann dann mit simplen Vergleich geprüft werden.

Es muss zudem sichergestellt sein, dass bei der Konversion auch tatsächlich valides \acrshort{json} bzw. wohlgeformtes \acrshort{xml} zurückgegeben wird. Da das von Konverter zurückgelieferte Dokument im Zuge der \acrlong{c14n} geparst wird, werden Verstöße gegen die Wohlgeformtheitsanforderungen der \acrshort{xml}-Spezifikation erkannt und führen zum Nichtbestehen des Tests.

Die \acrshort{json}-Zwischendaten werden ebenfalls geparst und auf Verstöße gegen die \acrshort{json}-Spezifikation untersucht. Zudem werden die Daten vor der Rückkonvertierung neu formatiert, sodass nur Informationen weitergegeben werden, die auch tatsächlich signifikant im Sinne der Spezifikation sind, d.h. dass beispielsweise Whitespace außerhalb von Zeichenketten oder die Reihenfolge von Wertpaaren innerhalb eines \acrshort{json}-Objekts verloren gehen.

\begin{figure}[b!]
    \begin{center}
        \includestandalone[width=0.8\textwidth]{flowchart_securitytest}
        \caption{Ablauf der Sicherheitstests}
    \end{center}
\end{figure}

\subsection{Überprüfung der Sicherheit}
\label{sec:method-sec}

Um die Sicherheit der Konversion zu überprüfen, wird eine modifizierte Vorgehensweise genutzt. Die auf Sicherheitslücken in \acrshort{xml}-Parsern abzielenden Test-Dokumente werden ebenfalls zunächst in das \acrshort{json}-Format und im Anschluss wieder zurück in \acrshort{xml} konvertiert. Allerdings ist dabei ausschließlich von Belang, dass keine der in Abschnitt~\ref{sec:xmlattacks} beschriebenen Angriffe ausgelöst wird -- die Korrektheit der Konversion wird hierbei nicht überprüft.

\begin{samepage}
Stattdessen wird ein Sicherheitstest als Fehlschlag gewerten, wenn eine oder mehrere der folgenden Bedingungen zutrifft:
\begin{enumerate}
    \item{} Der Konversionsvorgang überschreitet zuvor festgelege Obergrenzen für die Allozierung von Arbeitsspeicher, verbraucht zuviel CPU-Zeit oder dauert zu lange.\label{itm:cond-dos}
    \item{} Während der Konversion sendet der Konverter Anfragen an einen Webserver.\label{itm:cond-ssrf}
    \item{} Die vom Konverter ausgegebenen \acrshort{json}- oder \acrshort{xml}-Dateien enthalten eine bestimmte Zeichenkette.\label{itm:cond-fsa}
\end{enumerate}
\end{samepage}

Ist der eingesetzte \acrshort{xml}-Parser verwundbar gegenüber Angriffen aus dem Bereich \acrfull{dos}, führt das Parsen der Testdateien zu einem erhöhten Verbrauch an Systemressourcen. Die Obergrenzen für Arbeitsspeicherbelegung, \acrshort{cpu}-Auslastung und Zeitdauer sind so angelegt, dass ein erfolgreicher Angriff zwangsläufig gegen die erste Bedingung verstößt und damit zum Nichtbestehen des Tests führt.

Um zu Prüfen, ob \acrfull{ssrf} möglich ist, wird ein \acrshort{http}-Server gestartet, der von den entsprechenden Testfällen referenziert wird. Ruft ein Konverter während des Parsings eine der URLs auf, wird die Anfrage vom Server aufgezeichnet. Sind bei dem Server bis zum Abschluss der Konversion keine Anfragen eingegangen (Bed.~\ref{itm:cond-ssrf}), ist der Konverter offenbar nicht anfällig für solche Angriffe.

Die Testfälle für die Suche nach Schwachstellen der Kategorie \acrlong{fsa} verweisen auf Dateien, die eine bestimmte Zeichenkette enthalten. Ist diese Zeichenkette in den vom Konverter ausgegebenen \acrshort{json}- oder \acrshort{xml}-Daten enthalten, wurde vom Parser auf das Dateisystem zugegriffen und der Konverter ist verwundbar.

Alle eingesetzen Sicherheits-Testfälle basieren auf den \acrshort{xml}-Testdokumenten aus dem von Christoph Späth betreuten \enquote{DTD-Attacks}-Repository des Lehrstuhls für Netz- und Datensicherheit der Ruhr-Universität Bochum~\cite{dtdattacksrepo}.

\subsection{Technische Umsetzung}

% SLOC calculated via:
%     git ls-files | tr '\n' '\0' | wc -l --files0-from=- | grep -E ".*py$" | awk '{print $1}' | paste -sd+ | bc

Zur einfacheren Durchführung der Testreihen wurde ein Evaluations-Framework auf Basis von Python 3.5 implementiert, das ca. 2200 SLOC umfasst. Der Testprozess wird dabei weitestgehend automatisiert.

Um die \acrlong{c14n} durchzuführen, nutzt das Framework die~\mintinline{python}{write_c14n()}-Methode aus der \emph{lxml}-Bibliothek\footnote{http://lxml.de/}. Die Kanonisierung erfolgt nicht-exklusiv (vgl. Abschn.~\ref{sec:excc14n}) und ohne Kommentare -- ausgenommen ist dabei ein Testfall, der die Unterstützung für Kommentare überprüft.

Für das \acrshort{xml}-Parsing selbst kommt \emph{defusedxml}\footnote{https://pypi.python.org/pypi/defusedxml} zum Einsatz, das die gängigen Python-APIs zum Parsen von \acrshort{xml}-Dokumenten einem Security-Hardening unterzieht und daher nicht anfällig gegenüber fast allen Angriffsvektoren ist~\cite[Abschn. 9.5]{spaeth2016sok}.

Zur Prüfung und Formatierung der \acrshort{json}-Daten wird das Python-Modul \emph{demjson}\footnote{http://deron.meranda.us/python/demjson/} eingesetzt, das über umfangreiche Möglichkeiten zum Finden von Verstößen gegen die \acrshort{json} (einen sogenannten \emph{Linter}) verfügt. Zusätzlich zum \texttt{strict}-Modus wurden einige besondere Einstellungen vorgenommen: Byte-Order-Marks wurden in Übereinstimmung mit der \acrshort{json}-Spezifikation explizit verboten. Null-Bytes und andere von der Spezifikation nicht verbotene, aber möglicherweise Kompatibilitätsprobleme verursachende Zeichen wurden erlaubt, da es sich hierbei um korrektes \acrshort{json} handelt.

Für die Sicherheitstests wird ein eigener Prozess gestartet. Da als Host-System Linux zum Einsatz kommt, wurde die Begrenzung der Systemressourcen mithilfe des \texttt{setrlimit()}-Syscalls umgesetzt. Werden die so festgelegten Obergrenzen für die zur Verfügung stehende CPU-Zeit und Größe des adressierbaren Speichers des Prozesses überschritten, wird der Prozess vom Kernel durch das Signale \texttt{SIGXCPU} bzw. \texttt{SIGKILL} oder \texttt{SIGSEGV} beendet. Das Framework prüft nach Beendigung des Prozesses, ob der Exitcode auf die Terminierung durch ein Signal hinweist oder ob der Prozess normal beendet wurde.


\chapter{Ergebnisse} \label{chap:results}

Es wurden insgesamt 101 Testfälle verwendet und 11 verschiedene Konverter überprüft. Allerdings konnte keiner der Konverter alle Anforderungen aus Abschnitt~\ref{sec:criteria} erfüllen.

Im Folgenden werden die Ergebnisse der verschiedenenen Konverter vorgestellt.

\begin{figure}[b!]
    \label{tbl:results-basic}
    \includestandalone[width=\textwidth]{resulttable-basic}
    \captionof{table}{Konversions-Testergebnisse bezüglich verschiedener Konversionprobleme.}
\end{figure}

\begin{figure}[H]
    \label{tbl:results-chars}
    \includestandalone[width=\textwidth]{resulttable-chars}
    \captionof{table}{Ergebnisse der Tests bezüglich Unterstützung der von der \acrshort{xml}-Spezifikation erlaubten Zeichen.}

    \vspace*{\floatsep}

    \label{tbl:results-complex}
    \includestandalone[width=\textwidth]{resulttable-complex}
    \captionof{table}{Testergebnisse der Konverter bei komplexen Dokumenten.}
\end{figure}

\begin{figure}[t!]
    \label{tbl:results-sec}
    \includestandalone[width=\textwidth]{resulttable-sec}
    \captionof{table}{Ergebnisse der Tests auf Sicherheitlücken}
\end{figure}
\section{Cobra vs Mongoose}
\label{sec:cobravsmongoose}

Die Reihenfolge der Elemente sowie Whitespace werden von \emph{Cobra vs Mongoose} verworfen. Das Auftreten vom Mixed Content, \glspl{pi} und Kommentaren im Urspungsdokument führt zu Fehlern bei der Rückkonvertierung von \acrshort{json} zu \acrshort{xml}.

\section{GreenCape \acrshort{xml} Converter}
\label{sec:greencapexml}

Bei der Umwandlung von \acrshort{json} zu \acrshort{xml}-Daten fügt der Konverter automatisch Zeilenumbrüche hinter allen Elementen und Einrückungen mit einer Breite von vier Leerzeichen ein. Dies führte dazu, dass der Konverter keinen der Tests bestand. Um die Überprüfung der anderen, davon unabhängigen Aspekte des Konverters gewährleisten zu können, musste dieses Verhalten durch einen Patch entfernt werden (siehe Anhang~\ref{appx:greencapexml}).

Die Konversion der umfangreichen \acrshort{odf}-Spezifikation in Form einer \texttt{*.fodt}-Datei mithilfe des \emph{GreenCape \acrshort{xml} Converters} schug fehl. Der Versuch wurde abgebrochen, nachdem der PHP-Prozess seit rund 3 Stunden bei 100\% CPU-Last eingefroren war. Eine Analyse mithilfe des Tools \mintinline{shell}{strace} zeigte, dass sich der PHP-Interpreter in einer Endlosschleife aus aufeinanderfolgenden \mintinline{c}{mmap()}- und \mintinline{c}{mummap()}-Syscalls befand (siehe Abb.~\ref{fig:greencapeloop}).

\begin{figure}[h!]
    \inputminted{shell-session}{greencapexml-strace.txt}
    \captionof{figure}{Eine Endlosschleife im \emph{GreenCape \acrshort{xml} Converter} musste mittels \texttt{SIGTERM} unterbrochen werden.}
    \label{fig:greencapeloop}
\end{figure}

\section{Json-lib}
\label{sec:jsonlib}

Bei Kommentare oder \glspl{pi} innerhalb des Wurzelelement des \acrshort{xml}-Dokuments stürzt der Konverter ab. \glspl{pi} außerhalb des Wurzelelements werden von \emph{Json-lib} ignoriert. Bei der Konversion geht zudem der Tagname des Wurzelelements verloren. Tritt Mixed Content auf, werden bei der Rückkonversion alle Text-Knoten zusammengefasst.

Mehrere aufeinanderfolgende Elemente selben Namens werden von \emph{Json-lib} in ein \acrshort{json}-Array konvertiert. Dabei wird jedoch lediglich der Inhalt der Elemente übernommen, während der Tagname verloren geht.

Enthält das Wurzelelement eine \acrshort{xml}-Dokument ledglich Zeichendaten, so werden diese bei einem Round-Trip zum Inhalt eines Kindelements des Wurzelelements -- in diesem Fall fügt \emph{Json-lib} also eine Elementebene hinzu, die im Ursprungsdokument nicht existierte. Eine genauere Analyse der ausgegebene Daten zeigte, das dieses Verhalten auch der Grund für das Scheitern der CDATA-Testfälle war. CDATA-Sektionen werden von \emph{Json-lib} sehr wohl unterstützt.

Leider war es nicht möglich, \emph{Json-lib} ohne besondere Konfiguration in einem Prozess mit begrenztem virtuellen Adressraum zu verwenden, da die \acrfull{jvm} in diesem Fall nicht gestartet werden kann~\cite{jvmmemlimit}. Um dieses Problem zu umgehen, muss beim Start des Java-Prozesses die maximale Größe des Heap-Adressraums sowie die Größe des für Zeiger auf Metadaten von Java-Klassen zur Verfügung stehenden Adressraums angegeben werden.

Eine sinnvolle Aussage über die Anfälligkeit gegenüber Angriffen aus dem Bereich \acrlong{dos} ist unter diesen Umständen jedoch nicht möglich, sodass die entsprechenden Tests für \emph{Json-lib} manuell wiederholt werden mussten.

\todo[inline]{TODO: Ergebnisse ergänzen}
%TODO: Results?

Ein Blick in den Quellcode der Klasse \texttt{net.sf.json.XMLSerializer} zeigt, dass Json-lib den \acrshort{xml}-Parser XOM einsetzt. Dieser ist anfällig für \acrshort{xxe}-Angriffe mittels General Entites bzw. Parameter Entities, die sowohl für \acrlong{fsa} als auch für \acrlong{ssrf} genutzt werden können.

Ebenso erlaubt der Parser die Einbettung lokaler Dateien mittels Parameter Entities in DTDs, sowohl in der ursprünglichen Version des Angriffs~\cite[S.~10]{morgan2014xml}, als auch in einer modifizierten Variante, die 2016 von Sicherheitsforschern der Ruhr-Universität Bochum vorgestellt wurde.~\cite[Abschn.~5.2]{spaeth2016sok}.

\section{\acrshort{jsonml}}
\label{sec:jsonml}

Das Konversionverfahren \acrshort{jsonml} ist vergleichsweise vollständig. Lediglich die in Ursprungsdokumenten enthaltenen \glspl{pi} werden vom Konverter ignoriert und nicht in die \acrshort{json}-Ausgabe übernommen.

\acrshort{jsonml} ist von allen getesten Konvertern als einziger in der Lage, alle Test-Dokumente verlustlos zu konvertieren, solange sie keine \glspl{pi} enhielten.

\section{JXON}
\label{sec:jxon}

JXON fehlt die Unterstützung von Whitespace und Mixed Content. Zudem geht die Reihenfolge der Elemente verloren. \acrlongpl{pi} werden bei der Konversion ignoriert.

\section{org.json.XML}
\label{sec:orgjsonxml}

Als Java-Package ist \texttt{org.json.XML} ist die Untersuchung von \acrshort{dos}-Angriffen wegen der Beschränkungen der \acrlong{jvm} ebenso problematisch wie bei \emph{Json-lib}~(siehe Abschn.~\ref{sec:jsonlib}). Auch bei org.json.\acrshort{xml} musste die Verwundbarkeit gegenüber solchen Angriffe daher manuell verifiziert werden.

Attribute gehen bei der Konvertierung von \acrshort{xml}-Daten in \acrshort{json} zwar nicht verloren -- bei der Rückkonvertierung kann der Konverter jedoch nicht mehr erkennen, dass es sich um Attribute handelt und interpretiert diese stattdessen als Elemente.

\acrshort{xml}-Namespace-Prefixe für Tag-Namen werden zwar grundsätzlich unterstützt, Namensräume können aber aufgrund der fehlerhaften Behandlung von Attributen dennoch nicht genutzt werden.
 Mixed Content wird nicht unterstützt -- enthält ein Element neben Kindelement auch Text, wird dieser verworfen.

Ebenfalls gehen bei der Konversion \glspl{pi} und die Reihenfolge der Dokumenteninhalte verloren.

Zudem werden Zahlenwerte sowie die Zeichenketten \mintinline{json}{true} und \mintinline{json}{false} in die nativen Java-Datentypen konvertiert, wobei informationsverlust auftreten kann. Die Zeichenkette \texttt{1e-324} wird beispielweise bei der Übersetzung zu \acrshort{json} als Zahl interpretiert und erscheint daher in der Ausgabe gerundet als \texttt{0}.

\section{Pesterfish}
\label{sec:pesterfish}

Bei der Konversion gehen die Namen der Namesprace-Prefixe verloren und werden durch generische Bezeichnungen (\texttt{ns0}, \texttt{ns1}, \dots{}) ersetzt. Dies werden auch dann verwendet, wenn im Ursprungsdokument keinerlei Namensraum-Prefixe verwendet wurden, sondern ein eigener Default-Namespace genutzt wurde. Grund dafür ist die \texttt{ElementTree}-\acrshort{api}, die Namespace-Prefixes beim Parsen von \acrshort{xml} automatisch zur vollen URI expandiert und den ursprünglichen Prefixnamen verwirft~\cite[Abschn.~20.5.1.7]{pythonetreexmlns}.
Zudem gehen bei der Konvertierung auch \glspl{pi} verloren.

Bei den Sicherheitstests zeigte sich, dass \emph{Pesterfish} bei Nutzung der Vorgabeeinstellungen verwundbar für die \acrshort{dos}-Angriffe \emph{Billion Laughs} und \emph{Quadratic Blowup Attacks ist.}. Da die Bibliothek jedoch die Verwendung einer eigenen \texttt{ElementTree}-Implementierung erlaubt, kann diesem Problem durch den Einsatz eines sicheren Ersatzes -- beispielsweise aus der \emph{defusedxml}-Bibliothek -- entgegengewirkt werden.

\section{x2js}
\label{sec:x2js}

Die Beibehaltung der Dokumentreihenfolge ist beim Einsatz von \emph{x2js} nicht gegeben. Bei Mixed Content wird der gesamte Character Content eines Elements zu einem einzelnen Textknoten zusammengefasst. Auch \glspl{pi} werden nicht unterstützt -- treten diese innerhalb des Wurzelelements auf, wird anstelle der \gls{pi} ein leeres Element namens \enquote{undefined} eingefügt.

Whitespace bleibt zwar grundsätzlich erhalten, Einrückungen werden jedoch wie anderer Mixed Content auch zusammengefasst, sodass bei Dokumenten mit Einrückungen nach der Konversion jedes Element mit zuvor eingerücktem Inhalt stattdessen einen einzelnen, nur aus Whitespace bestehenden Textknoten enthält.

\section{xmljson}
\label{sec:xmljson}

Im Gegensatz zu anderen Konvertern verfügt das Python-Package xmljson über keine Möglichkeit, einen \acrshort{xml}-Daten enthaltenden String direkt zu konvertieren, sondern akzeptiert lediglich bereits geparste \texttt{ElementTree}-Objekte. Da der Nutzer selbst für das Parsen des \acrshort{xml}-Dokuments zuständig ist und es im Unterschied zu Pesterfish (Vgl.~Abschn.~\ref{sec:pesterfish}) auch keine Voreinstellung gibt, wird eine Sicherheitsanalyse dieses Konverters hinfällig. Zum Durchführen der Tests wurde \texttt{defusedxml.lxml} verwendet.

Der \emph{xmljson}-Konverter kann mehrere verschiedene Konvertierungskonventionen nutzen (vgl.~Abschn.~\ref{sec:converters}), die unabhängig voneinander getestet wurden.

Durch den Einsatz der \texttt{ElementTree}-\acrshort{api} hat \emph{xmljson} keinen Zugriff auf die im \acrshort{xml}-Dokument verwendeten Prefixnamen, sodass diese wie beim \emph{Pesterfish}-Konverter durch generische Namen ersetztt werden~(vgl.~Abschn.~\ref{sec:pesterfish}).
\glspl{pi} werden von allen Konventionen ignoriert.

Lediglich die \emph{Cobra}- und \emph{Yahoo}-Konventionen (Abschn.~\ref{sec:xmljson-cobra} und~\ref{sec:xmljson-yahoo}) nutzen keine Typinferenz. Alle anderen Konverter wandeln die Zeichenketten \texttt{true} und \texttt{false} in boolsche Werte um. Auch Zahlenwerte werden als numerische Datentypen interpretiert, wodurch beispielsweise Rundungsfehler auftreten können oder Formatierung verloren geht (z.B. \mintinline{json}{1e+39} anstatt \mintinline{json}{"1E39"}). Sehr großen Zahlen werden als \mintinline{js}{Infinity}-Literal dargestellt, der zwar valides JavaScript wäre, von der \acrshort{json}-Spezifikation jedoch nicht erlaubt ist.

\subsection{Abdera}
\label{sec:xmljson-abdera}

Attribute werden bei der \acrshort{json}-Konvertierung mithilfe der \emph{Abdera}-Konvention zwar in die Ausgabe übernommen, allerdings ist \emph{xmljson} bei der Rückkonvertierung nicht in der Lage, diese von normalen Elementen zu unterscheiden. Daher befinden sich Attribute nach der Rückkonvertierung nicht mehr an der ursprünglichen Stelle im Dokument, sondern in einem \mintinline{xml}{<attributes>}-Element, das als Kindelement des Ursprungselements eingefügt wird. Ein zusätzliches \mintinline{xml}{<children>}-Element enthält die ursprünglichen Kindelemente des Elements.

Andererseits werden teilweise Elemente und Textknoten zu Attributwerten umgedeutet. So wird aus \mintinline{xml}{<a><b>hello</b></a>} durch einen Round-Trip \mintinline{xml}{<a b="hello" />}

Aufgrund der Probleme des Konverters, anhand der \acrshort{json}-Daten zwischen Elementen und Attributen zu unterscheiden und diese später wieder korrekt rekonstruieren zu können, schlagen auch Tests fehl, die der Konverter eigentlich bestehen könnte. So nutzt der Konverter bei mehreren Kindelementen \acrshort{json}-Arrays, d.h. die Elementreihenfolge ist auch nach der Konversion noch ersichtlich.
Auch Tag-Name und Attribute des Wurzelelements gehen eigentlich nicht verloren.

Mixed Content wird nicht unterstützt, es ist lediglich der erste Textknoten im Dokument auffindbar.
Führender oder anhängender Whitespace wird bei der Konversion verworfen.

\subsection{Badgerfish}
\label{sec:xmljson-badgerfish}

Die Ergebnisse stimmen im Wesentlichen mit denen  denen des \emph{Cobra-vs-Mongoose}-Konverters (vgl.~Abschn.~\ref{sec:cobravsmongoose}) überein, der ebenfalls auf die sogenannte Badgerfish-Konvention zur Darstellung von \acrshort{xml}-Strukturen in \acrshort{json} setzt. Nicht unterstütze Features wie Mixed Content, \glspl{pi} oder Kommentare führten bei \emph{xmljson} jedoch nicht zu eineme Absturz des Programms. Zudem kommt es zu Problemen durch den Einsatz von Typinferenz und dem Verlust der Prefixnamen von Namespaces.

\subsection{Cobra}
\label{sec:xmljson-cobra}

Bei \emph{Cobra} handelt es sich um eine modifizierte Variante von \emph{Abdera}, die sich hauptsächtlich darin unterscheiden, welche Schlüssel in der \acrshort{json}-Objekte-Repräsentation eines \acrshort{xml}-Dokuments optional sind. Daher hat auch \emph{Cobra} große Probleme mit der Rückkonvertierung zu \acrshort{xml} und der Unterscheidung zwischen Elementen und Attributen.

Ein weiterer Unterschied ist, dass bei Cobra keine Datentypen erraten werden, sondern alles als String behandelt wird.

\subsection{GData}
\label{sec:xmljson-gdata}

Durch die Nutzung von ungeordneten \acrshort{json}-Objekten geht bei dem Einsatz der \emph{GData}-Konvention die Reihenfolge der im \acrshort{xml}-Dokument enthaltenen Elemente verloren. Bei Mixed Content wird lediglich der erste Textknoten übernommen, alle weiteren werden verworfen.

Bei der Konversion kann es zu Informationsverlust durch Typinferenz kommen.

\subsection{Parker}
\label{sec:xmljson-parker}

In der Standardeinstellung verwirft diese Konvention das Wurzelelement -- dieses Verhalten ist jedoch über einen Parameter abschaltbar.

\todo[inline]{TODO: Hier fehlt noch was }
% TODO: Add content

\subsection{Yahoo}
\label{sec:xmljson-yahoo}

Die \emph{Yahoo} hat ebenfalls Probleme bei der Unterscheidung zwischen Elementen und Attributen, fügt aber im Gegensatz zu \emph{Abdera} und \emph{Cobra} nicht neue, im Originaldokument nicht existierende Elemente in das Dokument ein.

Die Elementreihenfolge geht bei der Konvertierung verloren.

Im \acrshort{xml}-Dokument enthaltener Character Content wird in allen getesteten Fällen zu \acrshort{json}-Strings konvertiert, eine Typumwandlung findet nicht statt.

\chapter{Weiterentwicklung eines Konversionsverfahrens}
\label{chap:jsonml}

Mit insgesamt 89 von 100 bestandenen Testfällen erfüllt die \acrfull{jsonml} von allen Konversionverfahren die meisten Kriterien im Test.

In diesem Kapitel wird daher zunächst ein Überblick über die von \acrshort{jsonml} eingesetzte Syntax gegeben.
Darauf aufbauend werden dann die notwendigen Modifikationen am \acrshort{jsonml}-Verfahren beschrieben, die im Zuge dieser Arbeit entwickelt wurden, um das Ziel eines vollständig verlustlose Konversionsverfahrens zu erreichen. Im Anschluss daran werden die Ergebnisse einer Analyse der so weiterentwickelten \acrshort{jsonml}.

\acrfull{jsonml}
\section{Syntax}

Im Gegensatz zu anderen Konvertern nutzt \acrshort{jsonml} ungeordnete \acrshort{json}-Objekte ausschließlich für Attributlisten~\cite{jsonmlsyntax}. Die Baumstruktur eines \acrshort{xml}-Dokuments wird mittels \acrshort{json}-Arrays dargestellt, wobei ein Array immer genau ein Element repräsentiert. Textknoten bzw. CDATA-Sektionen werden zu einfachen \acrshort{json}-Strings umgewandelt.

\begin{figure}[H]
    \begin{definition}[{Formale Syntax der \acrfull{jsonml}}]
        \label{def:jsonml}
        Sowohl \synt{tag-name} als auch \synt{attribute-name} sind \acrshort{json}-Werte vom Typ String. Die Whitespace-Regeln sind identisch wie bei \acrshort{json} (Vgl. Definition~\ref{def:json}).
        \begin{grammar}
            <element> ::= \[[
        \begin{stack}
            `[' <tag-name>
                \begin{stack}
                    `,' <attributes>\\
                \end{stack}
                \begin{stack}
                    `,' \begin{rep}
                            <element>\\
                            `,'
                        \end{rep}\\
                \end{stack}
            `]'\\
            \tok{\color{black} \acrshort{json}-String}
        \end{stack}
    \]]

<attributes> ::= \[[
    `\{' \begin{stack}
            \begin{rep}
                <attribute-name> `:' <attribute-value>\\
                `,'
            \end{rep}\\
        \end{stack} `\}'
    \]]

<attribute-value> ::= \[[
        \begin{stack}
            \tok{\color{black} \acrshort{json}-String}\\
            \tok{\color{black} \acrshort{json}-Number}\\
            `true'\\
            `false'\\
            `null'
        \end{stack}
    \]]

        \end{grammar}
    \end{definition}
\end{figure}

\begin{figure}[h]
    \begin{example}[{\acrshort{jsonml}-Dokument}]~
    \inputminted{json}{xmltree.json}
        \captionof{figure}{Die \acrshort{jsonml}-Repräsentation des \acrshort{xml}-Dokuments aus Beispiel~\ref{ex:xmldoc}.}
    \label{fig:xmltreejsonml}
    \end{example}
\end{figure}

\acrshort{jsonml} sieht keine gesonderte Verarbeitung von Namespace-Deklarationen oder mit Namespace-Prefixes versehene Tag-Namen vor, sondern behandelt diese wie normale Attribute bzw. wie einen Teil des Tag-Namens.

\acrshort{jsonml} stellt die \acrshort{xml}-Inhalte recht effizient dar: Die \acrshort{json}-Repräsentation eines umfangreichen Office-Dokuments im \acrshort{fodt}-Format benötigt rund $6{,}6\%$ weniger Speicherplatz als die äquivalente Darstellung durch kanonisches \acrshort{xml}.

Im Vergleich zu anderen Konvertern ist der Overhead bei \acrshort{jsonml} deutlich geringer. So waren die vom Pesterfish-Konverter ausgegebenen Daten trotz Informationsverlust auch nach Entfernung von optionalem Whitespace und unnötiger Quotierung von Unicode-Zeichen mehr als dreieinhalb Mal so groß wie bei \acrshort{jsonml} (vgl. Abb.~\ref{fig:sizecomparison}).

\begin{center}
    \begin{threeparttable}
        \caption{Größenvergleich von ggü. \acrshort{xml} und Pesterfish anhand der Spezifikation des \acrlong{odf} im \acrshort{fodt}-Format.}
        \label{fig:sizecomparison}
        \begin{tabular}{lrrr}
            \toprule
            \rowcolor{white}     & \multicolumn{1}{c}{\fontfamily{rubflama}\selectfont\textbf{Größe (in bytes)}} & \multicolumn{2}{c}{\fontfamily{rubflama}\selectfont\textbf{Verhältnis zu \acrshort{xml} (in \%)}}\\
                                 &                  & \multicolumn{1}{c}{\fontfamily{rubflama}\selectfont\textbf{Größe}}   & \multicolumn{1}{c}{\fontfamily{rubflama}\selectfont\textbf{Veränderung}}\\
            \midrule
            \rowcolor{rubgray!50} Kanonisches \acrshort{xml} &  5787196 & $100{,}0$ &        $0$ \\
                                  \acrshort{jsonml}\tnote{a} &  5405329 &  $93{,}4$ &   $-6{,}6$ \\
            \rowcolor{rubgray!50} Pesterfish\tnote{a}        & 15061634 & $260{,}3$ & $+160{,}3$ \\
                                  Pesterfish\tnote{b}        & 14480612 & $250{,}2$ & $+150{,}2$ \\
            \bottomrule
        \end{tabular}
        \begin{tablenotes}
            \item[a] \acrshort{json} unverändert
            \item[b] Optionaler \acrshort{json}-Whitespace entfernt
        \end{tablenotes}
    \end{threeparttable}
\end{center}

\section{Unterstützung von \acrlongpl{pi}}

Probleme hatte der Konverter jedoch mit der Umwandlung von \acrfullpl{pi}. Diese werdem bei der Umwandlung in \acrshort{json} vollständig ignoriert. Stephen McKamey, der Entwickler von \acrshort{jsonml}, begründet damit, das es keine sinnvolle Entsprechung von \acrshortpl{pi} in \acrshort{json} gäbe~\cite{mckamey2006xml}.

Zwar bietet \acrshort{json} tatsächlich keinen vergleichbaren Mechanismus, eine Unterstützung von \glspl{pi} kann für bestimmte Einsatzzwecke aber sinnvoll sein, beispielsweise wenn sonst die Verknüpfung mit \acrshort{xml}-Stylesheets oder Formatierungsinformationen in DocBook-Dateien verloren gehen könnten. Im Rahmen der vorliegenden Arbeit wurde die \acrshort{jsonml}-Syntax daher um Unterstützung von \acrlongpl{pi} ergänzt.

\glspl{pi} bestehen aus einem \emph{Ziel} und \emph{Daten} (Vgl. Abschn. \ref{sec:xmlbasics}), bilden also das 2-Tupel $P \coloneqq \langle target, data \rangle$. Der Datenteil kann dabei auch leer sein.

Das Ziel muss ein gültiger Name im Sinne der \acrshort{xml}-Spezifikation sein~\cite[{Regel~[17]}]{maler2008xml}. Das heißt, dass der Name einer \gls{pi}\ ebenso wie auch der Tag-Name vom Elementen~\cite[{Regel~[40]}]{maler2008xml} mit einem sog. \texttt{NameStartChar} beginnen muss. Dadurch wird ausgeschlossen, dass Tag-Namen mit bestimmten Zeichen beginnen -- darunter auch das Fragezeichen, da dies dazu führen würde, dass Start-Tags mit \glspl{pi} verwechselt werden könnten. Insbesondere in \acrshort{sgml} -- zu dem \acrshort{xml} vollständig kompatibel sein soll -- wären solche Tags nicht mehr von \glspl{pi} zu unterscheiden, da laut \acrshort{sgml}-Spezifikation lediglich ein einfaches Größerzeichen anstatt der Kombination aus Fragezeichen und Größerzeiche (\texttt{?>}) zum Schließen der \gls{pi} ausreicht.

Dadurch wird es möglich, \glspl{pi} in \acrshort{jsonml} eindeutig in Form eines \acrshort{json}-Arrays \mintinline{json}{["?target", "data"]} darzustellen~(vgl. Definition \ref{def:jsonmlpi}), das dem 2-Tupel $P$ (s.o.) entspricht. Die Repräsentation von \glspl{pi} ähnelt damit der eines Elementknotens, der einen einzelnen Textknoten (\emph{Character Data}) enthält. Eine Verwechslung ist jedoch durch das dem Zielnamen vorangestellte Fragezeichen ausgeschlossen -- ein Tagname darfnicht mit einem Fragezeichen beginnen, wodurch die Kategorisierung als \gls{pi} eindeutig ist.  \begin{figure}[h]
    \begin{definition}[{Formale Syntax der \acrfull{jsonml} mit \emph{\glspl{pi}}}]
        \label{def:jsonmlpi}

        Die um Unterstützung von \emph{\glspl{pi}} erweitere Syntax ist mit Ausname der Produktionsregeln für \synt{element} identisch zu der Syntax aus Definition~\ref{def:jsonml}.
        \synt{tag-name}, \synt{pi-target} und \synt{pi-data} sind \acrshort{json}-Werte vom Typ String.

        \begin{grammar}
            <element> ::= \[[
        \begin{stack}
            `[' \begin{stack}
                    <tag-name>
                    \begin{stack}
                        `,' <attributes>\\
                    \end{stack}
                    \begin{stack}
                        `,' \begin{rep}
                                <element>\\
                                `,'
                            \end{rep}\\
                    \end{stack}\\
                    `?' <pi-target> `,' <pi-value>
                \end{stack} `]'\\
            \tok{\color{black} \acrshort{json}-String}
        \end{stack}
    \]]

        \end{grammar}

        Enthält das Dokument \emph{\glspl{pi}} auf Dokument-Ebene (d.h. als Top-Level-Konstrukt), dann ist das \acrshort{jsonml}-Wurzelelement ein \synt{element} mit einem leeren String als \synt{tag-name}, das die Child-Nodes des Dokuments (d.h. \emph{\glspl{pi}} auf Dokumentebene und das Wurzelelement des Dokuments) als Unterelemente enthält.
    \end{definition}
\end{figure}

\section{Überprüfung der Änderungen}

Die syntaktischen Änderungen aus Definition~\ref{def:jsonmlpi} wurden in die JavaScript-Referenz\-implementierung von Stephen McKamey eingearbeitet. Entsprechende \emph{Unittests} zur Sicherstellung der korrekten Umwandlung von \glspl{pi} wurden ebenfalls hinzugefügt.

\begin{figure}[h!]

    \begin{example}[{\acrshort{jsonml}-Dokument mit \glspl{pi}}]
        Die \acrshort{jsonml}-Repräsentation des \acrshort{xml}-Dokuments aus Beispiel \ref{ex:xmltree} kann nun die \gls{pi} darstellen -- auch solche, die sich außerhalb des Wurzelelements befinden.
        \begin{minted}[autogobble]{json}
            ["", "\n",
                [ "?xml-stylesheet", "href=\"style.css\"" ],"\n","\n",
                ["albums", "\n  ",
                    ["album", {"catno": "ARGO LP-628"}, "\n    ",
                        ["artist", "Ahmad Jamal Trio"], "\n    ",
                        ["title", "At The Pershing"], "\n    ",
                        ["recording", "Recorded ",
                            ["date", "January 16, 1958"], "."
                        ], "\n  "
                    ], "\n"
                ]
            ]
        \end{minted}
    \end{example}
\end{figure}

Bei einer erneuten Überprüfung des \acrshort{jsonml}-Konverters unter Berücksichtigung der o. g. Änderungen wurde deren Korrektheit bestätigt: Alle Testdokumente, auch die zuvor fehlgeschlagenen, lassen sich nun verlustlos von \acrshort{xml} nach \acrshort{json} und wieder zurück konvertieren.

Alle Änderungen wurdem dem \acrshort{jsonml}-Projekt zur Verfügung\footnote{Vgl.~\url{https://github.com/mckamey/jsonml/pull/14}} gestellt (siehe Anhang~\ref{appx:jsonmlpi}).


\chapter{Conclusion} \label{chap:conclusion}
Lorem ipsum dolor sit amet consectetuer parturient ac pulvinar magna porttitor. Accumsan vel ac eros laoreet Nulla leo Nulla vel Pellentesque Quisque. Adipiscing penatibus Phasellus egestas leo id neque nec quis est orci. Porta tellus ligula ut ridiculus eros eget ut Vivamus dictum nulla. Dui wisi enim vitae nulla Fusce Curabitur congue consectetuer urna Quisque. Felis Vestibulum Quisque sed Vestibulum et malesuada ac id tristique vitae. Aliquam Suspendisse mattis et libero et tincidunt quis tellus eget consectetuer. Libero Morbi cursus augue eget dapibus tincidunt nunc parturient id arcu. Donec sapien enim Aenean convallis Donec elit tincidunt dolor vitae tellus. Ac consectetuer at tortor malesuada ac dui ligula habitant habitasse congue. 


%% include appendix

\appendix

% \chapter{Acronyms}
\printabbreviations[title={Abkürzungsverzeichnis}]

%\glossarystyle{index}  % chose style here
%\renewcommand*{\glstreenamefmt}[1]{#1}
%\printglossary[type=main,title={Register}]

\pagestyle{scrplain} % turn off headers and footers
%% generate list of figures, optional, remove it if you do not like it
\listoffigures

\KOMAoptions{open=any} % Plaziert Kapitel auch auf linken Seiten

%% generate list of tables, optional, remove it if you do not like it
\listoftables

%% generate list of algorithms, optional, remove it if you do not like it
\clearpage \phantomsection{}
\addcontentsline{toc}{chapter}{Liste der Beispiele}
\renewcommand{\listtheoremname}{Liste der Beispiele}
\listoftheorems[ignoreall,show={mdexample}]
%
%%% generate list of listings, optional, remove it if you do not like it
%\renewcommand*{\listoflistingscaption}{Liste der Auflistungen}
%\listoflistings{}

%% generate bibliography with bibtex, the bibfile here is "paper.bib"
\flushbottom
\setcounter{biburllcpenalty}{7000}
\setcounter{biburlucpenalty}{8000}
\setlength{\emergencystretch}{3em}
\printbibliography{}

\KOMAoptions{open=right} % Plaziert Kapitel wieder nur auf rechten Seiten

\chapter{Ausgabebeispiele der Konverter}
\label{appx:convexamples}
\begin{figure}[h!]
    \begin{tabular}[t]{cc}
\subfloat[Cobra vs Mongoose]{
    \begin{minipage}[t]{.475\linewidth}
        \inputminted[fontsize=\footnotesize]{json}{examples/cobravsmongoose.json}
    \end{minipage}
} &
\subfloat[GreenCape \acrshort{xml} Converter]{
    \begin{minipage}[t]{.475\linewidth}
    \inputminted[fontsize=\footnotesize]{json}{examples/greencapexml.json}
    \end{minipage}
} \\
\subfloat[Json-lib]{
    \begin{minipage}[t]{.475\linewidth}
    \inputminted[fontsize=\footnotesize]{json}{examples/jsonlib.json}
    \end{minipage}
} &
\subfloat[\acrshort{jsonml}]{
    \begin{minipage}[t]{.475\linewidth}
    \inputminted[fontsize=\footnotesize]{json}{examples/jsonml.json}
    \end{minipage}
}
\end{tabular}
\end{figure}
\begin{figure}[h!]\ContinuedFloat
    \begin{tabular}[t]{cc}
\subfloat[org.json.XML]{
    \begin{minipage}[t]{.475\linewidth}
    \inputminted[fontsize=\footnotesize]{json}{examples/orgjsonxml.json}
    \end{minipage}
} &
\subfloat[x2js (Fork)]{
    \begin{minipage}[t]{.475\linewidth}
    \inputminted[fontsize=\footnotesize]{json}{examples/x2js.json}
    \end{minipage}
}\\
\subfloat[JXON]{
    \begin{minipage}[t]{.475\linewidth}
    \inputminted[fontsize=\footnotesize]{json}{examples/jxon.json}
    \end{minipage}
} &
\subfloat[Json.NET]{
    \begin{minipage}[t]{.475\linewidth}
    \inputminted[fontsize=\footnotesize]{json}{examples/newtonsoftjson.json}
    \end{minipage}
}
\end{tabular}
\end{figure}
\begin{figure}[h!]\ContinuedFloat
    \begin{tabular}[t]{cc}
\multirow{2}{*}[1.5ex]{
\subfloat[Pesterfish]{
    \begin{minipage}[t]{.475\linewidth}
    \inputminted[fontsize=\footnotesize]{json}{examples/pesterfish.json}
    \end{minipage}
}
} &
\subfloat[xmljson (Abdera)]{
    \begin{minipage}[t]{.475\linewidth}
    \inputminted[fontsize=\footnotesize]{json}{examples/xmljson-abdera.json}
    \end{minipage}
}\\
    &
\subfloat[xmljson (Yahoo)]{
    \begin{minipage}[t]{.475\linewidth}
    \inputminted[fontsize=\footnotesize]{json}{examples/xmljson-yahoo.json}
    \end{minipage}
}
\end{tabular}
\end{figure}

\begin{figure}[h!]\ContinuedFloat
    \begin{tabular}[t]{cc}
        \begin{minipage}[t]{.475\linewidth}
\subfloat[xmljson (Badgerfish)]{
    \begin{minipage}{\linewidth}
    \inputminted[fontsize=\footnotesize]{json}{examples/xmljson-badgerfish.json}
    \end{minipage}
}\\
\subfloat[xmljson (GData)]{
    \begin{minipage}{\linewidth}
    \inputminted[fontsize=\footnotesize]{json}{examples/xmljson-gdata.json}
    \end{minipage}
}
\end{minipage}
\hfill{}
\begin{minipage}[t]{.475\linewidth}
\subfloat[xmljson (Cobra)]{
    \begin{minipage}{\linewidth}
    \inputminted[fontsize=\footnotesize]{json}{examples/xmljson-cobra.json}
    \end{minipage}
}\\
\subfloat[xmljson (Parker)]{
    \begin{minipage}{\linewidth}
    \inputminted[fontsize=\footnotesize]{json}{examples/xmljson-parker.json}
    \end{minipage}
}
\end{minipage}
\end{tabular}
\end{figure}
\begin{figure}[h!]\ContinuedFloat
\end{figure}

\chapter{Patches}
\label{appx:patches}

\section{Entfernung des Whitespace im GreenCape XML Konverter}
\label{appx:greencapexml}

\inputminted[breakautoindent=false,fontsize=\footnotesize]{udiff}{patches/greencapexml-noindent.patch}
\pagebreak{3}

\section{Unterstützung für \acrshortpl{pi} in \acrshort{jsonml}}
\label{appx:jsonmlpi}

\inputminted[breakautoindent=false,fontsize=\footnotesize]{udiff}{patches/jsonml-pi.patch}


\end{document}
