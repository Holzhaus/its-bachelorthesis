%%% DOCUMENT-INFO
%% AUTHOR		: Jan Holthuis [jan.holthuis@ruhr-uni-bochum.de]
%% VERSION		: 0.1

\documentclass[german]{rubthesis}
\usepackage[mode=buildnew]{standalone}
\usepackage{tabularx}
\usepackage{threeparttable}
\usepackage{colortbl}
\usepackage{keyval}
\usepackage[useregional]{datetime2}
\DTMsetup{showdow=true}

\usepackage[nounderscore,rounded]{syntax}
\usepackage{todonotes}
\usepackage{hyphenat}
\usepackage{subfig}
\usepackage{multirow}
\usepackage{icomma}

\usepackage{mathtools}
\DeclarePairedDelimiter\abs{\lvert}{\rvert}
\DeclarePairedDelimiter\ceil{\lceil}{\rceil}
\DeclarePairedDelimiter\floor{\lfloor}{\rfloor}

\renewcommand{\sdsize}{\scriptsize\color{rubgreen}}

\renewcommand{\syntleft}{\color{rubblue}$\langle$\normalfont\itshape}
\renewcommand{\litleft}{\color{black}"\bgroup\ulitleft}
\renewcommand{\litright}{\ulitright\egroup"}
\renewcommand{\sdlengths}{%
  \setlength{\sdstartspace}{1em}%
  \setlength{\sdendspace}{1em}%
  \setlength{\sdmidskip}{0.5em}%
  \setlength{\sdtokskip}{0.25em}%
  \setlength{\sdfinalskip}{0.5em plus 10000fil}%
  \setlength{\sdrulewidth}{0.2pt}%
  \setlength{\sdcirclediam}{12pt}%
  \setlength{\sdindent}{0pt}%
}

\hyphenation{Java}
\hyphenation{Ruby}

\addbibresource{literature.bib}

\renewcommand{\thesistype}{Bacherlorarbeit}
\renewcommand{\thesisadvisor}{B.~Sc.~Paul~Rösler}
\renewcommand{\thesiskeywords}{XML, JSON, Konversion, Umwandlung, Abbildung}

% Remove theorem type name from list of theorems
\makeatletter
\def\ll@mdexample{%
  \protect\numberline{\csname the\thmt@envname\endcsname}%
  \ifx\@empty\thmt@shortoptarg
    \thmt@thmname
  \else
    \thmt@shortoptarg
  \fi}
\def\l@thmt@mdexample{}
\makeatother

% Set date here
%\day=6 \month=6 \year=2012

% Set name and title
\author{Jan~Holthuis}
\newcommand{\matrikelnummer}{108\,009\.215\,809}
\title{\nohyphens{%
Bridging the Gap: Verlustfreie und sichere Umwandlung von XML-Datenstrukturen
ins JSON-Format
}}
\date{\today}

% Load acronyms in preamble
%\loadglsentries{glossary}
\loadglsentries{acronyms}

\makeatletter
\define@key{convtool}{url}{\def\conv@url{#1}}
\define@key{convtool}{license}{\def\conv@license{#1}}
\define@key{convtool}{author}{\def\conv@author{#1}}
\define@key{convtool}{version}{\def\conv@version{#1}}
\define@key{convtool}{language}{\def\conv@language{#1}}
\define@key{convtool}{version}{\def\conv@version{#1}}
\define@key{convtool}{versiondate}{\def\conv@versiondate{#1}}
\setkeys{convtool}{url={},license={},author={},language={},version={Unbekannt}, versiondate={}}%
\newcommand{\convtool}[2][]{%
    \begingroup%
    \setkeys{convtool}{#1}% Set new keys
    \begin{tabular}{p{3.6cm}l}
    \ifx\conv@author\@empty
    \else
        {\fontfamily{rubflama}\selectfont{}\textbf{Autor:}} & \conv@author\\
    \fi
    \ifx\conv@url\@empty
    \else
        {\fontfamily{rubflama}\selectfont{}\textbf{URL:}} & \url{\conv@url}\\
    \fi
    \ifx\conv@license\@empty
    \else
        {\fontfamily{rubflama}\selectfont{}\textbf{Lizenz:}} & \conv@license\\
    \fi
    \ifx\conv@language\@empty
    \else
        {\fontfamily{rubflama}\selectfont{}\textbf{Programmiersprache:}} & \conv@language\\
    \fi
    \ifx\conv@version\@empty
    \else
        {\fontfamily{rubflama}\selectfont{}\textbf{Gestete Version:}} & \conv@version
        \ifx\conv@versiondate\@empty
        \else
            ~(\DTMdate{\conv@versiondate})
        \fi\\
    \fi
    \end{tabular}
    \endgroup%
}
\makeatother


\begin{document}

%% switch to roman paginating for the acknowledgements, table of contents etc.
\pagenumbering{roman} % uncomment this if you like it

%% title page --- made out of expressions defined above
\frontpage{}

\begin{abstract}
Insert abstract here.
\end{abstract}

\pagestyle{scrplain} % switch off headers and footers
\section*{Declaration}
{\selectlanguage{ngerman}
I hereby declare that this submission is my own work and that,\ to the best of my
knowledge and belief,\ it contains no material previously published or written by another
person nor material which to a substantial extent has been accepted for the award of any
other degree or diploma of the university or other institute of higher learning,\ except
where due acknowledgment has been made in the text\@.}

\section*{Erkl\"{a}rung}
{\selectlanguage{ngerman}
Hiermit versichere ich,\ dass ich die vorliegende Arbeit selbstst\"{a}ndig verfasst und keine
anderen als die angegebenen Quellen und Hilfsmittel benutzt habe,\ dass alle Stellen
der Arbeit,\ die w\"{o}rtlich oder sinngem\"{a}\ss{} aus anderen Quellen \"{u}bernommen wurden,\ als
solche kenntlich gemacht sind und dass die Arbeit in gleicher oder \"{a}hnlicher Form noch
keiner Pr\"{u}fungsbeh\"{o}rde vorgelegt wurde\@.}

\vspace{2cm}
\noindent\begin{tabularx}{\textwidth}{lX}
\makebox[5.5cm]{\hrulefill} & \hrulefill\\
Ort, Datum & Unterschrift\\[8ex]% adds space between the two sets of signatures
\end{tabularx}

\cleardoublepage{}
\pagestyle{scrheadings} % reenable headers and footers

%% generate table of contents
\tableofcontents

\cleardoublepage{}


\pagenumbering{arabic} %switches to arabic numbers for the rest of the text
\setcounter{page}{1}

%%
%% include all your chapters as .tex files,
%% each file contains sections \section{name of section},
%% subsections \subsections{...} and so on...
%%

\chapter{Einleitung} \label{chap:intro}
Für den implementationsunabhängigen Austausch von zugleich menschen- als auch
maschinenlesbaren Daten hat sich die \gls{xml}
bewährt. In bestimmten Bereichen wie Web-\acrshortpl{api} hat die \gls{json}
das bewährte \acrshort{xml}-Format inzwischen jedoch überflügelt.

Dabei kann neben der spezifischen Situation auch die eingesetzte
Programmiersprache, die Unterstützung durch das zugrundeliegende Framework
oder die persönliche Präferenz des Entwicklers den Ausschlag geben, welches
Format ein Webservice oder eine Programmbibliothek unterstützt.

Zwecks Interoperabilität zwischen verschiedenen Teilen einer Anwendung kann es
daher notwendig werden, Daten von einem Format temporär in das jeweils andere
zu überführen und später wieder in das ursprüngliche Format zu bringen.

Dabei sollen Daten, die von der Anwendungslogik nicht verändert wurden, auch
nach der Rückübersetzung ins Ursprungsformat unverändert bleiben. Damit dies
gewährleistet ist, darf das zugrunde liegende Konversionverfahren keine
Informationen bei der Umwandlung verwerfen -- es muss also verlustlos % chktex 8
arbeiten.

Das konkrete Konversionsverfahren ist dabei abhängig vom jeweiligen
Ausgangsformat, d.~h.\ die Umwandlungsverfahren für die beiden Richtungen
\begin{enumerate}
    \item $\text{\acrshort{json}} \rightarrow \text{\acrshort{xml}} \rightarrow \text{\acrshort{json}}$ und
    \item $\text{\acrshort{xml}} \rightarrow \text{\acrshort{json}} \rightarrow \text{\acrshort{xml}}$
\end{enumerate}
haben jeweils eigene Anforderungen und sind getrennt voneinander zu betrachten.

Während Abbildung von beliebigen \acrshort{json}-Datenstrukturen in \acrshort{xml} zumindest bei
oberflächlicher Betrachtung trivial erscheint, ist dies beim verlustlosen
Transfer von \acrshort{xml}-Daten ins \acrshort{json}-Format keineswegs der Fall.

Daher sollen in dieser Bachelorarbeit verschiedene Verfahren analysiert
werden, die beliebige \acrshort{xml}-Daten in \acrshort{json} abbilden und aus der resultierenden
\acrshort{json}-Datenstruktur wieder \acrshort{xml}-Dokument erstellen können. Sollte keines
der analysierten Verfahren den vorher aufgestellten Kriterien für eine
zuverlässige und sichere Umwandlung genügen, wird ein eigener
Abbildungsalgorithmus entwickelt, bei dem dies der Fall ist.

\section{Motivation}
\label{sec:motivation}
Das Aufkommen des sogenannten \emph{Web 2.0} und die zunehmenden Vernetzung
durch das \gls{iot} ging mit einer erhöhten
Verfügbarkeit von öffentlichen Web-\acrshortpl{api} einher. Als Datenformat wird dabei
häufig \acrshort{json} oder \acrshort{xml} verwendet.

Neben \acrshort{xml}-basierten Webservices, die beispielsweise das \acrshort{saml}-Framework, SOAP,
oder \gls{xmlrpc} verwenden, wird \acrshort{xml} auch in einer Vielzahl weiterer
Einsatzbereiche eingesetzt. So dient \acrshort{xml} den Dateiformaten \acrshort{rss}/\acrshort{asf}, \acrshort{mathml},
\gls{svg} oder \gls{xhtml} als Basis. Auch die gängigen
Office-Dateiformate -- das \acrfull{odf}, Microsofts % chktex 8
\acrfull{ooxml} und Apples iWorks -- bauen auf \acrshort{xml} auf. % chktex 8

Inzwischen gewinnt jedoch \acrshort{json} vor allem im Mobile-
und Web-Bereich immer mehr an Bedeutung. Laut dem \acrshort{api}-Verzeichnis
\emph{ProgrammableWeb} unterstützten im Jahr 2013 ca. 60\% aller neu
hinzugefügten \acrshortpl{api} das \acrshort{json}-Format, während
\acrshort{xml} im selben Zeitraum lediglich von 37\% der neuen \acrshortpl{api}
unterstützt wurde.~\cite{duvander2013convergence}

\acrshort{json} ist bei bestimmten Aufgaben in puncto Geschwindigkeit und
Ressourcenauslastung deutlich effizienter~\cite{nurseitov2009comparison} als
\acrshort{xml}\@. inzwischen setzen auch einige populäre NoSQL-Datenbanken wie
\emph{CouchDB} oder \emph{MongoDB} auf \acrshort{json} zur Speicherung. Auch
MySQL verfügt seit Version 5.7.8 über einen nativen \acrshort{json}-Datentyp.

Die Umwandlung zwischen den beiden Formaten kann aus vielen Gründen
notwendig werden. Soll beispielsweise ein SOAP-Webservice als moderne
\acrshort{json}-\acrshort{rest}-Ressource angeboten werden, muss zwischen \acrshort{xml} und \acrshort{json} konvertiert
werden. Auch bei der Speicherung von \acrshort{xml}-Daten in den oben genannten NoSQL-Datenbanken
ist dies der Fall. Zudem ist die Unterstützung der Formate durch
Programmiersprachen, Frameworks und Applikationen nicht immer gleich gut.

\section{Verwandte Arbeiten}
% TODO: Add related works
List related work \emph{and} the result of this work\@! What is the relevance of this work concerning your thesis\@? If necessary,\ \emph{emphasize} some words in your text, for example words like \emph{not} or \emph{and} are sometimes crucial for understanding\@.

\section{Zielsetzung}
% TODO: Add contribution
What is your contribution?

\section{Aufbau der Arbeit}
\label{sec:structure}

Nach der Einleitung wird zunächst eine Einführung in die Struktur der
Formate \acrshort{xml} und \acrshort{json} gegeben (wobei die Formate allerdings selbstverständlich
nicht erschöpfend dargestellt werden können, da dies den Rahmen sprengen
würde). Im Zuge dessen werden ebenfalls die wichtigsten Unterschiede zwischen
beiden Formaten aufgezeigt (z.~B.\ \acrshort{json}-Arrays, \acrshort{xml}-Attribute, Duck Typing,
etc.).

Daran anschließend wird zunächst das Ziel konkret definiert, d.~h.\ es werden
überprüfbaren Kriterien aufgestellt, denen ein Konversionsverfahren genügen
muss. Dazu gehört beispielsweise, Verlustlosigkeit genauer zu definieren (vor
allem im Bezug auf Fragen wie \enquote{Sind Kommentare relevant?}). Hier wird
auch ein Abschnitt zu Angriffen (\emph{\gls{xxe}}, \emph{Billion Laughs}, etc.)
eingefügt, gegen den der jeweilige Konverter gewappnet sein soll.

Dann werden die verschiedenen bestehenden Ansätze zur Umformung aufgezählt
und die zugrunde liegenden Konzepte kurz und grob beschrieben.

Der folgende Abschnitt erklärt den Versuchsaufbau (\acrshort{xml}-Testdokumente).
Die verschiedenen Konversionsverfahren werden dann anhand des zuvor
aufgestellten Anforderungskatalogs geprüft.

Anschließend werden die Ergebnisse der Tests vorgestellt und diskutiert.
Dabei wird erörtert, ob und welche Verfahren die zuvor genannten Bedingungen
erfüllen.

Falls keines der Konversionsverfahren den zuvor aufgestellten Kriterien genügt,
folgt der Abschnitt, der einen eigenen Umwandlungsalgorithmus
entwickelt.

Im Diskussions-Abschnitt werden die aktuellen Möglichkeiten zur Umwandlung
von \acrshort{xml} in \acrshort{json} und wieder zurück abschließend eingeschätzt und bewertet.
Dabei soll auch ein Ausblick auf weitere Verbesserungsmöglichkeiten oder
alternative Wege zum Ziel erörtert werden.

Im letzten Kapitel zu verwandten Arbeiten werden weitere Arbeiten zum Thema
beschrieben und Unterschiede zur eigenen Arbeit herausgearbeitet.


\chapter{Hintergrund} \label{chap:background}

\section{\acrfull{xml}}

Bei der \acrfull{xml} handelt es sich um eine weit verbreitete Auszeichnungssprache. Die Entwicklung von \acrshort{xml} begann im Jahr 1996, zwei Jahre später wurde die die Spezifikation erstmals als Empfehlung des \acrshort{w3c} veröffentlicht.

\acrshort{xml} ist eine plattformunabhängige Metasprache, die für den Einsatz im Internet und den Datenaustausch zwischen Anwendungen konzipiert wurde und kann als Basis für neue Datenformate genutzt werden.

Die Auszeichnungsprache basiert auf der im ISO-Standard 8879 beschriebenen \acrfull{sgml} und stellt eine Teilmenge von \acrshort{sgml} dar, weshalb \acrshort{xml}-Dokumente zugleich immer auch \acrshort{sgml}-Dokumente sind. Eines der Designziele war die Reduzierung der Komplexität im Vergleich zu \acrshort{sgml}, denn optionale Zusatzfeatures und das vom Standard erlaubte Auslassen von Teilen des Marksup macht das korrekte Parsing von \acrshort{sgml}-Dokumenten vergleichsweise schwierig. Im Gegensatz dazu soll \acrshort{xml} einfacher zu verarbeiten sein -- auch wenn das zu Lasten der Dateigröße geht.\cite[Abschnitt 1.1]{maler2008xml,bray1998axml}

Aktuell steht \acrshort{xml} in zwei unterschiedlichen Versionen zu Verfügung:
\begin{itemize}
    \item{} \acrshort{xml} 1.0 (Fifth Edition), veröffentlicht am \DTMdate{2008-11-26}
    \item{} \acrshort{xml} 1.1 (Second Edition), veröffentlicht am \DTMdate{2006-08-15}
\end{itemize}

\acrshort{xml} 1.1 ist in der Praxis kaum verbreitet und wird daher in der vorliegenden Arbeit nicht betrachtet.

Als Metasprache bildet \acrshort{xml} die Grundlage für eine große Anzahl von Dateiformaten wie \gls{svg}, \gls{odf} und \gls{ooxml} und wird in Protokollen wie SOAP und \acrshort{ebics} zum Einsatz.

Neben XML-basierten Webservices, die beispielsweise das SAML-Framework, SOAP
oder XML-RPC verwenden, wird XML auch in einer Vielzahl weiterer
Einsatzbereiche eingesetzt. So dient XML den Dateiformaten RSS/ASF, MathML,
Scalable Vector Graphics (SVG) oder XHTML als Basis. Auch die gängigen
Office-Dateiformate -- das \emph{Open Document Format}, Microsofts % chktex 8
\emph{Office Open XML} und Apples \emph{iWorks} -- bauen auf XML auf. % chktex 8
\subsection{Grundlagen}

\begin{figure}[h!]
    \begin{example}[\acrshort{xml}-Dokument] Ein  einfaches \acrshort{xml}-Dokument, das verschiedene Knoten-Typen enthält.

    \label{ex:xmldoc}
    \inputminted{xml}{xmltree.xml}
    \end{example}
\end{figure}

\begin{figure}[h!]
    \begin{example}[\acrshort{xml}-Nodes] Die verschiedenen \acrshort{xml}-Nodes und ihre Eltern-Kind-Beziehungen stellen eine Braumstruktur dar.

        \begin{captionbeside}%
            {Das \acrshort{xml}-Dokument aus Beispiel \ref{ex:xmldoc} kann als Baumstruktur dargestellt werden.}
            \label{ex:xmltree}
            \includestandalone[width=0.5\textwidth]{xmltree}
        \end{captionbeside}
    \end{example}
\end{figure}

\acrshort{xml}-Daten bestehen aus Text, deren Struktur durch die Produktionsregeln der \acrshort{xml}-Spezifikation vorgegeben ist. Der Text eines \acrshort{xml}-Dokuments besteht aus einem Gemisch von \emph{Markup} und \emph{Character Data}, wobei das Markup eine Vielzahl von Formen annehmen kann. Text der \emph{kein} Markup enthält, nennt man \emph{Character Data}.

Auf die in \acrshort{xml}-Dokumenten enthaltenen Informationen in der Regel über Schnittstellen zugegriffen, die das Markup auswerten und das Dokument in Form einer baumartigen Datenstruktur darstellen. Ein prominentes Beispiel dafür ist das \acrfull{dom}, das u.a. von \emph{JavaScript}-Programmen in Webbrowsern eingesetzt wird, um  \acrshort{html}-Seiten auszuwerden und zu manipulieren.

Das Beispiel \ref{ex:xmldoc} zeigt ein \acrshort{xml}-Dokument, das mehrere verschiedene Knotentypen enthält:

\begin{description}
    \item[Document] ist der Wurzelknoten der Baumstruktur und kommt daher genau ein Mal im Baum vor. Es repräsentiert das \acrshort{xml}-Dokument selbst und macht die darin enthaltenen Informationen über seine Kindknoten zugänglich.\cite[Abschnitt 2.1]{xmlinfoset}
    \item[Elemente] sind Knoten, die weitere Elemente oder andere Knoten als Kinder enthalten können. Auf der Dokumentebene muss es immer genau ein Element (das \emph{Document Element}) geben. Zudem könne sie mit Attributen versehen werden. Elemente bestehen entweder aus einem Paar von Start- und End-Tags (z.B. \mintinline{xml}{<foo>} und \mintinline{xml}{</foo>} oder können -- falls sie keine Kindknoten enthalten -- aus einem einzelnem leeren Element-Tag (z.B. \mintinline{xml}{<foo />}).
    \item[Attribute] sind Schlüssel-Wert-Paare, die einem einzelnen Element-Knoten zugeordnet ist. Sie werden im Start-Tag eines Elements angegeben. So enthält das \mintinline{xml}{<album>}-Tag im Beispiel \ref{ex:xmldoc} enthält das Attribut \mintinline{xml}{catno="ARGO LP 628"}.
    \item[Kommentare] können Text (z.B. Beschreibungen zu Teilen des \acrshort{xml}-Dokuments) enthalten, sind jedoch nicht Teil der \emph{Character Data} des Dokuments. Sie werden mit den Zeichenfolgen \texttt{<!--} und \texttt{-->} eingeleitet und beendet und dürfen die Zeichenfolge\texttt{--} nicht enthalten.\cite[Abschnitt 2.5]{maler2008xml}
    \item[\glspl{pi}] sind Steueranweisungen an ein bestimmtes, das \acrshort{xml}-Dokument verarbeitendes Programm. Sie bestehen aus einem \emph{Ziel} (einer Zeichenkette, die nicht \texttt{xml} sein darf) und \emph{Daten}, die häufig in einer Attributen nachempfundenen Form angegeben sind. \glspl{pi} werden mit den Zeichenfolgen {\emph{<?}} und {\emph{?>}} eingeleitet und beendet.\cite[Abschnitt 2.6]{maler2008xml}
    \item[Text] sind Blattknoten, die Zeichendaten (Character Data) enthalten. Im Beispiel \ref{ex:xmldoc} enthält das \mintinline{xml}{<artist>}-Element den einen Textknoten, der den Text \enquote{Ahmad Jamal Trio} repräsentiert.
\end{description}

Es existieren jedoch auch abweichende Datenmodelle um die logische Strukur von \acrshort{xml}-Daten zu beschreiben, beispielsweise das \acrfull{xdm} oder das \acrshort{xml} Information Set\cite{xmlinfoset}. Die verschiedenenen Modelle unterscheidenen sich dabei vor allem im Abstraktionsgrad -- so stellt das \gls{dom} Textknoten und CDATA-Sektionen als separate Knotentypen dar, während das XML Information Set Zeichendaten beide Knotentypen unter \emph{Character Information Data} subsumiert\cite[Abschnitt 2.6]{xmlinfoset}.

\begin{description}
    \item[CDATA-Abschnitte] kennzeichnen größere Bereiche von Text, die Markup-Zeichen wie \enquote{\texttt{<}} oder \enquote{\texttt{\&}} enthalten dürfen, ohne dass diese quotiert werden müssten. Sie werden mit der Zeichenkette \enquote{\texttt{<![CDATA[}} eingeleitet und mit \enquote{\texttt{]]>}} beendet.\cite[Abschnitt 2.7]{maler2008xml}
\end{description}

\gls{dom}, \acrshort{xdm} und XML Information Set kennen noch eine Vielzahl weiterer Konstrukte, die Teil eines \acrshort{xml}-Dokuments sein können, beispielseweise \acrfull{dtd}, Notation-Knoten, Entities bzw. Entity-Referenzen, etc. Für eine detaillierte Beschreibung empfiehlt es sich, die jeweiligen Spezifikationen zurate zu ziehen.\cite{dom,xmlinfoset,xdm,xml}

\subsection{\acrfull{xsd}}

Um die Struktur von \acrshort{xml}-Dokumenten beschreiben und festzulegen, können sogenannte Schemata eingesetzt werden. Diese legen fest, welche Elemente in einem Dokument vorhanden sein dürfen oder müssen. Ebenso definieren sie Anzahl und Reihenfolge von Elementen, aber auch die Datentypen von Attributwerten und Textknoten. Attributen eines Elements können mithilfe eines Schemas auch Standardwerte zugewiesen oder als unveränderbar deklariert werden.

Neben der vom \gls{w3c} spezifizierten \acrfull{xsd} gibt es noch eine Reihe ähnlicher Technologien wie \acrshort{relaxng}, Schematron, \gls{dsd}, etc.

\subsection{\acrshort{xml}-Namespaces}
\label{sec:xmlns}

\acrshort{xml}-Dokumente beinhalten häufige eine Vielzahl verschiedener Element- und Attributnamen, die die verschiedenen Inhalte des Dokuments strukturieren und semantisch voneinander unterscheidbar machen sollen. Namensräume bieten die Möglichkeit, logisch verwandte Namen zu gruppieren -- ein Namensraum ist also eine Menge von Element- und Attributnamen, die innerhalb des Namensraums eindeutig sein müssen.

Namensräume werden mittels des eines Bezeichners -- dem Namespace-Namen -- identifiziert, der die Form eines \glspl{uri} vorliegt. Dieser ist allerdings nur als Name zu verstehen -- eine Nutzung des \gls{uri} zum Abruf von Daten (z.B aus dem Internet) ist nicht vorgesehen.

\begin{figure}[h]
    \begin{example}[\acrshort{xml}-Namespaces]
        \label{ex:c14n}
        Anhand des Namenraum kann die semantische Bedeutung verscheidener Elemente trotz gleichen Tag-Namens unterschieden werden.
        \inputminted[firstline=2,firstnumber=1]{xml}{ex-xmlns.xml}
        \captionof{figure}{Ein \acrshort{xml}-Dokument, das mehrere verschiedene Namensräume enthält.}
    \end{example}
\end{figure}

Um in einem \acrshort{xml}-Dokument zu deklarieren, kann das \mintinline{xml}{xmlns}-Attribut verwendet werden. Dabei gibt es zwei Formen:

\begin{description}
    \item[Deklaration eines Namespace Prefix] Durch Angabe eines Attributes in der Form \mintinline{xml}{xmlns:foo} kann einem Namensraum ein Prefix zugewiesen werden. Um zu signalisieren, dass das Element oder eines seiner Kindelemente zu diesem Namensraum gehört, wird der Tagname mit diesem Prefix versehen. Dies ermöglicht auch die Nutzung und Vermischung des Vokabulars verschiedener Namensräume in einem einzelnen Dokument.
    \item[Deklaration des Vorganenamensraums] Sind Tags mit keinem Prefix versehen, gehören diese zum Vorgabenamensraum. Dieser kann mithilfe des \mintinline{xml}{xmlns}-Attribut eines Elements für das Element und seiner Kindelemente festgelegt werden.
\end{description}

\subsection{\acrfull{c14n}}

Logisch äquivalente \acrshort{xml}-Dokumente können sich in der konkreten Darstellung
stark unterscheiden. Das \gls{w3c} hat daher Richtlinien zur Umwandlung in eine
einheitliche Darstellungsweise---die sogenannte \emph{Kanonische Form}---
festgelegt. Der Umwandlungprozess wird \acrfull{c14n} genannt.\cite{boyer2001c14n}

\begin{figure}[h!]
\begin{example}[Kanonisierung]
\label{ex:c14n}

Ein \acrshort{xml}-Dokument vor und nach der Kanonisierung. Beide Versionen sind
logisch äquivalent.

\inputminted{xml}{ex-c14n-pre.xml}
\captionof{figure}{Das unkanonisierte \acrshort{xml}-Dokument enthält überflüssigen Whitespace und z.T. einfache Hochkommas als Attributwert-Trennzeichen.}
\inputminted{xml}{ex-c14n-post.xml}
\captionof{figure}{Im kanonisierten \acrshort{xml}-Dokument wurde überflüssiger Whitespace entfernt, durchgehend das Anführungszeichen als Attributtrennzeichen verwendet und lediglich aus einem Start-Tag bestehende leere Elemente durch das entsprechende Start-End-Tag-Paar ersetzt.}
\end{example}
\end{figure}

\subsubsection{Motivation und Anwendungsbereich}
\label{sec:c14nscope}

Durch die Flexibilität des \acrshort{xml}-Standards besteht die MöglichkeitInformationen durch eine Vielzahl \acrshort{xml}-Dokumente darzustellen.\cite{siddiqui2002c14n} Die im Beispiel~\ref{ex:c14n} abgebildeten \acrshort{xml}-Dokumente haben einen identischen Informationsgehalt, lediglich die Darstellungsweisen unterscheiden sich. Allein die Möglichkeit, innerhalb eines Start-Tags eine beliebe Anzahl von Whitespace einzusetzen\cite[Produktionsregeln 3 und 40] führt bereits zu einer unendlichen Anzahl von logisch äquivalenten XML-Darstellungen einer Information.

Um die verschiedenden XML-Dokumenten enthaltene Logik vergleichbar zu machen, muss daher von der konkreten Darstellung abstrahiert werden. \acrlong{c14n} überführt daher beliebige XML-Dokumente in eine \emph{Kanonische Form}, indem eine Reihe von Arbeitsschritten abgearbeitet werden, die in Abschnitt~\ref{sec:c14nsteps} beschrieben sind.

\acrlong{c14n} wird vor allem im Bereich der Signaturerstellung und -verifikation eingesetzt. Dabei wird nicht das zu siginierende Datenobjekt selbst, sondern lediglich ein \emph{Digest} d.h. ein Hashwert des Objekts) signiert.\cite[Abschnitt 2.0]{bartel2008xmlsig} Subtile und für die Dokumentlogik irrelevante Unterschiede wie der Einsatz von abweichenden Zeilenumbrüchen oder die Nutzung von einfachen statt doppelten Hochkomma als Trennzeichen für Attribute würden dabei den Hashwert ändern und die Signatur ungültig werden lassen. Daher werden die Datenobjekte standardmäßig mittels \acrlong{c14n} transformiert.\cite[Abschnitt 4.3.3.2]{bartel2008xmlsig}

\subsubsection{Arbeitsschritte}
\label{sec:c14nsteps}

Bei der \acrlong{c14n} werden grob folgende Änderungen am
\acrshort{xml}-Dokument vorgenommen\cite[Abschnitt 1.1]{boyer2001c14n}:

\begin{enumerate}
    \item{} {Kodierung der Eingabe mit dem UTF-8-Zeichensatz}
    \item{} {Normalisieren der Zeilenumbrüche}\\\\
        Die Konventionen für die Speicherung von Zeilenenden (\texttt{EOL})
        sind je nach Betriebssystem unterschiedlich. So nutzen unixode
        Betriebssysteme wie Linux und BSD beispielsweise einen
        Zeilenumbruch\footnote{Unicode Code-Point \texttt{U+000A}:
        \texttt{LINE FEED (LF)}} (\texttt{{\textbackslash}n}), Windows und DOS hingegen
        nutzen die Kombination Wagenrücklauf\footnote{Unicode Code-Point
        \texttt{U+000D}: \texttt{CARRIAGE RETURN (CR)}} und Zeilenumbruch
        (\texttt{{\textbackslash}r{\textbackslash}n}). Ältere Mac-OS-Versionen setzen hingegen die
        Kombination Zeilenumbruch und Wagenrücklauf (\texttt{{\textbackslash}n{\textbackslash}r})
        ein.\cite[S.~212]{unicode9}
        Bei der Kanonisierung werden alle Varianten durch einen einfachen
        Zeilenumbruch ohne Wagenrücklauf (\texttt{{\textbackslash}n}) ersetzt.
    \item{} {Normalisieren der Attributwerte}\\\\
        Dabei werden einzelne oder mehrere aufeinanderfolgende Whitespace-Zeichen wie Leerzeichen, Tabulatoren und Zeilenumbrüche in Attributwerten zu einem einzelnen Leerzeichen zusammengefasst.
    \item{} {Zeichen- und Entity-Referenzen werden ersetzt}
    \item{} \texttt{CDATA}-Abschnitte werden in normale Text-Nodes umgewandelt
    \item{} \acrshort{xml}-Deklaration und \gls{dtd} werden entfernt
    \item{} Keere Elemente werden in Paare bestehend aus Start-End-Tags umgewandelt
    \item{} Whitespace außerhalb des Wurzelelements und innerhalb von Start- und End-Tags wird normalisiert
    \item{} Jeglicher Whitespace innerhalb Wurzelelements wird (mit Ausnahme der o.g. normalisierten Zeilenenden) unverändert belassen
    \item{} Alle Attributwert-Trennzeichen werden in Anführungszeichen (\enquote{double quotes})\footnote{Unicode Code-Point \texttt{U+0022}: \texttt{QUOTATION MARK}} umgewandelt
    \item{} Sonderzeichen in Attributwerten und Zeicheninhalt werden durch \acrshort{xml} Character References ersetzt.
    \item{} Überflüssige Namespace-Deklarationen werden entfernt.
    \item{} In der \texttt{ATTLIST} angegebene Vorgabeattribute werden zu allen Elementen hinzugefügt.
    \item{} Attribute und Namespace-Deklarationen aller Elemente werden in lexikographische Ordnung gebracht
\end{enumerate}

\subsubsection{Kommentare}
\label{sec:c14ncomments}

Die \gls{w3c}-Empfehlung für kanonisches XML erlaubt sowohl das ersatzlose Entfernen als auch das Beibehalten der Kommentare, wobei das Resultat in letzterem Fall als \emph{Kanonisches XML mit Kommentaren} (engl. \enquote{canonical XML with comments})\cite[Abschnitt 2.1]{boyer2001c14n}. Implementierung müssen in der Lage sein, kanonisches XML ohne Kommentare zu erstellen, während die Ünterstützung für kanonisches XML mit Kommentaren zwar empfohlen, jedoch nicht vorgeschrieben ist.

\subsubsection{Exklusive \acrlong{c14n}}
\label{sec:excc14n}

Einen Sonderfall stellt die Exklusive \acrlong{c14n} dar, die in einer separaten Empfehlung des \gls{w3c} beschrieben ist.\cite{boyer2002excc14n} Während bei der regulären (inklusiven) \acrlong{c14n} der Vorfahren-Kontext (z.B. Namespaces und Attribute im \acrshort{xml}-Namensraum) von Teildokumente eines XML-Dokuments erhalten bleibt, wird dieser bei der exklusiven \acrshort{c14n} verworfen.\cite[Abschnitt~18]{siddiqui2002c14n2} Dies kann insbesondere dann sinnvoll sein, wenn Teildokumente aus dem Quelldokumente herausgelöst und ggf. andere Dokumente eingebettet werden sollen (Re-Enveloping), ohne dass sich ihre digitale Signatur ändert.

\section{\acrfull{json}}
\label{sec:json}

Vor allem im Bereich des Datenaustausch hat sich die \acrfull{json} als Alternative zum gängigen \acrshort{xml}-Format etabliert. Im Gegensatz zum dokumentorientierten Auszeichnungssprache \acrshort{xml} wird \acrshort{json} häufig als \emph{datenorientiert} bezeichnet.\cite{gupta2007xmljson}

\subsection{Datenmodell und Syntax}
\label{sec:jsontypes}

Das \acrshort{json}-Format verfügt über mehrere verschiedene Datentypen, darunter die Containertypen \emph{Objekt} und \emph{Array}:\cite{ecma404}

\begin{description}
    \item[Objekte] werden mit geschweiften Klammern \texttt{\{\}}angegeben und enhalten eine kommaseparierten ungeordneten Liste von beliebig vielen Schlüssel-Wert-Paaren. Die Schlüssel müssen vom Typ Strings sein und dürfen im Objekt nur einmal vorkommen. Die Werte dürfen beliebigen Typs sein (auch weitere Objekte).
    \item[Arrays] sind neben \emph{Objekten} die zweite Containerklasse des \acrshort{json}-Formats und enhalten eine kommaseparierte geordnete Liste von Werten beliebigen Typs. Diese Liste wird von eckigen Klammern (\texttt{[]} umschlossen darf auch leer sein.
    \item[Strings] sind einfache Zeichenketten, die aus beliebig vielen Zeichen bestehen und von doppelten Anführungszeichen (\texttt{"}) umschlossen sind. Kontrollzeichen, doppelte Anführungszeichen und der Backslash (\texttt{\textbackslash})  müssen mit einem Backslash quotiert werden.
    \item[Zahlen] werden als Dezimalwert ohne führende Null dargestellt. Für negativen Werte kann ein Minuszeichen vorangestellt werden. Bei der Angabe von Nachkommastellen werden diese mittels eines Punkts (\texttt{.}) vom Ganzzahlteil getrennt. Auch die wissenschaftliche \enquote{E-Notation} ist erlaubt.
    \item[\texttt{true} / \texttt{false}] bezeichnet die Booleschen Werte \enquote{wahr} und \enquote{falsch}.
    \item[\texttt{null}] steht für den Nullwert.
\end{description}

\begin{definition}Formale Syntax der \acrfull{json}
\label{def:json}

Unerheblicher Whitespace (Tabulator\footnote{Unicode-Codepoint \texttt{U+0009}: \texttt{CHARACTER TABULATION}}, Zeilenumbruch\footnote{Unicode-Codepoint \texttt{U+000A}: \texttt{LINE FEED (LF)}}, Wagenrücklauf\footnote{Unicode-Codepoint \texttt{U+000D}: \texttt{CARRIAGE RETURN (CR)}} und Leerzeichen\footnote{Unicode-Codepoint \texttt{U+0020}: \texttt{SPACE}}) kann vor oder nach allen Token (\lit{\{}\footnote{Unicode-Codepoint \texttt{U+007B}: \texttt{LEFT CURLY BRACKET}}, \lit{\}}\footnote{Unicode-Codepoint \texttt{U+007D}: \texttt{RIGHT CURLY BRACKET}}, \lit{[}\footnote{Unicode-Codepoint \texttt{U+005B}: \texttt{LEFT SQUARE BRACKET}}, \lit{]}\footnote{Unicode-Codepoint \texttt{U+005D}: \texttt{RIGHT SQUARE BRACKET}}, \lit{:}\footnote{Unicode-Codepoint \texttt{U+003A}: \texttt{COLON}}, \lit{,}\footnote{Unicode-Codepoint \texttt{U+002C}: \texttt{COMMA}}) eingefügt werden.

\begin{grammar}
    <value> ::= \[[ \begin{stack}
                <object>\\
                <array>\\
                <string>\\
                <number>\\
                `true'\\
                `false'\\
                `null'
            \end{stack} \]]

    <object> ::= \[[ `\{' \begin{stack}
                \begin{rep}
                    <string> `:' <value>\\
                    `,'
                \end{rep}\\
        \end{stack} `\}' \]]

    <array> ::= \[[ `[' \begin{stack}
                \begin{rep}
                    <value>\\
                    `,'
                \end{rep}\\
        \end{stack} `]' \]]

    <number> ::= \[[
        \begin{stack}
            \\`-'
        \end{stack}
        \begin{stack}
            `0'\\
            "\lit1 - \lit9"
            \begin{rep}
                \\"\lit0 - \lit9"
            \end{rep}
        \end{stack}
        \begin{stack}\\
            `.' \begin{rep}
                    "\lit0 - \lit9"
                \end{rep}
        \end{stack}
        \begin{stack}\\
            \begin{stack}
                `e'\\
                `E'
            \end{stack}
            \begin{stack}
                \\
                `+'\\
                `-'\\
            \end{stack}
            \begin{rep}
                "\lit0 - \lit9"
            \end{rep}
        \end{stack}
        \]]

    <string> ::= \[[
        `\textquotedbl' \begin{stack}\\
                \begin{rep}
                    \begin{stack}
                        \tok{\color{black} Any Unicode char except \texttt{\textquotedbl}, \texttt{\textbackslash} or control char}\\
                        `\textbackslash' \begin{stack}
                            `\textquotedbl'\\
                            `\textbackslash'\\
                            `/'\\
                            `b'\\
                            `f'\\
                            `n'\\
                            `r'\\
                            `t'\\
                            `u' \tok{\color{black} 4 hexadecimal digits}
                        \end{stack}
                    \end{stack}
                \end{rep}
        \end{stack} `\textquotedbl'
        \]]
\end{grammar}
\end{definition}

\subsection{Angriffe}

Das \acrshort{json}-Format ist sehr einfach aufgebaut -- komplexere Features wie die beispielsweise die Referenzierung von internen oder externen Werten und Dateien fehlen in der Spezifikation. Damit gibt es kein Äquivalent zu den \emph{Entities} des \acrshort{xml}-Formats, die Grundlage der meisten Angriffe auf \acrshort{xml}-Parser sind.

Ein \acrfull{idraft} der \gls{ietf} von September 2012 schlug eine solche Funktionalität unter dem Titel \enquote{\acrshort{json} Reference} zwar vor, diese kam aber aber nie über das Entwurfsstadium hinaus.\cite{jsonref}

Eine Spezifikation für die Nutzung von Schemas ist aktuell im Entwurfsstadium. Sie beschreibt ebenfalls eine Möglichkeit zur Referenzierung von Schemata und \acrshort{json}-Werten.\cite[Abschnitt 8]{jsonschema}. Im Gegensatz zu \acrshort{xml} sieht die Spezifikation allerdings zur Zeit keine Möglichkeit zur Einbettung eines Schemas in ein Datendokument vor. Auch der Verweis auf ein Schema im Datendokument selbst nicht vorgesehen, stattdessen können aber HTTP-Header \cite[Abschnitt 10.1]{jsonschema} verwendet werden, um auf ein Schema zu verweisen. Daher werden die Sicherheitsaspekte von \emph{\acrshort{json} Schema} im Rahmen dieser Bacherlorarbeit nicht betrachtet.

\begin{figure}[h]
\begin{example}[Unsicheres \acrshort{json}-Parsing] Wird \acrshort{json} mittels \mintinline{javascript}{eval()}-Funktion geparst, kann ein Angreifer u.U. Schadcode ausführen.
\inputminted{javascript}{json-eval.js}
\begin{center}
\frame{\includegraphics[width=\textwidth]{json-eval-alert}}
\end{center}
\caption{Der im String eingebetten Funktionsaufruf öffnet ein \mintinline{javascript}{alert()}-Benachrichtungsfenster im Browser.}
\end{example}
\end{figure}

Ein möglicher Angriffspunkt beim Einsatz von \acrshort{json}-verarbeitenden Applikationen auf JavaScript-Basis kann der Einsatz der \mintinline{javascript}{eval()}-Funktion\cite[Abschnitt 18.2.1]{ecma262}. Als Teilmenge von JavaScript\footnote{Auch wenn diese Aussage vom \acrshort{json}-Entwickler selbst stammt\cite{crockford2006fatfree}, ist sie umstritten, da JavaScript auf Zahlen im IEEE-754 binary64-Format\cite[Abschnitt 6.1.6]{ecma262} beschränkt ist. Im Gegensatz dazu enthält die \acrshort{json}-Spezifikation keinerlei Einschränkungen bezüglich der Fließkommagenauigkeit\cite[Abschnitt 8]{ecma404} -- \acrshort{json}-Dokumente können daher auch Zahlen enthalten, die in JavaScript nicht darstellbar sind.} ist die Nutzung der Funktion zwar möglich, kann jedoch ohne weitergehende Validierung die Einschleusung von Schadcode ermöglichen.


\chapter{Umsetzung}
\label{chap:impl}

\section{Bewertungskriterien}

\subsection{Kriterien für verlustlose Konversion}

\subsubsection{CDATA-Abschnitt}

Mithilfe von CDATA-Abschnitten lässt sich Text, der Markup-Zeichen wie beispielsweise das Kleiner-als-Zeichen\footnote{Unicode-Codepoint \texttt{U+003C}: \texttt{LESS-THAN SIGN}} enthält, direkt in ein XML-Dokument einbetten, ohne dass diese Zeichen als Markup interpretiert werden.

Dies ist vor allem dann sinnvoll, wenn es unpraktikabel ist, alle Markup-Zeichen im Text einzeln durch die jeweilige Zeichen- oder Entity-Referenz zu ersetzen. CDATA stellt somit eine weitere Weg dar, Zeichendaten in einem XML-Dokument anzugeben.\cite[Abschnitt~2.4]{maler2008xml}

Der Unterschied zwischen Zeichendaten aus CDATA-Abschnitten und solchen, bei denen dies nicht der Fall ist, ist jedoch lediglich ein syntaktischer. Daher werden bei der \acrlong{c14n} alle CDATA-Abschnitte im Eingabedokument durch den entsprechenden \emph{Character Content} ersetzt\cite[Abschnitt~2.1]{boyer2001c14n}.

Für die Verlustlosigkeit der Konversion ist es daher unerheblich, ob die CDATA-Abschnitte im Ursprungsdokument als solche erhalten bleiben, oder lediglich die Zeichendaten beibehalten werden.

\subsubsection{Kommentare}

\acrshort{xml} verfügt über die Möglichkeit, Dokumente mit Kommentaren zu versehen. Allerdings müssen diese nicht durch den XML-Prozessor zugänglich gemacht werden. In der Empfehlung des \gls{w3c} heißt es dazu:

\begin{foreigndisplayquote}{english}[{\cite[Abschnitt~2.5]{maler2008xml}}]
They are not part of the document's character data; an XML processor may, but need not, make it possible for an application to retrieve the text of comments.
\end{foreigndisplayquote}

Zudem ist auch bei der Implementierung von \acrlong{c14n} die Unterstützung von \emph{Kanonischem XML mit Kommentaren} lediglich empfohlen, während die Möglichkeit der Umwandlung in \emph{Kanonisches XML} ausschließlich aller Kommentare zwingend erforderlich ist.\cite[Abschnitt~2.1]{boyer2001c14n}

Folglich ist es nicht nötig, dass sich die Kommentare im XML-Eingabedokument nach dem $XML\rightarrow{}JSON\rightarrow{}XML$ Roundtrip auch in der Ausgabe wiederfinden.

\subsubsection{Dokumentordnung}

\paragraph{Elementordnung}

Die Spezifikation der \acrshort{c14n} bezieht sich bezüglich der Ordnung auf die \acrshort{w3c}-Empfehlung zur \acrfull{xpath}\cite[Abschnitt~2.2]{boyer2001c14n}.

\begin{foreigndisplayquote}{english}[{\cite[Abschnitt~5]{clark1999xpath1}}]
There is an ordering, document order, defined on all the nodes in the document corresponding to the order in which the first character of the XML representation of each node occurs in the XML representation of the document after expansion of general entities. Thus, the root node will be the first node. Element nodes occur before their children. Thus, document order orders element nodes in order of the occurrence of their start-tag in the XML (after expansion of entities).
\end{foreigndisplayquote}

Folglich muss die Reihenfolge der Element-Nodes bei der Konversion beibehalten werden.

\paragraph{Ordnung von Attributen und Namespaces}

Laut dem \acrshort{xml}-Standard unwichtig kommen der Reihenfolge, in die Attribute eines Elements angegeben wurden, keine Bedeutung zu.

\begin{foreigndisplayquote}{english}[{\cite[Abschnitt~3.1]{maler2008xml}}]
Note that the order of attribute specifications in a start-tag or empty-element tag is not significant.
\end{foreigndisplayquote}

Dies wird auch von der \acrshort{xpath}-Spezifikation untermauert:

\begin{foreigndisplayquote}{english}[{\cite[Abschnitt~5]{clark1999xpath1}}]
The relative order of namespace nodes is implementation-dependent. The relative order of attribute nodes is implementation-dependent.
\end{foreigndisplayquote}

Die Reihenfolge der Attribute eines Elements nach einem $\text{\acrshort{xml}}\rightarrow{}\text{\acrshort{json}}\rightarrow{}\text{\acrshort{xml}}$-Roundtrip ist daher beliebig und muss nicht mit der Reihenfolge vor der Umwandlung identisch sein.

\subsubsection{Whitespace}

Whitespace innerhalb des Wurzelelements des Dokumentens muss beibehalten werden, während 

\begin{foreigndisplayquote}{english}[{\cite[Abschnitt~2.1]{boyer2001c14n}}]
    All whitespace within the root document element MUST be preserved (except for any \texttt{\#xD} characters deleted by line delimiter normalization). This includes all whitespace in external entities. Whitespace outside of the root document element MUST be discarded.
\end{foreigndisplayquote}


\section{XML-JSON-Konversionsverfahren}

\subsection{Json-lib}
\convtool[
    author={Andres Almiray (basierend auf Douglas Crockford)},
    url={http://json-lib.sourceforge.net/},
    license={Apache Software License 2.0},
    language=Java,
    version={2.4 (2010-12-14)},
]{jsonlib}

\subsection{JsonML}
\convtool[
    author={Stephen M. McKamey},
    url={http://www.jsonml.org/},
    license=MIT,
    language=JavaScript,
    version=2.0.0 (2016-04-09),
]{jsonml}

\subsection{JXON}
\convtool[
    author={MDN / Martin Raifer},
    url={https://github.com/tyrasd/jxon},
    license=GNU General Public License 3.0,
    language=JavaScript,
    version=2.0.0-beta.4 (2016-11-22),
]{jxon}

\subsection{org.json.XML}
\convtool[
    author={JSON.org / Sean Leary},
    url={https://github.com/stleary/JSON-java},
    license=MIT,
    language=Java,
    version=20160810 (2016-08-10),
]{orgjsonxml}

\subsection{Pesterfish}
\convtool[
    author={Jacob Smullyan},
    url={https://bitbucket.org/smulloni/pesterfish/},
    license=MIT,
    language=Python,
    version=1578db9 (2010-11-22),
]{pesterfish}

\subsection{x2js}
\convtool[
    author={Axinom},
    url={https://github.com/x2js/x2js},
    license=Apache Software License 2.0,
    language=JavaScript,
    version=3.1.0 (2016-12-05),
]{x2js}

\subsection{xmljson}
\convtool[
    author={S Anand},
    url={https://github.com/sanand0/xmljson},
    license=MIT,
    language=Python,
    version=0.1.7 (2016-09-13),
]{xmljson}


\chapter{Ergebnisse} \label{chap:results}

Es wurden insgesamt 101 Testfälle verwendet und 11 verschiedene Konverter überprüft. Allerdings konnte keiner der Konverter alle Anforderungen aus Abschnitt~\ref{sec:criteria} erfüllen.

Im Folgenden werden die Ergebnisse der verschiedenenen Konverter vorgestellt.

\begin{figure}[b!]
    \label{tbl:results-basic}
    \includestandalone[width=\textwidth]{resulttable-basic}
    \captionof{table}{Konversions-Testergebnisse bezüglich verschiedener Konversionprobleme.}
\end{figure}

\begin{figure}[H]
    \label{tbl:results-chars}
    \includestandalone[width=\textwidth]{resulttable-chars}
    \captionof{table}{Ergebnisse der Tests bezüglich Unterstützung der von der \acrshort{xml}-Spezifikation erlaubten Zeichen.}

    \vspace*{\floatsep}

    \label{tbl:results-complex}
    \includestandalone[width=\textwidth]{resulttable-complex}
    \captionof{table}{Testergebnisse der Konverter bei komplexen Dokumenten.}
\end{figure}

\begin{figure}[t!]
    \label{tbl:results-sec}
    \includestandalone[width=\textwidth]{resulttable-sec}
    \captionof{table}{Ergebnisse der Tests auf Sicherheitlücken}
\end{figure}
\section{Cobra vs Mongoose}
\label{sec:cobravsmongoose}

Die Reihenfolge der Elemente sowie Whitespace werden von \emph{Cobra vs Mongoose} verworfen. Das Auftreten vom Mixed Content, \glspl{pi} und Kommentaren im Urspungsdokument führt zu Fehlern bei der Rückkonvertierung von \acrshort{json} zu \acrshort{xml}.

\section{GreenCape \acrshort{xml} Converter}
\label{sec:greencapexml}

Bei der Umwandlung von \acrshort{json} zu \acrshort{xml}-Daten fügt der Konverter automatisch Zeilenumbrüche hinter allen Elementen und Einrückungen mit einer Breite von vier Leerzeichen ein. Dies führte dazu, dass der Konverter keinen der Tests bestand. Um die Überprüfung der anderen, davon unabhängigen Aspekte des Konverters gewährleisten zu können, musste dieses Verhalten durch einen Patch entfernt werden (siehe Anhang~\ref{appx:greencapexml}).

Die Konversion der umfangreichen \acrshort{odf}-Spezifikation in Form einer \texttt{*.fodt}-Datei mithilfe des \emph{GreenCape \acrshort{xml} Converters} schug fehl. Der Versuch wurde abgebrochen, nachdem der PHP-Prozess seit rund 3 Stunden bei 100\% CPU-Last eingefroren war. Eine Analyse mithilfe des Tools \mintinline{shell}{strace} zeigte, dass sich der PHP-Interpreter in einer Endlosschleife aus aufeinanderfolgenden \mintinline{c}{mmap()}- und \mintinline{c}{mummap()}-Syscalls befand (siehe Abb.~\ref{fig:greencapeloop}).

\begin{figure}[h!]
    \inputminted{shell-session}{greencapexml-strace.txt}
    \captionof{figure}{Eine Endlosschleife im \emph{GreenCape \acrshort{xml} Converter} musste mittels \texttt{SIGTERM} unterbrochen werden.}
    \label{fig:greencapeloop}
\end{figure}

\section{Json-lib}
\label{sec:jsonlib}

Bei Kommentare oder \glspl{pi} innerhalb des Wurzelelement des \acrshort{xml}-Dokuments stürzt der Konverter ab. \glspl{pi} außerhalb des Wurzelelements werden von \emph{Json-lib} ignoriert. Bei der Konversion geht zudem der Tagname des Wurzelelements verloren. Tritt Mixed Content auf, werden bei der Rückkonversion alle Text-Knoten zusammengefasst.

Mehrere aufeinanderfolgende Elemente selben Namens werden von \emph{Json-lib} in ein \acrshort{json}-Array konvertiert. Dabei wird jedoch lediglich der Inhalt der Elemente übernommen, während der Tagname verloren geht.

Enthält das Wurzelelement eine \acrshort{xml}-Dokument ledglich Zeichendaten, so werden diese bei einem Round-Trip zum Inhalt eines Kindelements des Wurzelelements -- in diesem Fall fügt \emph{Json-lib} also eine Elementebene hinzu, die im Ursprungsdokument nicht existierte. Eine genauere Analyse der ausgegebene Daten zeigte, das dieses Verhalten auch der Grund für das Scheitern der CDATA-Testfälle war. CDATA-Sektionen werden von \emph{Json-lib} sehr wohl unterstützt.

Leider war es nicht möglich, \emph{Json-lib} ohne besondere Konfiguration in einem Prozess mit begrenztem virtuellen Adressraum zu verwenden, da die \acrfull{jvm} in diesem Fall nicht gestartet werden kann~\cite{jvmmemlimit}. Um dieses Problem zu umgehen, muss beim Start des Java-Prozesses die maximale Größe des Heap-Adressraums sowie die Größe des für Zeiger auf Metadaten von Java-Klassen zur Verfügung stehenden Adressraums angegeben werden.

Eine sinnvolle Aussage über die Anfälligkeit gegenüber Angriffen aus dem Bereich \acrlong{dos} ist unter diesen Umständen jedoch nicht möglich, sodass die entsprechenden Tests für \emph{Json-lib} manuell wiederholt werden mussten.

\todo[inline]{TODO: Ergebnisse ergänzen}
%TODO: Results?

Ein Blick in den Quellcode der Klasse \texttt{net.sf.json.XMLSerializer} zeigt, dass Json-lib den \acrshort{xml}-Parser XOM einsetzt. Dieser ist anfällig für \acrshort{xxe}-Angriffe mittels General Entites bzw. Parameter Entities, die sowohl für \acrlong{fsa} als auch für \acrlong{ssrf} genutzt werden können.

Ebenso erlaubt der Parser die Einbettung lokaler Dateien mittels Parameter Entities in DTDs, sowohl in der ursprünglichen Version des Angriffs~\cite[S.~10]{morgan2014xml}, als auch in einer modifizierten Variante, die 2016 von Sicherheitsforschern der Ruhr-Universität Bochum vorgestellt wurde.~\cite[Abschn.~5.2]{spaeth2016sok}.

\section{\acrshort{jsonml}}
\label{sec:jsonml}

Das Konversionverfahren \acrshort{jsonml} ist vergleichsweise vollständig. Lediglich die in Ursprungsdokumenten enthaltenen \glspl{pi} werden vom Konverter ignoriert und nicht in die \acrshort{json}-Ausgabe übernommen.

\acrshort{jsonml} ist von allen getesten Konvertern als einziger in der Lage, alle Test-Dokumente verlustlos zu konvertieren, solange sie keine \glspl{pi} enhielten.

\section{JXON}
\label{sec:jxon}

JXON fehlt die Unterstützung von Whitespace und Mixed Content. Zudem geht die Reihenfolge der Elemente verloren. \acrlongpl{pi} werden bei der Konversion ignoriert.

\section{org.json.XML}
\label{sec:orgjsonxml}

Als Java-Package ist \texttt{org.json.XML} ist die Untersuchung von \acrshort{dos}-Angriffen wegen der Beschränkungen der \acrlong{jvm} ebenso problematisch wie bei \emph{Json-lib}~(siehe Abschn.~\ref{sec:jsonlib}). Auch bei org.json.\acrshort{xml} musste die Verwundbarkeit gegenüber solchen Angriffe daher manuell verifiziert werden.

Attribute gehen bei der Konvertierung von \acrshort{xml}-Daten in \acrshort{json} zwar nicht verloren -- bei der Rückkonvertierung kann der Konverter jedoch nicht mehr erkennen, dass es sich um Attribute handelt und interpretiert diese stattdessen als Elemente.

\acrshort{xml}-Namespace-Prefixe für Tag-Namen werden zwar grundsätzlich unterstützt, Namensräume können aber aufgrund der fehlerhaften Behandlung von Attributen dennoch nicht genutzt werden.
 Mixed Content wird nicht unterstützt -- enthält ein Element neben Kindelement auch Text, wird dieser verworfen.

Ebenfalls gehen bei der Konversion \glspl{pi} und die Reihenfolge der Dokumenteninhalte verloren.

Zudem werden Zahlenwerte sowie die Zeichenketten \mintinline{json}{true} und \mintinline{json}{false} in die nativen Java-Datentypen konvertiert, wobei informationsverlust auftreten kann. Die Zeichenkette \texttt{1e-324} wird beispielweise bei der Übersetzung zu \acrshort{json} als Zahl interpretiert und erscheint daher in der Ausgabe gerundet als \texttt{0}.

\section{Pesterfish}
\label{sec:pesterfish}

Bei der Konversion gehen die Namen der Namesprace-Prefixe verloren und werden durch generische Bezeichnungen (\texttt{ns0}, \texttt{ns1}, \dots{}) ersetzt. Dies werden auch dann verwendet, wenn im Ursprungsdokument keinerlei Namensraum-Prefixe verwendet wurden, sondern ein eigener Default-Namespace genutzt wurde. Grund dafür ist die \texttt{ElementTree}-\acrshort{api}, die Namespace-Prefixes beim Parsen von \acrshort{xml} automatisch zur vollen URI expandiert und den ursprünglichen Prefixnamen verwirft~\cite[Abschn.~20.5.1.7]{pythonetreexmlns}.
Zudem gehen bei der Konvertierung auch \glspl{pi} verloren.

Bei den Sicherheitstests zeigte sich, dass \emph{Pesterfish} bei Nutzung der Vorgabeeinstellungen verwundbar für die \acrshort{dos}-Angriffe \emph{Billion Laughs} und \emph{Quadratic Blowup Attacks ist.}. Da die Bibliothek jedoch die Verwendung einer eigenen \texttt{ElementTree}-Implementierung erlaubt, kann diesem Problem durch den Einsatz eines sicheren Ersatzes -- beispielsweise aus der \emph{defusedxml}-Bibliothek -- entgegengewirkt werden.

\section{x2js}
\label{sec:x2js}

Die Beibehaltung der Dokumentreihenfolge ist beim Einsatz von \emph{x2js} nicht gegeben. Bei Mixed Content wird der gesamte Character Content eines Elements zu einem einzelnen Textknoten zusammengefasst. Auch \glspl{pi} werden nicht unterstützt -- treten diese innerhalb des Wurzelelements auf, wird anstelle der \gls{pi} ein leeres Element namens \enquote{undefined} eingefügt.

Whitespace bleibt zwar grundsätzlich erhalten, Einrückungen werden jedoch wie anderer Mixed Content auch zusammengefasst, sodass bei Dokumenten mit Einrückungen nach der Konversion jedes Element mit zuvor eingerücktem Inhalt stattdessen einen einzelnen, nur aus Whitespace bestehenden Textknoten enthält.

\section{xmljson}
\label{sec:xmljson}

Im Gegensatz zu anderen Konvertern verfügt das Python-Package xmljson über keine Möglichkeit, einen \acrshort{xml}-Daten enthaltenden String direkt zu konvertieren, sondern akzeptiert lediglich bereits geparste \texttt{ElementTree}-Objekte. Da der Nutzer selbst für das Parsen des \acrshort{xml}-Dokuments zuständig ist und es im Unterschied zu Pesterfish (Vgl.~Abschn.~\ref{sec:pesterfish}) auch keine Voreinstellung gibt, wird eine Sicherheitsanalyse dieses Konverters hinfällig. Zum Durchführen der Tests wurde \texttt{defusedxml.lxml} verwendet.

Der \emph{xmljson}-Konverter kann mehrere verschiedene Konvertierungskonventionen nutzen (vgl.~Abschn.~\ref{sec:converters}), die unabhängig voneinander getestet wurden.

Durch den Einsatz der \texttt{ElementTree}-\acrshort{api} hat \emph{xmljson} keinen Zugriff auf die im \acrshort{xml}-Dokument verwendeten Prefixnamen, sodass diese wie beim \emph{Pesterfish}-Konverter durch generische Namen ersetztt werden~(vgl.~Abschn.~\ref{sec:pesterfish}).
\glspl{pi} werden von allen Konventionen ignoriert.

Lediglich die \emph{Cobra}- und \emph{Yahoo}-Konventionen (Abschn.~\ref{sec:xmljson-cobra} und~\ref{sec:xmljson-yahoo}) nutzen keine Typinferenz. Alle anderen Konverter wandeln die Zeichenketten \texttt{true} und \texttt{false} in boolsche Werte um. Auch Zahlenwerte werden als numerische Datentypen interpretiert, wodurch beispielsweise Rundungsfehler auftreten können oder Formatierung verloren geht (z.B. \mintinline{json}{1e+39} anstatt \mintinline{json}{"1E39"}). Sehr großen Zahlen werden als \mintinline{js}{Infinity}-Literal dargestellt, der zwar valides JavaScript wäre, von der \acrshort{json}-Spezifikation jedoch nicht erlaubt ist.

\subsection{Abdera}
\label{sec:xmljson-abdera}

Attribute werden bei der \acrshort{json}-Konvertierung mithilfe der \emph{Abdera}-Konvention zwar in die Ausgabe übernommen, allerdings ist \emph{xmljson} bei der Rückkonvertierung nicht in der Lage, diese von normalen Elementen zu unterscheiden. Daher befinden sich Attribute nach der Rückkonvertierung nicht mehr an der ursprünglichen Stelle im Dokument, sondern in einem \mintinline{xml}{<attributes>}-Element, das als Kindelement des Ursprungselements eingefügt wird. Ein zusätzliches \mintinline{xml}{<children>}-Element enthält die ursprünglichen Kindelemente des Elements.

Andererseits werden teilweise Elemente und Textknoten zu Attributwerten umgedeutet. So wird aus \mintinline{xml}{<a><b>hello</b></a>} durch einen Round-Trip \mintinline{xml}{<a b="hello" />}

Aufgrund der Probleme des Konverters, anhand der \acrshort{json}-Daten zwischen Elementen und Attributen zu unterscheiden und diese später wieder korrekt rekonstruieren zu können, schlagen auch Tests fehl, die der Konverter eigentlich bestehen könnte. So nutzt der Konverter bei mehreren Kindelementen \acrshort{json}-Arrays, d.h. die Elementreihenfolge ist auch nach der Konversion noch ersichtlich.
Auch Tag-Name und Attribute des Wurzelelements gehen eigentlich nicht verloren.

Mixed Content wird nicht unterstützt, es ist lediglich der erste Textknoten im Dokument auffindbar.
Führender oder anhängender Whitespace wird bei der Konversion verworfen.

\subsection{Badgerfish}
\label{sec:xmljson-badgerfish}

Die Ergebnisse stimmen im Wesentlichen mit denen  denen des \emph{Cobra-vs-Mongoose}-Konverters (vgl.~Abschn.~\ref{sec:cobravsmongoose}) überein, der ebenfalls auf die sogenannte Badgerfish-Konvention zur Darstellung von \acrshort{xml}-Strukturen in \acrshort{json} setzt. Nicht unterstütze Features wie Mixed Content, \glspl{pi} oder Kommentare führten bei \emph{xmljson} jedoch nicht zu eineme Absturz des Programms. Zudem kommt es zu Problemen durch den Einsatz von Typinferenz und dem Verlust der Prefixnamen von Namespaces.

\subsection{Cobra}
\label{sec:xmljson-cobra}

Bei \emph{Cobra} handelt es sich um eine modifizierte Variante von \emph{Abdera}, die sich hauptsächtlich darin unterscheiden, welche Schlüssel in der \acrshort{json}-Objekte-Repräsentation eines \acrshort{xml}-Dokuments optional sind. Daher hat auch \emph{Cobra} große Probleme mit der Rückkonvertierung zu \acrshort{xml} und der Unterscheidung zwischen Elementen und Attributen.

Ein weiterer Unterschied ist, dass bei Cobra keine Datentypen erraten werden, sondern alles als String behandelt wird.

\subsection{GData}
\label{sec:xmljson-gdata}

Durch die Nutzung von ungeordneten \acrshort{json}-Objekten geht bei dem Einsatz der \emph{GData}-Konvention die Reihenfolge der im \acrshort{xml}-Dokument enthaltenen Elemente verloren. Bei Mixed Content wird lediglich der erste Textknoten übernommen, alle weiteren werden verworfen.

Bei der Konversion kann es zu Informationsverlust durch Typinferenz kommen.

\subsection{Parker}
\label{sec:xmljson-parker}

In der Standardeinstellung verwirft diese Konvention das Wurzelelement -- dieses Verhalten ist jedoch über einen Parameter abschaltbar.

\todo[inline]{TODO: Hier fehlt noch was }
% TODO: Add content

\subsection{Yahoo}
\label{sec:xmljson-yahoo}

Die \emph{Yahoo} hat ebenfalls Probleme bei der Unterscheidung zwischen Elementen und Attributen, fügt aber im Gegensatz zu \emph{Abdera} und \emph{Cobra} nicht neue, im Originaldokument nicht existierende Elemente in das Dokument ein.

Die Elementreihenfolge geht bei der Konvertierung verloren.

Im \acrshort{xml}-Dokument enthaltener Character Content wird in allen getesteten Fällen zu \acrshort{json}-Strings konvertiert, eine Typumwandlung findet nicht statt.

\chapter{Weiterentwicklung eines Konversionsverfahrens}
\label{chap:jsonml}

Mit insgesamt 89 von 100 bestandenen Testfällen erfüllt die \acrfull{jsonml} von allen Konversionverfahren die meisten Kriterien im Test.

In diesem Kapitel wird daher zunächst ein Überblick über die von \acrshort{jsonml} eingesetzte Syntax gegeben.
Darauf aufbauend werden dann die notwendigen Modifikationen am \acrshort{jsonml}-Verfahren beschrieben, die im Zuge dieser Arbeit entwickelt wurden, um das Ziel eines vollständig verlustlose Konversionsverfahrens zu erreichen. Im Anschluss daran werden die Ergebnisse einer Analyse der so weiterentwickelten \acrshort{jsonml}.

\acrfull{jsonml}
\section{Syntax}

Im Gegensatz zu anderen Konvertern nutzt \acrshort{jsonml} ungeordnete \acrshort{json}-Objekte ausschließlich für Attributlisten~\cite{jsonmlsyntax}. Die Baumstruktur eines \acrshort{xml}-Dokuments wird mittels \acrshort{json}-Arrays dargestellt, wobei ein Array immer genau ein Element repräsentiert. Textknoten bzw. CDATA-Sektionen werden zu einfachen \acrshort{json}-Strings umgewandelt.

\begin{figure}[H]
    \begin{definition}[{Formale Syntax der \acrfull{jsonml}}]
        \label{def:jsonml}
        Sowohl \synt{tag-name} als auch \synt{attribute-name} sind \acrshort{json}-Werte vom Typ String. Die Whitespace-Regeln sind identisch wie bei \acrshort{json} (Vgl. Definition~\ref{def:json}).
        \begin{grammar}
            <element> ::= \[[
        \begin{stack}
            `[' <tag-name>
                \begin{stack}
                    `,' <attributes>\\
                \end{stack}
                \begin{stack}
                    `,' \begin{rep}
                            <element>\\
                            `,'
                        \end{rep}\\
                \end{stack}
            `]'\\
            \tok{\color{black} string}
        \end{stack}
    \]]

<attributes> ::= \[[
    `\{' \begin{stack}
            \begin{rep}
                <attribute-name> `:' <attribute-value>\\
                `,'
            \end{rep}\\
        \end{stack} `\}'
    \]]

<attribute-value> ::= \[[
        \begin{stack}
            \tok{\color{black} string}\\
            \tok{\color{black} number}\\
            `true'\\
            `false'\\
            `null'
        \end{stack}
    \]]

        \end{grammar}
    \end{definition}
\end{figure}

\begin{figure}[h]
    \begin{example}[{\acrshort{jsonml}-Dokument}]~
    \inputminted{json}{xmltree.json}
        \captionof{figure}{Die \acrshort{jsonml}-Repräsentation des \acrshort{xml}-Dokuments aus Beispiel~\ref{ex:xmldoc}.}
    \label{fig:xmltreejsonml}
    \end{example}
\end{figure}

\acrshort{jsonml} sieht keine gesonderte Verarbeitung von Namespace-Deklarationen oder mit Namespace-Prefixes versehene Tag-Namen vor, sondern behandelt diese wie normale Attribute bzw. wie einen Teil des Tag-Namens.

\acrshort{jsonml} stellt die \acrshort{xml}-Inhalte recht effizient dar: Die \acrshort{json}-Repräsentation eines umfangreichen Office-Dokuments im \acrshort{fodt}-Format benötigt rund $6{,}6\%$ weniger Speicherplatz als die äquivalente Darstellung durch kanonisches \acrshort{xml}.

Im Vergleich zu anderen Konvertern ist der Overhead bei \acrshort{jsonml} deutlich geringer. So waren die vom Pesterfish-Konverter ausgegebenen Daten trotz Informationsverlust auch nach Entfernung von optionalem Whitespace und unnötiger Quotierung von Unicode-Zeichen mehr als dreieinhalb Mal so groß wie bei \acrshort{jsonml} (vgl. Abb.~\ref{fig:sizecomparison}).

\begin{center}
    \begin{threeparttable}
        \caption{Größenvergleich von ggü. \acrshort{xml} und Pesterfish anhand der Spezifikation des \acrlong{odf} im \acrshort{fodt}-Format.}
        \label{fig:sizecomparison}
        \begin{tabular}{lrrr}
            \toprule
            \rowcolor{white}     & \multicolumn{1}{c}{\fontfamily{rubflama}\selectfont\textbf{Größe (in bytes)}} & \multicolumn{2}{c}{\fontfamily{rubflama}\selectfont\textbf{Verhältnis zu \acrshort{xml} (in \%)}}\\
                                 &                  & \multicolumn{1}{c}{\fontfamily{rubflama}\selectfont\textbf{Größe}}   & \multicolumn{1}{c}{\fontfamily{rubflama}\selectfont\textbf{Veränderung}}\\
            \midrule
            \rowcolor{rubgray!50} Kanonisches \acrshort{xml} &  5787196 & $100{,}0$ &        $0$ \\
                                  \acrshort{jsonml}\tnote{a} &  5405329 &  $93{,}4$ &   $-6{,}6$ \\
            \rowcolor{rubgray!50} Pesterfish\tnote{a}        & 15061634 & $260{,}3$ & $+160{,}3$ \\
                                  Pesterfish\tnote{b}        & 14480612 & $250{,}2$ & $+150{,}2$ \\
            \bottomrule
        \end{tabular}
        \begin{tablenotes}
            \item[a] \acrshort{json} unverändert
            \item[b] Optionaler \acrshort{json}-Whitespace entfernt
        \end{tablenotes}
    \end{threeparttable}
\end{center}

\section{Unterstützung von \acrlongpl{pi}}

Probleme hatte der Konverter jedoch mit der Umwandlung von \acrfullpl{pi}. Diese werdem bei der Umwandlung in \acrshort{json} vollständig ignoriert. Stephen McKamey, der Entwickler von \acrshort{jsonml}, begründet damit, das es keine sinnvolle Entsprechung von \acrshortpl{pi} in \acrshort{json} gäbe~\cite{mckamey2006xml}.

Zwar bietet \acrshort{json} tatsächlich keinen vergleichbaren Mechanismus, eine Unterstützung von \glspl{pi} kann für bestimmte Einsatzzwecke aber sinnvoll sein, beispielsweise wenn sonst die Verknüpfung mit \acrshort{xml}-Stylesheets oder Formatierungsinformationen in DocBook-Dateien verloren gehen könnten. Im Rahmen der vorliegenden Arbeit wurde die \acrshort{jsonml}-Syntax daher um Unterstützung von \acrlongpl{pi} ergänzt.

\glspl{pi} bestehen aus einem \emph{Ziel} und \emph{Daten} (Vgl. Abschn. \ref{sec:xmlbasics}), bilden also das 2-Tupel $P \coloneqq \langle target, data \rangle$. Der Datenteil kann dabei auch leer sein.

Das Ziel muss ein gültiger Name im Sinne der \acrshort{xml}-Spezifikation sein~\cite[{Regel~[17]}]{maler2008xml}. Das heißt, dass der Name einer \gls{pi}\ ebenso wie auch der Tag-Name vom Elementen~\cite[{Regel~[40]}]{maler2008xml} mit einem sog. \texttt{NameStartChar} beginnen muss. Dadurch wird ausgeschlossen, dass Tag-Namen mit bestimmten Zeichen beginnen -- darunter auch das Fragezeichen, da dies dazu führen würde, dass Start-Tags mit \glspl{pi} verwechselt werden könnten. Insbesondere in \acrshort{sgml} -- zu dem \acrshort{xml} vollständig kompatibel sein soll -- wären solche Tags nicht mehr von \glspl{pi} zu unterscheiden, da laut \acrshort{sgml}-Spezifikation lediglich ein einfaches Größerzeichen anstatt der Kombination aus Fragezeichen und Größerzeiche (\texttt{?>}) zum Schließen der \gls{pi} ausreicht.

Dadurch wird es möglich, \glspl{pi} in \acrshort{jsonml} eindeutig in Form eines \acrshort{json}-Arrays \mintinline{json}{["?target", "data"]} darzustellen~(vgl. Definition \ref{def:jsonmlpi}), das dem 2-Tupel $P$ (s.o.) entspricht. Die Repräsentation von \glspl{pi} ähnelt damit der eines Elementknotens, der einen einzelnen Textknoten (\emph{Character Data}) enthält. Eine Verwechslung ist jedoch durch das dem Zielnamen vorangestellte Fragezeichen ausgeschlossen -- ein Tagname darfnicht mit einem Fragezeichen beginnen, wodurch die Kategorisierung als \gls{pi} eindeutig ist.  \begin{figure}[h]
    \begin{definition}[{Formale Syntax der \acrfull{jsonml} mit \emph{\glspl{pi}}}]
        \label{def:jsonmlpi}

        Die um Unterstützung von \emph{\glspl{pi}} erweitere Syntax ist mit Ausname der Produktionsregeln für \synt{element} identisch zu der Syntax aus Definition~\ref{def:jsonml}.
        \synt{tag-name}, \synt{pi-target} und \synt{pi-data} sind \acrshort{json}-Werte vom Typ String.

        \begin{grammar}
            <element> ::= \[[
        \begin{stack}
            `[' \begin{stack}
                    <tag-name>
                    \begin{stack}
                        `,' <attributes>\\
                    \end{stack}
                    \begin{stack}
                        `,' \begin{rep}
                                <element>\\
                                `,'
                            \end{rep}\\
                    \end{stack}\\
                    `?' <pi-target> `,' <pi-value>
                \end{stack} `]'\\
            \tok{\color{black} \acrshort{json}-String}
        \end{stack}
    \]]

        \end{grammar}

        Enthält das Dokument \emph{\glspl{pi}} auf Dokument-Ebene (d.h. als Top-Level-Konstrukt), dann ist das \acrshort{jsonml}-Wurzelelement ein \synt{element} mit einem leeren String als \synt{tag-name}, das die Child-Nodes des Dokuments (d.h. \emph{\glspl{pi}} auf Dokumentebene und das Wurzelelement des Dokuments) als Unterelemente enthält.
    \end{definition}
\end{figure}

\section{Überprüfung der Änderungen}

Die syntaktischen Änderungen aus Definition~\ref{def:jsonmlpi} wurden in die JavaScript-Referenz\-implementierung von Stephen McKamey eingearbeitet. Entsprechende \emph{Unittests} zur Sicherstellung der korrekten Umwandlung von \glspl{pi} wurden ebenfalls hinzugefügt.

\begin{figure}[h!]

    \begin{example}[{\acrshort{jsonml}-Dokument mit \glspl{pi}}]
        Die \acrshort{jsonml}-Repräsentation des \acrshort{xml}-Dokuments aus Beispiel \ref{ex:xmltree} kann nun die \gls{pi} darstellen -- auch solche, die sich außerhalb des Wurzelelements befinden.
        \begin{minted}[autogobble]{json}
            ["", "\n",
                [ "?xml-stylesheet", "href=\"style.css\"" ],"\n","\n",
                ["albums", "\n  ",
                    ["album", {"catno": "ARGO LP-628"}, "\n    ",
                        ["artist", "Ahmad Jamal Trio"], "\n    ",
                        ["title", "At The Pershing"], "\n    ",
                        ["recording", "Recorded ",
                            ["date", "January 16, 1958"], "."
                        ], "\n  "
                    ], "\n"
                ]
            ]
        \end{minted}
    \end{example}
\end{figure}

Bei einer erneuten Überprüfung des \acrshort{jsonml}-Konverters unter Berücksichtigung der o. g. Änderungen wurde deren Korrektheit bestätigt: Alle Testdokumente, auch die zuvor fehlgeschlagenen, lassen sich nun verlustlos von \acrshort{xml} nach \acrshort{json} und wieder zurück konvertieren.

Alle Änderungen wurdem dem \acrshort{jsonml}-Projekt zur Verfügung\footnote{Vgl.~\url{https://github.com/mckamey/jsonml/pull/14}} gestellt (siehe Anhang~\ref{appx:jsonmlpi}).


\chapter{Fazit und Ausblick} \label{chap:conclusion}
Ziel der vorliegenden Arbeit war das Finden eines sicheren und verlustlosen Verfahrens zur Konversion von beliebigen \acrshort{xml}-Dokumenten in \acrshort{json}-Datenstrukturen. Dazu wurden die Anforderungen \enquote{Sicherheit} und \enquote{Verlustlosigkeit} zunächst näher bestimmt und ein Kriterienkatalog für ein sicheres und verlustloses Konversionsverfahren erstellt.

Auf dieser Basis wurden dann eine Reihe von Konversionsprogrammen analysiert. Dazu wurde ein Test-Framework implementiert, das die Durchführung der Überprüfung weitestgehen automatisiert.

Es zeigte sich, das keines der überprüften Konversionsverfahren bisher in der Lage ist, \acrshort{xml}-Strukturen verlustlos abzubilden. Teile der \acrshort{xml}-Spezifikation werden von vielen Konversionsverfahren nicht oder nur ungenügend unterstützt. Darunter sind eher selten genutzte Features wie \glspl{pi}, aber auch grundlegende \acrshort{xml}-Eigenschaften wie die Beibehaltung der Elementreihenfolge oder die Möglichkeit zur Nutzung von Mixed Content.

Die \acrfull{jsonml} erfüllte bei der Analyse die meisten der zuvor aufgestellten Kritieren. Probleme hatte es nur bei der Unterstützung von \acrlongpl{pi}. Durch eine Weiterentwicklung des Konversionsverfahrens konnte dieser Mangel behoben werden, sodass das nun alle Anforderungen vollumfänglich erfüllt werden. Auch die Umwandlung von komplexen \acrshort{xml}-basierten Formaten wie \gls{ooxml}, Flat \gls{odf} oder \gls{svg} in \acrshort{json} ist so verlustlos und sicher möglich.

Das hier gezeigte Prüfverfahren vergleicht die \acrshort{xml}-Dokumente vor und nach einem Konversions-Round-Trip. Viele der Konversionsverfahren sind jedoch für eine Analyse dieser Art schon allein deshalb nicht geeignet, da ihnen die Möglichkeit der Rückkonvertierung zu \acrshort{xml} fehlt. Ohne eine solche Funktionalität ist eine Prüfung der Verlustlosigkeit in der hier beschriebenen Art jedoch nicht möglich. Projekte wie die PHP-Bibliothek \emph{xml2jsonphp} von IBM\footnote{https://www.ibm.com/developerworks/library/x-xml2jsonphp/index.html} oder das JavaScript-Modul \emph{xmlToJSON} von William Summers\footnote{https://github.com/metatribal/xmlToJSON} konnten daher nicht berücksichtigt werden.

Ansatzpunkt weiterer Forschung könnte neben der Evaluierung von weiteren Konversionverfahren und -programmen auch die Einbeziehung zusätzlicher Metadaten wie etwa \acrshortpl{dtd} oder Schema-Informationen sein. Dies könnte Verbesserungen bei dem Verwendung von nativen \acrshort{json}-Datentypen bringen.

Auch die Überprüfung von Verfahren, die beliebige \acrshort{json}-Daten in ein \acrshort{xml}-basiertes Format übersetzen, könnte Gegenstand weiterer Betrachtungen sein, inbesondere im Hinblick auf die Sicherheit gegenüber Angriffen auf \acrshort{xml}-Parser.

Das Aufkommen von Technologien wie \acrshort{json} Schema, \acrshort{json} Reference oder \acrshort{json} Include könnte die Komplexität von \acrshort{json}-Parsern in Zukunft deutlich ansteigen lassen und diese für Angriffe verwundbar machen, für die bislang typischerweise gegen \acrshort{xml} eingesetzt. Eine Evaluierung der Verwundbarkeit von Konvertern gegenüber solchen Angriffen erscheint daher sinnvoll.


%% include appendix

\appendix

% \chapter{Acronyms}
\printabbreviations[title={Abkürzungsverzeichnis}]

%\glossarystyle{index}  % chose style here
%\renewcommand*{\glstreenamefmt}[1]{#1}
%\printglossary[type=main,title={Register}]

\pagestyle{scrplain} % turn off headers and footers
%% generate list of figures, optional, remove it if you do not like it
\listoffigures

\KOMAoptions{open=any} % Plaziert Kapitel auch auf linken Seiten

%% generate list of tables, optional, remove it if you do not like it
\listoftables

%% generate list of algorithms, optional, remove it if you do not like it
\clearpage \phantomsection{}
\addcontentsline{toc}{chapter}{Liste der Beispiele}
\renewcommand{\listtheoremname}{Liste der Beispiele}
\listoftheorems[ignoreall,show={mdexample}]
%
%%% generate list of listings, optional, remove it if you do not like it
%\renewcommand*{\listoflistingscaption}{Liste der Auflistungen}
%\listoflistings{}

%% generate bibliography with bibtex, the bibfile here is "paper.bib"
\flushbottom
\setcounter{biburllcpenalty}{7000}
\setcounter{biburlucpenalty}{8000}
\setlength{\emergencystretch}{3em}
\printbibliography{}

\KOMAoptions{open=right} % Plaziert Kapitel wieder nur auf rechten Seiten

\chapter{Ausgabebeispiele der Konverter}
\label{appx:convexamples}
\begin{figure}[h!]
    \begin{tabular}[t]{cc}
\subfloat[Cobra vs Mongoose]{
    \begin{minipage}[t]{.475\linewidth}
        \inputminted[fontsize=\footnotesize]{json}{examples/cobravsmongoose.json}
    \end{minipage}
} &
\subfloat[GreenCape \acrshort{xml} Converter]{
    \begin{minipage}[t]{.475\linewidth}
    \inputminted[fontsize=\footnotesize]{json}{examples/greencapexml.json}
    \end{minipage}
} \\
\subfloat[Json-lib]{
    \begin{minipage}[t]{.475\linewidth}
    \inputminted[fontsize=\footnotesize]{json}{examples/jsonlib.json}
    \end{minipage}
} &
\subfloat[\acrshort{jsonml}]{
    \begin{minipage}[t]{.475\linewidth}
    \inputminted[fontsize=\footnotesize]{json}{examples/jsonml.json}
    \end{minipage}
}
\end{tabular}
\end{figure}
\begin{figure}[h!]\ContinuedFloat
    \begin{tabular}[t]{cc}
\subfloat[org.json.XML]{
    \begin{minipage}[t]{.475\linewidth}
    \inputminted[fontsize=\footnotesize]{json}{examples/orgjsonxml.json}
    \end{minipage}
} &
\subfloat[x2js (Fork)]{
    \begin{minipage}[t]{.475\linewidth}
    \inputminted[fontsize=\footnotesize]{json}{examples/x2js.json}
    \end{minipage}
}\\
\subfloat[JXON]{
    \begin{minipage}[t]{.475\linewidth}
    \inputminted[fontsize=\footnotesize]{json}{examples/jxon.json}
    \end{minipage}
} &
\subfloat[Json.NET]{
    \begin{minipage}[t]{.475\linewidth}
    \inputminted[fontsize=\footnotesize]{json}{examples/newtonsoftjson.json}
    \end{minipage}
}
\end{tabular}
\end{figure}
\begin{figure}[h!]\ContinuedFloat
    \begin{tabular}[t]{cc}
\multirow{2}{*}[1.5ex]{
\subfloat[Pesterfish]{
    \begin{minipage}[t]{.475\linewidth}
    \inputminted[fontsize=\footnotesize]{json}{examples/pesterfish.json}
    \end{minipage}
}
} &
\subfloat[xmljson (Abdera)]{
    \begin{minipage}[t]{.475\linewidth}
    \inputminted[fontsize=\footnotesize]{json}{examples/xmljson-abdera.json}
    \end{minipage}
}\\
    &
\subfloat[xmljson (Yahoo)]{
    \begin{minipage}[t]{.475\linewidth}
    \inputminted[fontsize=\footnotesize]{json}{examples/xmljson-yahoo.json}
    \end{minipage}
}
\end{tabular}
\end{figure}

\begin{figure}[h!]\ContinuedFloat
    \begin{tabular}[t]{cc}
        \begin{minipage}[t]{.475\linewidth}
\subfloat[xmljson (Badgerfish)]{
    \begin{minipage}{\linewidth}
    \inputminted[fontsize=\footnotesize]{json}{examples/xmljson-badgerfish.json}
    \end{minipage}
}\\
\subfloat[xmljson (GData)]{
    \begin{minipage}{\linewidth}
    \inputminted[fontsize=\footnotesize]{json}{examples/xmljson-gdata.json}
    \end{minipage}
}
\end{minipage}
\hfill{}
\begin{minipage}[t]{.475\linewidth}
\subfloat[xmljson (Cobra)]{
    \begin{minipage}{\linewidth}
    \inputminted[fontsize=\footnotesize]{json}{examples/xmljson-cobra.json}
    \end{minipage}
}\\
\subfloat[xmljson (Parker)]{
    \begin{minipage}{\linewidth}
    \inputminted[fontsize=\footnotesize]{json}{examples/xmljson-parker.json}
    \end{minipage}
}
\end{minipage}
\end{tabular}
\end{figure}

\chapter{Problematische Unicode-Zeichen}
\label{appx:unicode}

\section{Whitespace}
\label{appx:unicode-whitespace}

Bei der Konversion mittels \emph{JXON} und \emph{xmljson} mit \emph{Badgerfish}- oder \emph{GData}-Konversion gehen die folgenden Unicode-Whitespace-Zeichen verloren:

\begin{figure}[h!]
    \begin{center}
        \begingroup
        \footnotesize
        \begin{threeparttable}
        \begin{tabular}{lr}
            \toprule
            {\fontfamily{rubflama}\selectfont\textbf{Name}} & {\fontfamily{rubflama}\selectfont\textbf{Unicode-Codepoint}}\\
            \midrule
            \rowcolor{rubgray!50}\texttt{CHARACTER TABULATION}               & \texttt{000009}\\
                                 \texttt{LINE FEED (LF)}                     & \texttt{00000A}\\
            \rowcolor{rubgray!50}\texttt{CARRIAGE RETURN (CR)}               & \texttt{00000D}\\
                                 \texttt{SPACE}                              & \texttt{000020}\\
            \rowcolor{rubgray!50}\texttt{NEXT LINE (NEL)}\tnote{1}           & \texttt{U+0085}\\
                                 \texttt{NO-BREAK SPACE}                     & \texttt{U+00A0}\\
            \rowcolor{rubgray!50}\texttt{OGHAM SPACE MARK}                   & \texttt{U+1680}\\
                                 \texttt{MONGOLIAN VOWEL SEPARATOR}          & \texttt{U+180E}\\
            \rowcolor{rubgray!50}\texttt{EN QUAD}                            & \texttt{U+2000}\\
                                 \texttt{EM QUAD}                            & \texttt{U+2001}\\
            \rowcolor{rubgray!50}\texttt{EN SPACE}                           & \texttt{U+2002}\\
                                 \texttt{EM SPACE}                           & \texttt{U+2003}\\
            \rowcolor{rubgray!50}\texttt{THREE-PER-EM SPACE}                 & \texttt{U+2004}\\
                                 \texttt{FOUR-PER-EM SPACE}                  & \texttt{U+2005}\\
            \rowcolor{rubgray!50}\texttt{SIX-PER-EM SPACE}                   & \texttt{U+2006}\\
                                 \texttt{FIGURE SPACE}                       & \texttt{U+2007}\\
            \rowcolor{rubgray!50}\texttt{PUNCTUATION SPACE}                  & \texttt{U+2008}\\
                                 \texttt{THIN SPACE}                         & \texttt{U+2009}\\
            \rowcolor{rubgray!50}\texttt{HAIR SPACE}                         & \texttt{U+200A}\\
                                 \texttt{LINE SEPARATOR}                     & \texttt{U+2028}\\
            \rowcolor{rubgray!50}\texttt{PARAGRAPH SEPARATOR}                & \texttt{U+2029}\\
                                 \texttt{NARROW NO-BREAK SPACE}              & \texttt{U+202F}\\
            \rowcolor{rubgray!50}\texttt{MEDIUM MATHEMATICAL SPACE}          & \texttt{U+205F}\\
                                 \texttt{IDEOGRAPHIC SPACE}                  & \texttt{U+3000}\\
            \rowcolor{rubgray!50}\texttt{ZERO WIDTH NO-BREAK SPACE}\tnote{2} & \texttt{U+FEFF}\\
            \bottomrule
            \end{tabular}
            \begin{tablenotes}
                \item[1] Nur bei \emph{xmljson} mit \emph{Badgerfish}- und \emph{GData}-Konvention
                \item[2] Nur bei \emph{JXON}
            \end{tablenotes}
        \end{threeparttable}
        \endgroup
    \end{center}
\end{figure}

\newpage{}
\section{Zahlen}
\label{appx:unicode-digits}

Bei der Konversion mittels \emph{xmljson} mit \emph{Badgerfish}-, \emph{GData}- oder \emph{Parker}-Konvention wurden folgende Unicode-Zahlenbereiche in ihr ASCII-Äquivalent im Bereich \texttt{U+0030} bis \texttt{U+0039} umgewandelt:

\begin{figure}[hb!]
    \begin{center}
        \begingroup
        \footnotesize
        \begin{tabular}{lrr}
            \toprule
            {\fontfamily{rubflama}\selectfont\textbf{Name}} & \multicolumn{2}{c}{\fontfamily{rubflama}\selectfont\textbf{Unicode-Bereich}}\\
                                                            & {\fontfamily{rubflama}\selectfont\textbf{Beginn}} & {\fontfamily{rubflama}\selectfont\textbf{Ende}}\\
            \midrule
\texttt{ARABIC-INDIC DIGITS} & \texttt{U+0669} & \texttt{U+0660}\\
\texttt{EXTENDED ARABIC-INDIC DIGITS} & \texttt{U+06F9} &  \texttt{U+06F0}\\
\texttt{NKO DIGITS} & \texttt{U+07C9} &  \texttt{U+07C0}\\
\texttt{DEVANAGARI DIGITS} & \texttt{U+096F} &  \texttt{U+0966}\\
\texttt{BENGALI DIGITS} & \texttt{U+09EF} &  \texttt{U+09E6}\\
\texttt{GURMUKHI DIGITS} & \texttt{U+0A6F} &  \texttt{U+0A66}\\
\texttt{GUJARATI DIGITS} & \texttt{U+0AEF} &  \texttt{U+0AE6}\\
\texttt{ORIYA DIGITS} & \texttt{U+0B6F} &  \texttt{U+0B66}\\
\texttt{TAMIL DIGITS} & \texttt{U+0BEF} &  \texttt{U+0BE6}\\
\texttt{TELUGU DIGITS} & \texttt{U+0C6F} &  \texttt{U+0C66}\\
\texttt{KANNADA DIGITS} & \texttt{U+0CEF} &  \texttt{U+0CE6}\\
\texttt{MALAYALAM DIGITS} & \texttt{U+0D6F} &  \texttt{U+0D66}\\
\texttt{SINHALA LITH DIGITS} & \texttt{U+0DEF} &  \texttt{U+0DE6}\\
\texttt{THAI DIGITS} & \texttt{U+0E59} &  \texttt{U+0E50}\\
\texttt{LAO DIGITS} & \texttt{U+0ED9} &  \texttt{U+0ED0}\\
\texttt{TIBETAN DIGITS} & \texttt{U+0F29} &  \texttt{U+0F20}\\
\texttt{MYANMAR DIGITS} & \texttt{U+1049} &  \texttt{U+1040}\\
\texttt{MYANMAR SHAN DIGITS} & \texttt{U+1099} &  \texttt{U+1090}\\
\texttt{KHMER DIGITS} & \texttt{U+17E9} &  \texttt{U+17E0}\\
\texttt{MONGOLIAN DIGITS} & \texttt{U+1819} &  \texttt{U+1810}\\
\texttt{LIMBU DIGITS} & \texttt{U+194F} &  \texttt{U+1946}\\
\texttt{NEW TAI LUE DIGITS} & \texttt{U+19D9} &  \texttt{U+19D0}\\
\texttt{TAI THAM HORA DIGITS} & \texttt{U+1A89} &  \texttt{U+1A80}\\
\texttt{TAI THAM THAM DIGITS} & \texttt{U+1A99} &  \texttt{U+1A90}\\
\texttt{BALINESE DIGITS} & \texttt{U+1B59} &  \texttt{U+1B50}\\
\texttt{SUNDANESE DIGITS} & \texttt{U+1BB9} &  \texttt{U+1BB0}\\
\texttt{LEPCHA DIGITS} & \texttt{U+1C49} &  \texttt{U+1C40}\\
\texttt{OL CHIKI DIGITS} & \texttt{U+1C59} &  \texttt{U+1C50}\\
\texttt{VAI DIGITS} & \texttt{U+A629} &  \texttt{U+A620}\\
\texttt{SAURASHTRA DIGITS} & \texttt{U+A8D9} &  \texttt{U+A8D0}\\
\texttt{KAYAH LI DIGITS} & \texttt{U+A909} &  \texttt{U+A900}\\
\texttt{JAVANESE DIGITS} & \texttt{U+A9D9} &  \texttt{U+A9D0}\\
\texttt{MYANMAR TAI LAING DIGITS} & \texttt{U+A9F9} &  \texttt{U+A9F0}\\
\texttt{CHAM DIGITS} & \texttt{U+AA59} &  \texttt{U+AA50}\\
\texttt{MEETEI MAYEK DIGITS} & \texttt{U+ABF9} &  \texttt{U+ABF0}\\
\texttt{FULLWIDTH DIGITS} & \texttt{U+FF19} &  \texttt{U+FF10}\\
            \bottomrule
        \end{tabular}
        \endgroup
    \end{center}
\end{figure}

\chapter{Patches}
\label{appx:patches}

\section{Entfernung des Whitespace im GreenCape XML Konverter}
\label{appx:greencapexml}

\inputminted[breakautoindent=false,fontsize=\footnotesize]{udiff}{patches/greencapexml-noindent.patch}

\newpage{}
\section{Unterstützung für \acrshortpl{pi} in \acrshort{jsonml}}
\label{appx:jsonmlpi}

\inputminted[breakautoindent=false,fontsize=\footnotesize]{udiff}{patches/jsonml-pi.patch}


\end{document}
